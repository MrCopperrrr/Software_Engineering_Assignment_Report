\part*{5. Working Demonstration}
\addcontentsline{toc}{part}{5. Working Demonstration}

Minh họa quá trình hoạt động của hệ thống MentorLink, trình bày 1 chuỗi các màn hình của dự án thể hiện các chức năng chính mà nhóm đã thực hiện.
%========================================================================================
\section*{5.1. Luồng sử dụng của sinh viên}
\addcontentsline{toc}{section}{5.1. Luồng sử dụng của sinh viên}
\subsection*{5.1.1. Đăng ký/Đăng nhập tài khoản}
Khi sinh viên đăng nhập vào hệ thống MentorLink. Nếu chưa có tài khoản thì sinh viên sẽ được đăng ký mới tài khoản. Sinh viên sẽ điền tên người dùng, Email, mật khẩu, chọn vai trò người dùng. Sau đó chọn tạo tài khoản.
\begin{figure}[H]
    \centering
    \setlength{\fboxsep}{2pt}     
    \setlength{\fboxrule}{0.5pt}   
    \fbox{\includegraphics[scale=0.3]{Picture/Demo/signin.png}}
    \caption{Đăng ký tài khoản}
\end{figure}
Trường hợp sinh viên đã có tài khoản thì sẽ đăng nhập vào hệ thống
\begin{figure}[H]
    \centering
    \setlength{\fboxsep}{2pt}     
    \setlength{\fboxrule}{0.5pt}   
    \fbox{\includegraphics[scale=0.3]{Picture/Demo/login.png}}
    \caption{Đăng nhập tài khoản}
\end{figure}


\subsection*{5.1.2. Xác thực Email}
Sau khi sinh viên tạo tài khoản xong, ngay lập thức được hiển thị thông báo xác thực Email. Sinh viên sẽ kiểm tra hộp mail cá nhân để nhập mã OTP gửi về. Sau đó chọn xác nhận để hoàn thành thủ tục đăng ký tài khoản.

\begin{figure}[H]
    \centering
    \setlength{\fboxsep}{2pt}     
    \setlength{\fboxrule}{0.5pt}   
    \fbox{\includegraphics[scale=0.3]{Picture/Demo/otp.png}}
    \caption{Xác thực Email bằng OTP}
\end{figure}

\subsection*{5.1.3. Trang chủ}
Sau khi hoàn thành thủ tục đăng ký hoặc đăng nhập thành công. Sinh viên sẽ vào được trang chủ của hệ thống. Ở đây có các chức năng để sinh viên chọn lựa sử dung.

\begin{figure}[H]
    \centering
    \setlength{\fboxsep}{2pt}     
    \setlength{\fboxrule}{0.5pt}   
    \fbox{\includegraphics[scale=0.3]{Picture/Demo/homepage.png}}
    \caption{Trang chủ của sinh viên}
\end{figure}

\subsection*{5.1.4. Cập nhật hồ sơ}
Sinh viên cần phải cập nhật hồ sơ cá nhân của mình bằng cách chọn vào biểu tượng hình người ở góc phải phía trên. Giao diện cập nhât hồ sơ sẽ được hiển thị.
\begin{figure}[H]
    \centering
    \setlength{\fboxsep}{2pt}     
    \setlength{\fboxrule}{0.5pt}   
    \fbox{\includegraphics[scale=0.3]{Picture/Demo/profilepage.png}}
    \caption{Giao diện hồ sơ của sinh viên}
\end{figure}

Sinh viên chọn chỉnh sửa hồ sơ để bổ sung các thông tin cần thiết. Sinh viên cập nhật Mã số sinh viên (chỉ được 1 lần duy nhất), Số điện thoại, Khoa, và Mô tả năng lực.
\begin{figure}[H]
    \centering
    \setlength{\fboxsep}{2pt}     
    \setlength{\fboxrule}{0.5pt}   
    \fbox{\includegraphics[scale=0.3]{Picture/Demo/profilepageupdating.png}}
    \caption{Cập nhật hồ sơ mới}
\end{figure}

Sau đó chọn lưu lại. Hồ sơ của sinh viên đã được cập nhật mới.
\begin{figure}[H]
    \centering
    \setlength{\fboxsep}{2pt}     
    \setlength{\fboxrule}{0.5pt}   
    \fbox{\includegraphics[scale=0.3]{Picture/Demo/profilepageupdated.png}}
    \caption{Sau khi cập nhật hồ sơ}
\end{figure}

\subsection*{5.1.5. Đăng ký môn học}
Sinh viên có thể đăng ký môn học mà mình muốn học, sau đó chọn đăng ký. 

\begin{figure}[H]
    \centering
    \setlength{\fboxsep}{2pt}     
    \setlength{\fboxrule}{0.5pt}   
    \fbox{\includegraphics[scale=0.3]{Picture/Demo/courseregis.png}}
    \caption{Giao diện đăng ký môn học}
\end{figure}

Môn học đucợ sinh viên đăng ký sẽ hiển thị ở cột môn học đã đăng ký. Sinh viên có các lựa chọn hiển thị chi tiết môn học, hủy môn học nếu không muốn học hoặc đăng ký nhầm, tìm tutor phù hợp với môn học.
\begin{figure}[H]
    \centering
    \setlength{\fboxsep}{2pt}     
    \setlength{\fboxrule}{0.5pt}   
    \fbox{\includegraphics[scale=0.3]{Picture/Demo/courseregising.png}}
    \caption{Giao diện đăng ký môn học}
\end{figure}

Sinh viên chọn chi tiết môn học để hiển thị các thông tin khác như nội dung môn học và tài liệu tham khảo.
\begin{figure}[H]
    \centering
    \setlength{\fboxsep}{2pt}     
    \setlength{\fboxrule}{0.5pt}   
    \fbox{\includegraphics[scale=0.3]{Picture/Demo/courseregisdetail.png}}
    \caption{Giao diện chi tiết môn học}
\end{figure}

\subsection*{5.1.6. Tìm Tutor}
Sinh viên chọn chức năng tìm tutor để tìm tutor phù hợp với mình. Ở giao diện này cũng hiển thị các môn học mà sinh viên đã đăng ký.
\begin{figure}[H]
    \centering
    \setlength{\fboxsep}{2pt}     
    \setlength{\fboxrule}{0.5pt}   
    \fbox{\includegraphics[scale=0.3]{Picture/Demo/findtutor.png}}
    \caption{Giao diện chi tiết môn học}
\end{figure}

Sinh viên lựa chọn môn học, sau đó danh sách tutor sẽ hiện thị theo từng môn học. Sinh viên có thể chọn ghép thủ công hoặc tự động. Mặc định sinh viên sẽ được chọn thủ công. 
\begin{figure}[H]
    \centering
    \setlength{\fboxsep}{2pt}     
    \setlength{\fboxrule}{0.5pt}   
    \fbox{\includegraphics[scale=0.3]{Picture/Demo/tutorfound.png}}
    \caption{Giao diện chi tiết môn học}
\end{figure}

Nếu sinh viên chọn ghép tự động, hệ thống sẽ tự ghép cặp với tutor phù hợp. Sinh viên có thể đồng ý hoặc chọn lại.
\begin{figure}[H]
    \centering
    \setlength{\fboxsep}{2pt}     
    \setlength{\fboxrule}{0.5pt}   
    \fbox{\includegraphics[scale=0.3]{Picture/Demo/findtutordetail.png}}
    \caption{Giao diện chi tiết môn học}
\end{figure}
%========================================================================================
\section*{5.2. Luồng sử dụng của Tutor}
\addcontentsline{toc}{section}{5.1. Luồng sử dụng của Tutor}

\subsection*{5.2.1. }

\subsection*{5.2.2. }

\subsection*{5.2.3. }




% \begin{figure}[H]
%     \centering
%     \setlength{\fboxsep}{2pt}     
%     \setlength{\fboxrule}{0.5pt}   
%     \fbox{\includegraphics[scale=0.45]{Picture/hcmut.png}}
%     \caption{Sơ đồ tổ chức các gói của hệ thống}
% \end{figure}