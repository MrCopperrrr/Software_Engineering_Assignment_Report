\part*{1. Mô hình hóa hệ thống}
\addcontentsline{toc}{part}{1. Mô hình hóa hệ thống}
%========================================================================================
\section*{1.1. Sơ đồ hoạt động và sơ đồ tuần tự}
\addcontentsline{toc}{section}{1.1. Sơ đồ hoạt động và sơ đồ tuần tự}
Phần này trình bày chi tiết các quy trình nghiệp vụ của hệ thống thông qua việc mô hình hóa từng Use Case. Đối với mỗi Use Case, sẽ được trực quan hóa bằng 2 sơ đồ Activity Diagram và Sequence Diagram.\\
\textbf{Sơ đồ Hoạt động (Activity Diagram):} Tập trung mô tả luồng công việc tổng quan, các bước xử lý, các điểm quyết định và phân định rõ trách nhiệm của từng tác nhân tham gia vào quy trình.\\
\textbf{Sơ đồ Tuần tự (Sequence Diagram):} Đi sâu vào chi tiết kỹ thuật, mô tả sự tương tác và các thông điệp được trao đổi giữa các thành phần của hệ thống (người dùng, giao diện, server, database) theo đúng thứ tự thời gian.\\
Đường dẫn: \href{https://ntpdeveloper-my.sharepoint.com/:f:/g/personal/ntp0802_ntpdeveloper_onmicrosoft_com/El5BwJ6g3lRDh9b6-NjF5gUByR29AVHc170IPoQx_BTKDg?e=x0MAH0}{Activity and Sequence Diagram}
%========================================================================================
\subsection*{1.1.1. Use Case 01: Đăng ký tài khoản}
\addcontentsline{toc}{subsection}{1.1.1. Use Case 01: Đăng ký tài khoản}
Để bắt đầu sử dụng hệ thống, sinh viên và Tutor cần tạo một tài khoản cá nhân. Hệ thống sẽ hướng dẫn người dùng qua các bước nhập thông tin, đồng thời kiểm tra để đảm bảo dữ liệu là chính xác và duy nhất. Điểm nhấn của quy trình là bước xác thực bằng mã OTP qua email, một lớp bảo mật quan trọng giúp xác minh danh tính và bảo vệ tài khoản ngay từ đầu.
\begin{itemize}
    \item Sơ đồ hoạt động
    \begin{figure}[H]
    \centering
    \includegraphics[scale=0.35 ]{Picture/ACUC01.png}
    \caption{Sơ đồ hoạt động Use Case 01: Đăng ký tài khoản}
    \end{figure}
    \pagebreak
    \item Sơ đồ tuần tự
    \begin{figure}[H]
    \centering
    \includegraphics[scale=0.35 ]{Picture/SEUC01.png}
    \caption{Sơ đồ tuần tự Use Case 01: Đăng ký tài khoản}
    \end{figure}
\end{itemize}
%========================================================================================
\pagebreak
\subsection*{1.1.2. Use Case 02: Đăng nhập}
\addcontentsline{toc}{subsection}{1.1.2. Use Case 02: Đăng nhập}
Khi cần truy cập vào các chức năng của hệ thống, người dùng sẽ sử dụng thông tin đã đăng ký để đăng nhập. Hệ thống sẽ xác thực thông tin này một cách nhanh chóng và kiểm tra trạng thái của tài khoản. Để tăng cường bảo mật, hệ thống cũng được thiết kế để tự động khóa tạm thời tài khoản nếu phát hiện có dấu hiệu đăng nhập sai quá nhiều lần.
\begin{itemize}
    \item Sơ đồ hoạt động
    \begin{figure}[H]
    \centering
    \includegraphics[scale=0.28 ]{Picture/ACUC02.png}
    \caption{Sơ đồ hoạt động Use Case 02: Đăng nhập}
    \end{figure}
    \item Sơ đồ tuần tự
    \begin{figure}[H]
    \centering
    \includegraphics[scale=0.35 ]{Picture/SEUC02.png}
    \caption{Sơ đồ tuần tự Use Case 02: Đăng nhập}
    \end{figure}
\end{itemize}
%========================================================================================
\pagebreak
\subsection*{1.1.3. Use Case 03: Cập nhật hồ sơ}
\addcontentsline{toc}{subsection}{1.1.3. Use Case 03: Cập nhật hồ sơ}
Để đảm bảo thông tin cá nhân luôn chính xác, người dùng có thể dễ dàng chỉnh sửa hồ sơ của mình bất cứ lúc nào. Chức năng này cho phép họ cập nhật các thông tin như số điện thoại, email hay chuyên ngành. Hệ thống sẽ kiểm tra tính hợp lệ của dữ liệu mới trước khi lưu lại, giúp hồ sơ luôn được duy trì một cách toàn vẹn.
\begin{itemize}
    \item Sơ đồ hoạt động
    \begin{figure}[H]
    \centering
    \includegraphics[scale=0.32 ]{Picture/ACUC03.png}
    \caption{Sơ đồ hoạt động Use Case 03: Cập nhật hồ sơ}
    \end{figure}
    \item Sơ đồ tuần tự
    \begin{figure}[H]
    \centering
    \includegraphics[scale=0.35 ]{Picture/SEUC03.png}
    \caption{Sơ đồ tuần tự Use Case 03: Cập nhật hồ sơ}
    \end{figure}
\end{itemize}
%========================================================================================
\pagebreak
\subsection*{1.1.4. Use Case 04: Đăng ký môn học}
\addcontentsline{toc}{subsection}{1.1.4. Use Case 04: Đăng ký môn học}
Khi có nhu cầu cần hỗ trợ, sinh viên có thể tìm và đăng ký các môn học ngay trên hệ thống. Hệ thống sẽ hiển thị một danh sách các môn học khả dụng, với điều kiện là các môn này đã có Tutor sẵn sàng giảng dạy. Không chỉ vậy, hệ thống còn giúp sinh viên quản lý kế hoạch học tập bằng cách giới hạn số môn có thể đăng ký cùng lúc.
\begin{itemize}
    \item Sơ đồ hoạt động
    \begin{figure}[H]
    \centering
    \includegraphics[scale=0.32 ]{Picture/ACUC04.png}
    \caption{Sơ đồ hoạt động Use Case 04: Đăng ký môn học}
    \end{figure}
    \pagebreak
    \item Sơ đồ tuần tự
    \begin{figure}[H]
    \centering
    \includegraphics[scale=0.35 ]{Picture/SEUC04.png}
    \caption{Sơ đồ tuần tự Use Case 04: Đăng ký môn học}
    \end{figure}
\end{itemize}
%========================================================================================
\pagebreak
\subsection*{1.1.5. Use Case 05: Hủy đăng ký môn học}
\addcontentsline{toc}{subsection}{1.1.5. Use Case 05: Hủy đăng ký môn học}
Nếu kế hoạch thay đổi, sinh viên hoàn toàn có thể chủ động hủy đăng ký một môn học không còn phù hợp. Hệ thống sẽ xử lý yêu cầu này sau khi kiểm tra các điều kiện, chẳng hạn như việc hủy phải được thực hiện trước một thời hạn nhất định. Điều này mang lại sự linh hoạt cho sinh viên và giải phóng suất học cho người khác.
\begin{itemize}
    \item Sơ đồ hoạt động
    \begin{figure}[H]
    \centering
    \includegraphics[scale=0.3 ]{Picture/ACUC05.png}
    \caption{Sơ đồ hoạt động Use Case 05: Hủy đăng ký môn học}
    \end{figure}
    \pagebreak
    \item Sơ đồ tuần tự
    \begin{figure}[H]
    \centering
    \includegraphics[scale=0.35 ]{Picture/SEUC05.png}
    \caption{Sơ đồ tuần tự Use Case 05: Hủy đăng ký môn học}
    \end{figure}
\end{itemize}
%========================================================================================
\pagebreak
\subsection*{1.1.6. Use Case 06: Ghép thủ công (SV chọn Tutor)}
\addcontentsline{toc}{subsection}{1.1.6. Use Case 06: Ghép thủ công (SV chọn Tutor)}
Chức năng này cho sinh viên chủ động trong việc tìm kiếm người đồng hành học tập. Dựa trên môn học đã chọn, hệ thống sẽ gợi ý một danh sách các Tutor phù hợp về chuyên môn và lịch trình. Sinh viên có thể xem qua thông tin và lựa chọn Tutor mà mình tin tưởng nhất, miễn là Tutor đó vẫn còn suất trống.
\begin{itemize}
    \item Sơ đồ hoạt động
    \begin{figure}[H]
    \centering
    \includegraphics[scale=0.33 ]{Picture/ACUC06.png}
    \caption{Sơ đồ hoạt động Use Case 06: Ghép thủ công (SV chọn Tutor)}
    \end{figure}
    \pagebreak
    \item Sơ đồ tuần tự
    \begin{figure}[H]
    \centering
    \includegraphics[scale=0.33 ]{Picture/SEUC06.png}
    \caption{Sơ đồ tuần tự Use Case 06: Ghép thủ công (SV chọn Tutor)}
    \end{figure}
\end{itemize}
%========================================================================================
\pagebreak
\subsection*{1.1.7. Use Case 07: Ghép tự động (Hệ thống đề xuất Tutor)}
\addcontentsline{toc}{subsection}{1.1.7. Use Case 07: Ghép tự động (Hệ thống đề xuất Tutor)}
Để tiết kiệm thời gian và công sức cho sinh viên, hệ thống cung cấp một cơ chế ghép cặp thông minh. Sinh viên chỉ cần đưa ra các tiêu chí mong muốn, chẳng hạn như khung giờ học, và hệ thống sẽ tự động phân tích để tìm ra Tutor phù hợp nhất. Đây là giải pháp tối ưu giúp kết nối nhanh chóng và hiệu quả.
\begin{itemize}
    \item Sơ đồ hoạt động
    \begin{figure}[H]
    \centering
    \includegraphics[scale=0.35 ]{Picture/ACUC07.png}
    \caption{Sơ đồ hoạt động Use Case 07: Ghép tự động (Hệ thống đề xuất Tutor)}
    \end{figure}
    \pagebreak
    \item Sơ đồ tuần tự
    \begin{figure}[H]
    \centering
    \includegraphics[scale=0.35 ]{Picture/SEUC07.png}
    \caption{Sơ đồ tuần tự Use Case 07: Ghép tự động (Hệ thống đề xuất Tutor)}
    \end{figure}
\end{itemize}
%========================================================================================
\pagebreak
\subsection*{1.1.8. Use Case 08: Tạo lịch rảnh (Tutor)}
\addcontentsline{toc}{subsection}{1.1.8. Use Case 08: Tạo lịch rảnh (Tutor)}
Để hệ thống có thể sắp xếp các buổi học, Tutor cần cung cấp thông tin về các khung giờ mình có thể giảng dạy. Chức năng này cho phép Tutor dễ dàng thêm, sửa hoặc xóa các créneaux thời gian rảnh của mình. Dữ liệu này sẽ là nền tảng để sinh viên có thể tìm và đặt lịch học phù hợp.
\begin{itemize}
    \item Sơ đồ hoạt động
    \begin{figure}[H]
    \centering
    \includegraphics[scale=0.3 ]{Picture/ACUC08.png}
    \caption{Sơ đồ hoạt động Use Case 08: Tạo lịch rảnh (Tutor)}
    \end{figure}
    \pagebreak
    \item Sơ đồ tuần tự
    \begin{figure}[H]
    \centering
    \includegraphics[scale=0.35 ]{Picture/SEUC08.png}
    \caption{Sơ đồ tuần tự Use Case 08: Tạo lịch rảnh (Tutor)}
    \end{figure}
\end{itemize}
%========================================================================================
\pagebreak
\subsection*{1.1.9. Use Case 09: Đặt lịch học (SV)}
\addcontentsline{toc}{subsection}{1.1.9. Use Case 09: Đặt lịch học (SV)}
Sau khi đã được ghép cặp, sinh viên có thể tiến hành đặt một lịch học cố định cho suốt môn học. Hệ thống sẽ hiển thị các khung giờ còn trống của Tutor để sinh viên lựa chọn. Để tránh xung đột, hệ thống cũng sẽ kiểm tra và đảm bảo lịch học mới không bị trùng với các lịch trình khác của sinh viên.
\begin{itemize}
    \item Sơ đồ hoạt động
    \begin{figure}[H]
    \centering
    \includegraphics[scale=0.3 ]{Picture/ACUC09.png}
    \caption{Sơ đồ hoạt động Use Case 09: Đặt lịch học (SV)}
    \end{figure}
    \pagebreak
    \item Sơ đồ tuần tự
    \begin{figure}[H]
    \centering
    \includegraphics[scale=0.35 ]{Picture/SEUC09.png}
    \caption{Sơ đồ tuần tự Use Case 09: Đặt lịch học (SV)}
    \end{figure}
\end{itemize}
%========================================================================================
\pagebreak
\subsection*{1.1.10. Use Case 10: Hủy/Đổi lịch học cố định}
\addcontentsline{toc}{subsection}{1.1.10. Use Case 10: Hủy/Đổi lịch học cố định}
Cuộc sống luôn có những thay đổi, vì vậy hệ thống cho phép sinh viên linh hoạt điều chỉnh lịch học cố định của mình. Sinh viên có thể chọn hủy hoàn toàn lịch của một môn hoặc đổi sang một khung giờ khác phù hợp hơn. Mọi thay đổi đều được xử lý và thông báo một cách minh bạch đến các bên.
\begin{itemize}
    \item Sơ đồ hoạt động
    \begin{figure}[H]
    \centering
    \includegraphics[scale=0.28 ]{Picture/ACUC10.png}
    \caption{Sơ đồ hoạt động Use Case 10: Hủy/Đổi lịch học cố định}
    \end{figure}
    \pagebreak
    \item Sơ đồ tuần tự
    \begin{figure}[H]
    \centering
    \includegraphics[scale=0.35 ]{Picture/SEUC10.png}
    \caption{Sơ đồ tuần tự Use Case 10: Hủy/Đổi lịch học cố định}
    \end{figure}
\end{itemize}
%========================================================================================
\pagebreak
\subsection*{1.1.11. Use Case 11: Gửi thông báo lịch học}
\addcontentsline{toc}{subsection}{1.1.11. Use Case 11: Gửi thông báo lịch học}
Mọi thay đổi về lịch học, dù là đặt mới, hủy hay điều chỉnh, đều cần được thông báo ngay lập tức. Vì vậy, hệ thống được thiết kế để tự động nhận diện các sự kiện này và ngay lập tức tạo ra thông báo. Thông báo này sẽ được gửi đồng thời đến cả sinh viên và Tutor, đảm bảo không ai bỏ lỡ thông tin quan trọng.
\begin{itemize}
    \item Sơ đồ hoạt động
    \begin{figure}[H]
    \centering
    \includegraphics[scale=0.5 ]{Picture/ACUC11.png}
    \caption{Sơ đồ hoạt động Use Case 11: Gửi thông báo lịch học}
    \end{figure}
    \item Sơ đồ tuần tự
    \begin{figure}[H]
    \centering
    \includegraphics[scale=0.4 ]{Picture/SEUC11.png}
    \caption{Sơ đồ tuần tự Use Case 11: Gửi thông báo lịch học}
    \end{figure}
\end{itemize}
%========================================================================================
\pagebreak
\subsection*{1.1.12. Use Case 12: Gửi nhắc nhở buổi học}
\addcontentsline{toc}{subsection}{1.1.12. Use Case 12: Gửi nhắc nhở buổi học}
Để giúp sinh viên và Tutor không bỏ lỡ các buổi học đã lên kế hoạch, hệ thống sẽ tự động gửi lời nhắc một cách thông minh. Các nhắc nhở này được gửi vào những thời điểm hợp lý, chẳng hạn như trước 24 giờ và 1 giờ, giúp mọi người luôn chủ động và chuẩn bị tốt nhất cho buổi học.
\begin{itemize}
    \item Sơ đồ hoạt động
    \begin{figure}[H]
    \centering
    \includegraphics[scale=0.4 ]{Picture/ACUC12.png}
    \caption{Sơ đồ hoạt động Use Case 12: Gửi nhắc nhở buổi học}
    \end{figure}
    \item Sơ đồ tuần tự
    \begin{figure}[H]
    \centering
    \includegraphics[scale=0.4 ]{Picture/SEUC12.png}
    \caption{Sơ đồ tuần tự Use Case 12: Gửi nhắc nhở buổi học}
    \end{figure}
\end{itemize}
%========================================================================================
\pagebreak
\subsection*{1.1.13. Use Case 13: Điểm danh sinh viên}
\addcontentsline{toc}{subsection}{1.1.13. Use Case 13: Điểm danh sinh viên}
Việc ghi nhận sự tham gia của sinh viên là một phần quan trọng của quá trình học. Chức năng này cho phép Tutor dễ dàng điểm danh ngay trên hệ thống trong mỗi buổi học. Dữ liệu điểm danh không chỉ giúp theo dõi chuyên cần mà còn là cơ sở cho các báo cáo học tập sau này.
\begin{itemize}
    \item Sơ đồ hoạt động
    \begin{figure}[H]
    \centering
    \includegraphics[scale=0.35 ]{Picture/ACUC13.png}
    \caption{Sơ đồ hoạt động Use Case 13: Điểm danh sinh viên}
    \end{figure}
    \pagebreak
    \item Sơ đồ tuần tự
    \begin{figure}[H]
    \centering
    \includegraphics[scale=0.4 ]{Picture/SEUC13.png}
    \caption{Sơ đồ tuần tự Use Case 13: Điểm danh sinh viên}
    \end{figure}
\end{itemize}
%========================================================================================
\pagebreak
\subsection*{1.1.14. Use Case 14: Cập nhật trạng thái buổi học}
\addcontentsline{toc}{subsection}{1.1.14. Use Case 14: Cập nhật trạng thái buổi học}
Mỗi buổi học sẽ có một vòng đời riêng, từ "Sắp diễn ra" cho đến "Đã hoàn tất". Hệ thống cho phép Tutor cập nhật trạng thái này một cách thủ công, đồng thời cũng có thể tự động thay đổi dựa trên thời gian thực. Điều này giúp tình trạng của tất cả các buổi học luôn được phản ánh một cách chính xác.
\begin{itemize}
    \item Sơ đồ hoạt động
    \begin{figure}[H]
    \centering
    \includegraphics[scale=0.35 ]{Picture/ACUC14.png}
    \caption{Sơ đồ hoạt động Use Case 14: Cập nhật trạng thái buổi học}
    \end{figure}
    \item Sơ đồ tuần tự
    \begin{figure}[H]
    \centering
    \includegraphics[scale=0.35 ]{Picture/SEUC14.png}
    \caption{Sơ đồ tuần tự Use Case 14: Cập nhật trạng thái buổi học}
    \end{figure}
\end{itemize}
%========================================================================================
\pagebreak
\subsection*{1.1.15. Use Case 15: Quản lý tài liệu (Tutor)}
\addcontentsline{toc}{subsection}{1.1.15. Use Case 15: Quản lý tài liệu (Tutor)}
Để hỗ trợ tốt nhất cho sinh viên, Tutor có thể tải lên và quản lý các tài liệu học tập liên quan đến môn học. Chức năng này cho phép họ dễ dàng chia sẻ slide, bài tập hay tài liệu tham khảo. Tutor toàn quyền kiểm soát các tài liệu do mình đăng tải, bao gồm cả việc chỉnh sửa và xóa bỏ.
\begin{itemize}
    \item Sơ đồ hoạt động
    \begin{figure}[H]
    \centering
    \includegraphics[scale=0.28 ]{Picture/ACUC15.png}
    \caption{Sơ đồ hoạt động Use Case 15: Quản lý tài liệu (Tutor)}
    \end{figure}
    \pagebreak
    \item Sơ đồ tuần tự
    \begin{figure}[H]
    \centering
    \includegraphics[scale=0.35 ]{Picture/SEUC15.png}
    \caption{Sơ đồ tuần tự Use Case 15: Quản lý tài liệu (Tutor)}
    \end{figure}
\end{itemize}
%========================================================================================
\pagebreak
\subsection*{1.1.16. Use Case 16: SV tải tài liệu}
\addcontentsline{toc}{subsection}{1.1.16. Use Case 16: SV tải tài liệu}
Sinh viên có thể dễ dàng truy cập vào kho tài liệu mà Tutor đã chia sẻ cho môn học của mình. Để đảm bảo tài liệu được chia sẻ đúng đối tượng, hệ thống sẽ kiểm tra quyền truy cập trước khi cho phép tải xuống. Mỗi lượt tải cũng sẽ được ghi nhận lại để phục vụ cho việc thống kê.
\begin{itemize}
    \item Sơ đồ hoạt động
    \begin{figure}[H]
    \centering
    \includegraphics[scale=0.33 ]{Picture/ACUC16.png}
    \caption{Sơ đồ hoạt động Use Case 16: SV tải tài liệu}
    \end{figure}
    \pagebreak
    \item Sơ đồ tuần tự
    \begin{figure}[H]
    \centering
    \includegraphics[scale=0.33 ]{Picture/SEUC16.png}
    \caption{Sơ đồ tuần tự Use Case 16: SV tải tài liệu}
    \end{figure}
\end{itemize}
%========================================================================================
\pagebreak
\subsection*{1.1.17. Use Case 17: SV đánh giá Tutor}
\addcontentsline{toc}{subsection}{1.1.17. Use Case 17: SV đánh giá Tutor}
Phản hồi từ người học là nguồn thông tin vô giá. Sau khi kết thúc môn học, sinh viên được khuyến khích đưa ra những đánh giá về chất lượng giảng dạy của Tutor. Những góp ý này không chỉ giúp Tutor cải thiện mà còn cung cấp cho Khoa/BM cái nhìn sâu sắc về hiệu quả của chương trình.
\begin{itemize}
    \item Sơ đồ hoạt động
    \begin{figure}[H]
    \centering
    \includegraphics[scale=0.3 ]{Picture/ACUC17.png}
    \caption{Sơ đồ hoạt động Use Case 17: SV đánh giá Tutor}
    \end{figure}
    \pagebreak
    \item Sơ đồ tuần tự
    \begin{figure}[H]
    \centering
    \includegraphics[scale=0.35 ]{Picture/SEUC17.png}
    \caption{Sơ đồ tuần tự Use Case 17: SV đánh giá Tutor}
    \end{figure}
\end{itemize}
%========================================================================================
\pagebreak
\subsection*{1.1.18. Use Case 18: Tutor đánh giá sinh viên}
\addcontentsline{toc}{subsection}{1.1.18. Use Case 18: Tutor đánh giá sinh viên}
Tương tự, Tutor cũng có thể đưa ra những nhận xét về quá trình học tập của sinh viên. Những đánh giá này tập trung vào thái độ, sự chuyên cần và tiến bộ của người học. Đây là một kênh thông tin tham khảo quan trọng, được bảo mật và chỉ dành cho cấp quản lý.
\begin{itemize}
    \item Sơ đồ hoạt động
    \begin{figure}[H]
    \centering
    \includegraphics[scale=0.3 ]{Picture/ACUC18.png}
    \caption{Sơ đồ hoạt động Use Case 18: Tutor đánh giá sinh viên}
    \end{figure}
    \pagebreak
    \item Sơ đồ tuần tự
    \begin{figure}[H]
    \centering
    \includegraphics[scale=0.4 ]{Picture/SEUC18.png}
    \caption{Sơ đồ tuần tự Use Case 18: Tutor đánh giá sinh viên}
    \end{figure}
\end{itemize}
%========================================================================================
\pagebreak
\subsection*{1.1.19. Use Case 19: Khoa/BM tổng hợp đánh giá}
\addcontentsline{toc}{subsection}{1.1.19. Use Case 19: Khoa/BM tổng hợp đánh giá}
Để có một cái nhìn toàn diện, Khoa/BM có thể truy cập vào một báo cáo tổng hợp về tất cả các đánh giá hai chiều. Hệ thống sẽ tự động tính toán các chỉ số quan trọng và trình bày chúng một cách trực quan. Công cụ này giúp nhà quản lý dễ dàng nắm bắt và phân tích chất lượng tương tác trong toàn hệ thống.
\begin{itemize}
    \item Sơ đồ hoạt động
    \begin{figure}[H]
    \centering
    \includegraphics[scale=0.35 ]{Picture/ACUC19.png}
    \caption{Sơ đồ hoạt động Use Case 19: Khoa/BM tổng hợp đánh giá}
    \end{figure}
    \item Sơ đồ tuần tự
    \begin{figure}[H]
    \centering
    \includegraphics[scale=0.45 ]{Picture/SEUC19.png}
    \caption{Sơ đồ tuần tự Use Case 19: Khoa/BM tổng hợp đánh giá}
    \end{figure}
\end{itemize}
%========================================================================================
\pagebreak
\subsection*{1.1.20. Use Case 20: Báo cáo kết quả học tập SV}
\addcontentsline{toc}{subsection}{1.1.20. Use Case 20: Báo cáo kết quả học tập SV}
Vào cuối mỗi kỳ, hệ thống sẽ tổng hợp dữ liệu học tập của từng sinh viên thành một báo cáo sơ bộ. Báo cáo này sau đó sẽ được chuyển đến Khoa để xem xét và xác nhận. Một khi đã được phê duyệt, sinh viên có thể truy cập để xem kết quả chính thức của mình.
\begin{itemize}
    \item Sơ đồ hoạt động
    \begin{figure}[H]
    \centering
    \includegraphics[scale=0.35 ]{Picture/ACUC20.png}
    \caption{Sơ đồ hoạt động Use Case 20: Báo cáo kết quả học tập SV}
    \end{figure}
    \pagebreak
    \item Sơ đồ tuần tự
    \begin{figure}[H]
    \centering
    \includegraphics[scale=0.4 ]{Picture/SEUC20.png}
    \caption{Sơ đồ tuần tự Use Case 20: Báo cáo kết quả học tập SV}
    \end{figure}
\end{itemize}
%========================================================================================
\pagebreak
\subsection*{1.1.21. Use Case 21: Báo cáo chất lượng Tutor}
\addcontentsline{toc}{subsection}{1.1.21. Use Case 21: Báo cáo chất lượng Tutor}
Tương tự, chất lượng giảng dạy của mỗi Tutor cũng được tổng hợp thành một báo cáo chi tiết, dựa trên dữ liệu buổi học và phản hồi từ sinh viên. Báo cáo này cũng cần được Khoa xác nhận trước khi được gửi đến Tutor. Đây là cơ sở để ghi nhận và đề xuất các phương án phát triển cho đội ngũ Tutor.
\begin{itemize}
    \item Sơ đồ hoạt động
    \begin{figure}[H]
    \centering
    \includegraphics[scale=0.35 ]{Picture/ACUC21.png}
    \caption{Sơ đồ hoạt động Use Case 21: Báo cáo chất lượng Tutor}
    \end{figure}
    \pagebreak
    \item Sơ đồ tuần tự
    \begin{figure}[H]
    \centering
    \includegraphics[scale=0.4 ]{Picture/SEUC21.png}
    \caption{Sơ đồ tuần tự Use Case 21: Báo cáo chất lượng Tutor}
    \end{figure}
\end{itemize}
%========================================================================================
\pagebreak
\subsection*{1.1.22. Use Case 22: Báo cáo tổng hợp (Khoa, PCTSV, PĐT)}
\addcontentsline{toc}{subsection}{1.1.22. Use Case 22: Báo cáo tổng hợp (Khoa, PCTSV, PĐT)}
Chức năng này cung cấp cho các cấp quản lý cao nhất một bức tranh tổng thể về hoạt động của hệ thống. Dữ liệu từ các báo cáo con sẽ được tổng hợp và phân tích theo nhiều góc độ khác nhau. Báo cáo này đóng vai trò quan trọng trong việc đánh giá hiệu quả và định hướng chiến lược phát triển cho chương trình.
\begin{itemize}
    \item Sơ đồ hoạt động
    \begin{figure}[H]
    \centering
    \includegraphics[scale=0.35 ]{Picture/ACUC22.png}
    \caption{Sơ đồ hoạt động Use Case 22: Báo cáo tổng hợp (Khoa, PCTSV, PĐT)}
    \end{figure}
    \item Sơ đồ tuần tự
    \begin{figure}[H]
    \centering
    \includegraphics[scale=0.35 ]{Picture/SEUC22.png}
    \caption{Sơ đồ tuần tự Use Case 22: Báo cáo tổng hợp (Khoa, PCTSV, PĐT)}
    \end{figure}
\end{itemize}
%========================================================================================
\pagebreak
\subsection*{1.1.23. Use Case 23: Tutor tạo chương trình học}
\addcontentsline{toc}{subsection}{1.1.23. Use Case 23: Tutor tạo chương trình học}
Hệ thống trao quyền cho Tutor để có thể sáng tạo và mở thêm các chương trình học mới, từ ôn tập kiến thức chuyên sâu đến phát triển kỹ năng mềm. Tutor chỉ cần nhập các thông tin cần thiết, và sau khi được phê duyệt, chương trình sẽ xuất hiện trên hệ thống, sẵn sàng chào đón các sinh viên đăng ký.
\begin{itemize}
    \item Sơ đồ hoạt động
    \begin{figure}[H]
    \centering
    \includegraphics[scale=0.4 ]{Picture/ACUC23.png}
    \caption{Sơ đồ hoạt động Use Case 23: Tutor tạo chương trình học}
    \end{figure}
    \pagebreak
    \item Sơ đồ tuần tự
    \begin{figure}[H]
    \centering
    \includegraphics[scale=0.4 ]{Picture/SEUC23.png}
    \caption{Sơ đồ tuần tự Use Case 23: Tutor tạo chương trình học}
    \end{figure}
\end{itemize}
%========================================================================================
\pagebreak
\subsection*{1.1.24. Use Case 24: SV đăng ký chương trình học thuật}
\addcontentsline{toc}{subsection}{1.1.24. Use Case 24: SV đăng ký chương trình học thuật}
Sinh viên có thể dễ dàng tìm thấy và đăng ký các chương trình bổ trợ kiến thức phù hợp với nhu cầu của mình. Để đảm bảo chất lượng và sự tập trung, hệ thống sẽ áp dụng một số quy tắc, chẳng hạn như giới hạn số lượng chương trình học thuật mà một sinh viên có thể tham gia trong cùng một thời điểm.
\begin{itemize}
    \item Sơ đồ hoạt động
    \begin{figure}[H]
    \centering
    \includegraphics[scale=0.33 ]{Picture/ACUC24.png}
    \caption{Sơ đồ hoạt động Use Case 24: SV đăng ký chương trình học thuật}
    \end{figure}
    \item Sơ đồ tuần tự
    \begin{figure}[H]
    \centering
    \includegraphics[scale=0.33 ]{Picture/SEUC24.png}
    \caption{Sơ đồ tuần tự Use Case 24: SV đăng ký chương trình học thuật}
    \end{figure}
\end{itemize}
%========================================================================================
\pagebreak
\subsection*{1.1.25. Use Case 25: SV đăng ký chương trình phi học thuật}
\addcontentsline{toc}{subsection}{1.1.25. Use Case 25: SV đăng ký chương trình phi học thuật}
Bên cạnh kiến thức chuyên môn, các hoạt động phát triển kỹ năng mềm cũng rất được khuyến khích. Sinh viên có thể tự do đăng ký tham gia các chương trình này mà không bị giới hạn về số lượng. Hệ thống chỉ cần đảm bảo rằng chương trình mà sinh viên chọn vẫn còn chỗ trống.
\begin{itemize}
    \item Sơ đồ hoạt động
    \begin{figure}[H]
    \centering
    \includegraphics[scale=0.4 ]{Picture/ACUC25.png}
    \caption{Sơ đồ hoạt động Use Case 25: SV đăng ký chương trình phi học thuật}
    \end{figure}
    \item Sơ đồ tuần tự
    \begin{figure}[H]
    \centering
    \includegraphics[scale=0.4 ]{Picture/SEUC25.png}
    \caption{Sơ đồ tuần tự Use Case 25: SV đăng ký chương trình phi học thuật}
    \end{figure}
\end{itemize}


%========================================================================================
\newpage
\section*{1.2. Giao diện}
\addcontentsline{toc}{section}{1.2. Giao diện}
Sau khi đã mô hình hóa các luồng nghiệp vụ và quy trình hệ thống, phần này sẽ trình bày về thiết kế giao diện người dùng (UI). Các giao diện được nhóm thiết kế trên website Figma.com. \\
Đường dẫn: \href{https://www.figma.com/design/JVJlph8kWxybLC45mcYPDN/BTL-CNPM?node-id=0-1&t=kgGeKYXHL2xgpn6x-1}{MentorLinkUI}
\subsection*{1.2.1. Đăng ký và đăng nhập}
\addcontentsline{toc}{subsection}{1.2.1. Đăng ký và đăng nhập} 
\begin{figure}[H]
    \centering
    \setlength{\fboxsep}{2pt}      % khoảng cách giữa viền và ảnh
    \setlength{\fboxrule}{0.5pt}   % độ dày viền
    \fbox{\includegraphics[scale=0.24 ]{Picture/UI/SignUp.png}}
    \caption{Giao diện đăng ký tài khoản}
\end{figure}
\begin{figure}[H]
    \centering
    \setlength{\fboxsep}{2pt}      
    \setlength{\fboxrule}{0.5pt}   
    \fbox{\includegraphics[scale=0.24 ]{Picture/UI/SignIn.png}}
    \caption{Giao diện đăng nhập tài khoản}
\end{figure}

%========================================================================================
\newpage
\subsection*{1.2.2. Giao diện dành cho sinh viên}
\addcontentsline{toc}{subsection}{1.2.2. Giao diện dành cho sinh viên}
\subsubsection*{Trang chủ}
Giao diện trang chủ của sinh viên
\begin{figure}[H]
    \centering
    \setlength{\fboxsep}{2pt}     
    \setlength{\fboxrule}{0.5pt}   
    \fbox{\includegraphics[scale=0.24]{Picture/UI/HomePage_Student.png}}
    \caption{Giao diện trang chủ của sinh viên}
\end{figure}

\subsubsection*{Đăng ký môn học}
\begin{itemize}
    \item Sinh viên chọn chức năng đăng kí môn học từ trang chủ (Hình 53), giao diện hiện ra các môn học khả dụng và và các môn học đã đăng ký.
    \begin{figure}[H]
    \centering
    \setlength{\fboxsep}{2pt}     
    \setlength{\fboxrule}{0.5pt}   
    \fbox{\includegraphics[scale=0.24]{Picture/UI/Student_CourseRegistration.png}}
    \caption{Giao diện đăng ký môn học}
    \end{figure}
    \item Sinh viên nhấn nút "Chi tiết" (Hình 54) để mở thông tin chi tiết của môn học.
    \begin{figure}[H]
    \centering
    \setlength{\fboxsep}{2pt}     
    \setlength{\fboxrule}{0.5pt}   
    \fbox{\includegraphics[scale=0.24]{Picture/UI/Student_CourseRegistration_Detail.png}}
    \caption{Giao diện chi tiết môn học đã đăng ký}
    \end{figure}
\end{itemize}

\subsubsection*{Tìm và ghép cặp Tutor}
\begin{itemize}
    \item Sinh viên chọn chức năng tìm và ghép cặp Tutor từ trang chủ (Hình 53), sinh viên chọn nút "thủ công", chọn môn học và hệ thống sẽ hiện danh sách Tutor để sinh viên chọ thủ công Tutor.
    \begin{figure}[H]
    \centering
    \setlength{\fboxsep}{2pt}     
    \setlength{\fboxrule}{0.5pt}   
    \fbox{\includegraphics[scale=0.24]{Picture/UI/Student_FindTutor.png}}
    \caption{Giao diện tìm và ghép cặp Tutor thủ công}
    \end{figure}
    \pagebreak
    \item Sinh viên nhấn nút "Chi tiết" (Hình 56) để hiển thị thông tin chi tiết Tutor ở chế độ thủ công.
    \begin{figure}[H]
    \centering
    \setlength{\fboxsep}{2pt}     
    \setlength{\fboxrule}{0.5pt}   
    \fbox{\includegraphics[scale=0.24]{Picture/UI/Student_ManualPairing_Detail.png}}
    \caption{Giao diện tìm và ghép cặp Tutor thủ công}
    \end{figure}
    \item Nếu sinh viên chọn tự động (Hình 56), hệ thống sẽ tự hiển thị thông tin chi tiết của Tutor.
    \begin{figure}[H]
    \centering
    \setlength{\fboxsep}{2pt}     
    \setlength{\fboxrule}{0.5pt}   
    \fbox{\includegraphics[scale=0.24]{Picture/UI/Student_AutoPairing_Detail.png}}
    \caption{Giao diện tìm và ghép cặp Tutor tự động}
    \end{figure}
\end{itemize}
\pagebreak
\subsubsection*{Quản lý lịch}
\begin{itemize}
    \item Sinh viên chọn chức năng quản lý lịch từ trang chủ (Hình 53), hệ thống hiển thị các môn học đã đăng ký, sinh viên chọn nút "đăng ký lịch học". 
    \begin{figure}[H]
    \centering
    \setlength{\fboxsep}{2pt}     
    \setlength{\fboxrule}{0.5pt}   
    \fbox{\includegraphics[scale=0.24]{Picture/UI/Student_ScheduleMan_ChooseCourse.png}}
    \caption{Giao diện quản lý lịch học}
    \end{figure}
    \item Hệ thống hiển thị lịch học để sinh viên đăng ký, sinh viên chọn nút "Đăng ký" để đăng ký lịch học phù hợp.
    \begin{figure}[H]
    \centering
    \setlength{\fboxsep}{2pt}     
    \setlength{\fboxrule}{0.5pt}   
    \fbox{\includegraphics[scale=0.24]{Picture/UI/Student_ScheduleMan_SelectTime.png}}
    \caption{Giao diện đăng ký lịch học}
    \end{figure}
    \pagebreak
    \item Lịch học sẽ hiển thị ở Tab lịch học, có thể hủy lịch học và sửa đổi lịch học.
    \begin{figure}[H]
    \centering
    \setlength{\fboxsep}{2pt}     
    \setlength{\fboxrule}{0.5pt}   
    \fbox{\includegraphics[scale=0.24]{Picture/UI/Student_ScheduleMan_Selected.png}}
    \caption{Giao diện chọn lịch học}
    \end{figure}
    \item Nếu sinh viên đổi lịch học (Hình 61), hệ thống hiển thị lịch học để sinh viên chọn.
    \begin{figure}[H]
    \centering
    \setlength{\fboxsep}{2pt}     
    \setlength{\fboxrule}{0.5pt}   
    \fbox{\includegraphics[scale=0.24]{Picture/UI/Student_ScheduleMan_ChangeTime.png}}
    \caption{Giao diện đổi lịch học}
    \end{figure}
    \pagebreak
    \item Nếu sinh viên hủy lịch học (Hình 61), hệ thống sẽ gửi cảnh báo, nếu chọn Đồng ý hệ thống sẽ loại bỏ lịch học khỏi Tab lịch học, nếu chọn Hủy hệ thống sẽ hoàn tác hành động hủy lịch.
    \begin{figure}[H]
    \centering
    \setlength{\fboxsep}{2pt}     
    \setlength{\fboxrule}{0.5pt}   
    \fbox{\includegraphics[scale=0.24]{Picture/UI/Student_ScheduleMan_DiscardTime.png}}
    \caption{Giao diện hủy lịch học}
    \end{figure}
\end{itemize}

\subsubsection*{Tài liệu và buổi học}
\begin{itemize}
    \item Sinh viên chọn chức năng tài liệu và buổi học (Hình 53), hệ thống hiển thị các môn học đã đăng ký, các tài liệu và record buổi học. 
    \begin{figure}[H]
    \centering
    \setlength{\fboxsep}{2pt}     
    \setlength{\fboxrule}{0.5pt}   
    \fbox{\includegraphics[scale=0.24]{Picture/UI/Student_Materials.png}}
    \caption{Giao diện tài liệu và record buổi học}
    \end{figure}
\end{itemize}
\pagebreak
\subsubsection*{Đánh giá và phản hồi}
\begin{itemize}
    \item Sinh viên chọn chức năng tài liệu và buổi học (Hình 53), hệ thống hiển thị các môn học đã đăng ký, sinh viên chọn "Khảo sát" để đánh giá Tutor. 
    \begin{figure}[H]
    \centering
    \setlength{\fboxsep}{2pt}     
    \setlength{\fboxrule}{0.5pt}   
    \fbox{\includegraphics[scale=0.24]{Picture/UI/Student_Rating.png}}
    \caption{Giao diện đánh giá Tutor}
    \end{figure}
\end{itemize}

%========================================================================================
\subsection*{1.2.3. Giao diện dành cho Tutor}
\addcontentsline{toc}{subsection}{1.2.3. Giao diện dành cho Tutor}

\subsubsection*{Trang chủ}
Giao diện trang chủ của Tutor
\begin{figure}[H]
    \centering
    \setlength{\fboxsep}{2pt}     
    \setlength{\fboxrule}{0.5pt}   
    \fbox{\includegraphics[scale=0.24]{Picture/UI/HomePage_Tutor.png}}
    \caption{Giao diện trang chủ của Tutor}
\end{figure}
\pagebreak
\subsubsection*{Thiết lập lịch dạy}
\begin{itemize}
    \item Tutor chọn chức năng thiết lập lịch dạy (Hình 66), giao diện hiện ra nút "Đăng ký" để Tutor đăng ký. 
    \begin{figure}[H]
    \centering
    \setlength{\fboxsep}{2pt}     
    \setlength{\fboxrule}{0.5pt}   
    \fbox{\includegraphics[scale=0.24]{Picture/UI/Tutor_ScheduleMan_HomePage.png}}
    \caption{Giao diện lịch trống}
    \end{figure}
    \item Hệ thống hiển thị ngày giờ và hình thức mặc định là Online, Tutor ấn nút "Đăng ký".
    \begin{figure}[H]
    \centering
    \setlength{\fboxsep}{2pt}     
    \setlength{\fboxrule}{0.5pt}   
    \fbox{\includegraphics[scale=0.24]{Picture/UI/Tutor_ScheduleMan_SelectTimeOnline.png}}
    \caption{Giao diện chọn ngày, giờ, hình thức Online}
    \end{figure}
    \pagebreak
    \item Nếu chọn hình thức Offline (Hình 68), Tutor phải nhập thêm số phòng, Tutor ấn nút "Đăng ký".
    \begin{figure}[H]
    \centering
    \setlength{\fboxsep}{2pt}     
    \setlength{\fboxrule}{0.5pt}   
    \fbox{\includegraphics[scale=0.24]{Picture/UI/Tutor_ScheduleMan_SelectTimeOffline.png}}
    \caption{Giao diện chọn ngày, giờ, hình thức Offine}
    \end{figure}
    \item Sau khi ấn nút đăng ký (Hình 68, 69), hệ thống gửi thông báo "Đã đăng ký thành công", chọn "Hủy" để tắt thông báo.
    \begin{figure}[H]
    \centering
    \setlength{\fboxsep}{2pt}     
    \setlength{\fboxrule}{0.5pt}   
    \fbox{\includegraphics[scale=0.24]{Picture/UI/Tutor_ScheduleMan_Selected.png}}
    \caption{Giao diện thông báo "Đã đăng ký thành công"}
    \end{figure}
    \pagebreak
    \item Hệ thống sẽ hiển thị các lịch mà Tutor đã đăng ký, có thể sửa hoặc xóa.
    \begin{figure}[H]
    \centering
    \setlength{\fboxsep}{2pt}     
    \setlength{\fboxrule}{0.5pt}   
    \fbox{\includegraphics[scale=0.24]{Picture/UI/Tutor_ScheduleMan_HomePage_FilledSchedule.png}}
    \caption{Giao diện lịch đã đăng ký}
    \end{figure}    
    \item Nếu chọn sửa lịch (Hình 71), hiển thị lại giao diện chọn lại ngày giờ và hình thức mặc định là Online (Hình 68), chọn đổi để xác nhận đổi lịch.
    \begin{figure}[H]
    \centering
    \setlength{\fboxsep}{2pt}     
    \setlength{\fboxrule}{0.5pt}   
    \fbox{\includegraphics[scale=0.24]{Picture/UI/Tutor_ScheduleMan_ChangeTimeOnline.png}}
    \caption{Giao diện sửa lịch đã đăng ký, hình thức Online}
    \end{figure}
    \pagebreak
    \item Nếu chọn sửa lịch (Hình 71), nếu chọn hình thức Offline thì Tutor nhập thêm số phòng, sau đó chọn "Đổi".
    \begin{figure}[H]
    \centering
    \setlength{\fboxsep}{2pt}     
    \setlength{\fboxrule}{0.5pt}   
    \fbox{\includegraphics[scale=0.24]{Picture/UI/Tutor_ScheduleMan_ChangeTimeOffline.png}}
    \caption{Giao diện sửa lịch đã đăng ký, hình thức Offline}
    \end{figure}
    \item Nếu chọn đổi xóa lịch (Hình 71), lịch sẽ tự động xóa khỏi danh sách lịch đã đăng ký.
    \begin{figure}[H]
    \centering
    \setlength{\fboxsep}{2pt}     
    \setlength{\fboxrule}{0.5pt}   
    \fbox{\includegraphics[scale=0.24]{Picture/UI/Tutor_ScheduleMan_DeletePage.png}}
    \caption{Giao diện sau khi xóa lịch đã đăng ký}
    \end{figure}
\end{itemize}

\pagebreak
\subsubsection*{Quản lý buổi học và điểm danh}
\begin{itemize}
    \item Tutor chọn chức năng quản lý buổi học và điểm danh (Hình 66), giao diện hiện ra danh sách môn học mà Tutor đã đăng ký lịch.
    \begin{figure}[H]
    \centering
    \setlength{\fboxsep}{2pt}     
    \setlength{\fboxrule}{0.5pt}   
    \fbox{\includegraphics[scale=0.24]{Picture/UI/Tutor_CourseMan_HomePage.png}}
    \caption{Giao diện quản lý môn học đã đăng ký dạy}
    \end{figure}
    \item Tutor chọn môn học và chọn hình thức Online (Hình 75).
    \begin{figure}[H]
    \centering
    \setlength{\fboxsep}{2pt}     
    \setlength{\fboxrule}{0.5pt}   
    \fbox{\includegraphics[scale=0.24]{Picture/UI/Tutor_CourseMan_ChooseOnline.png}}
    \caption{Giao diện chọn hình thức Online}
    \end{figure}
    \pagebreak
    \item Hệ thống sẽ hiển thị danh sách lớp, bao gồm mã lớp của hình thức Online. Tutor chọn Chi tiết.
    \begin{figure}[H]
    \centering
    \setlength{\fboxsep}{2pt}     
    \setlength{\fboxrule}{0.5pt}   
    \fbox{\includegraphics[scale=0.24]{Picture/UI/Tutor_CourseMan_ListOfClassOnline.png}}
    \caption{Giao diện danh sách các lớp Online và mã lớp}
    \end{figure}
    \item Hệ thống sẽ hiển thị thêm danh sách các buổi học Online học ở Tab buổi học, Tutor chọn 1 buổi học.
    \begin{figure}[H]
    \centering
    \setlength{\fboxsep}{2pt}     
    \setlength{\fboxrule}{0.5pt}   
    \fbox{\includegraphics[scale=0.24]{Picture/UI/Tutor_CourseMan_Online_ChooseDate.png}}
    \caption{Giao diện chọn buổi học Online của môn học}
    \end{figure}
    \pagebreak
    \item Sau khi chọn 1 buổi học (Hình 78), hiển thị Tab thông tin buổi dạy và Tab danh sách sinh viên, chọn Có để điểm danh sinh viên hoặc chọn vắng nếu sinh viên đó không học buổi đó.
    \begin{figure}[H]
    \centering
    \setlength{\fboxsep}{2pt}     
    \setlength{\fboxrule}{0.5pt}   
    \fbox{\includegraphics[scale=0.24]{Picture/UI/Tutor_CourseMan_Online_ChooseStudent.png}}
    \caption{Giao diện chọn sinh viên để điểm danh}
    \end{figure}
    \item Nếu chọn vắng (Hình 79), hệ thống hiển thị thông báo, Tutor nhập lí do nếu vắng có phép ngược lại chọn không phép.
    \begin{figure}[H]
    \centering
    \setlength{\fboxsep}{2pt}     
    \setlength{\fboxrule}{0.5pt}   
    \fbox{\includegraphics[scale=0.24]{Picture/UI/Tutor_CourseMan_Online_Absent.png}}
    \caption{Giao diện thông tin vắng có phép/không phép}
    \end{figure}
    \pagebreak
    \item Tutor đăng ký lịch dạy bù (Hình 79), chọn ngày, giờ, hình thức Online. 
    \begin{figure}[H]
    \centering
    \setlength{\fboxsep}{2pt}     
    \setlength{\fboxrule}{0.5pt}   
    \fbox{\includegraphics[scale=0.24]{Picture/UI/Tutor_CourseMan_Online_Make-upClasses.png}}
    \caption{Giao diện đăng ký dạy bù hình thức Online}
    \end{figure}
    \item Tutor chọn môn học và chọn hình thức Offline (Hình 75).
    \begin{figure}[H]
    \centering
    \setlength{\fboxsep}{2pt}     
    \setlength{\fboxrule}{0.5pt}   
    \fbox{\includegraphics[scale=0.24]{Picture/UI/Tutor_CourseMan_ChooseOffline.png}}
    \caption{Giao diện chọn hình thức Offline}
    \end{figure}
    \pagebreak
    \item Hệ thống sẽ hiển thị danh sách lớp, bao gồm mã lớp của hình thức Offline. Tutor chọn Chi tiết.
    \begin{figure}[H]
    \centering
    \setlength{\fboxsep}{2pt}     
    \setlength{\fboxrule}{0.5pt}   
    \fbox{\includegraphics[scale=0.24]{Picture/UI/Tutor_CourseMan_ListOfClassOffline.png}}
    \caption{Giao diện danh sách các lớp Offline và mã lớp}
    \end{figure}
    \item Hệ thống sẽ hiển thị thêm danh sách các buổi học Offline ở Tab buổi học, Tutor chọn 1 buổi học.
    \begin{figure}[H]
    \centering
    \setlength{\fboxsep}{2pt}     
    \setlength{\fboxrule}{0.5pt}   
    \fbox{\includegraphics[scale=0.24]{Picture/UI/Tutor_CourseMan_Offline_ChooseDate.png}}
    \caption{Giao diện chọn buổi học Offline của môn học}
    \end{figure}
    \pagebreak
    \item Sau khi chọn 1 buổi học (Hình 84), hiển thị Tab thông tin buổi dạy và Tab danh sách sinh viên, chọn Có để điểm danh sinh viên hoặc chọn vắng nếu sinh viên đó không học buổi đó.
    \begin{figure}[H]
    \centering
    \setlength{\fboxsep}{2pt}     
    \setlength{\fboxrule}{0.5pt}   
    \fbox{\includegraphics[scale=0.24]{Picture/UI/Tutor_CourseMan_Offline_ChooseStudent.png}}
    \caption{Giao diện chọn sinh viên để điểm danh}
    \end{figure}
    \item Nếu chọn vắng (Hình 85), hệ thống hiển thị thông báo, Tutor nhập lí do nếu vắng có phép ngược lại chọn không phép.
    \begin{figure}[H]
    \centering
    \setlength{\fboxsep}{2pt}     
    \setlength{\fboxrule}{0.5pt}   
    \fbox{\includegraphics[scale=0.24]{Picture/UI/Tutor_CourseMan_Offline_Absent.png}}
    \caption{Giao diện thông tin vắng có phép/không phép}
    \end{figure}
    \pagebreak
    \item Tutor đăng ký lịch dạy bù (Hình 85), chọn ngày, giờ, hình thức Offline, nhập số phòng. 
    \begin{figure}[H]
    \centering
    \setlength{\fboxsep}{2pt}     
    \setlength{\fboxrule}{0.5pt}   
    \fbox{\includegraphics[scale=0.24]{Picture/UI/Tutor_CourseMan_Offline_Make-upClasses.png}}
    \caption{Giao diện đăng ký lịch dạy bù hình thức Offline}
    \end{figure}
    \item Sau khi đăng ký lịch dạy bù (Hình 81, 87), hệ thống hiển thị thông báo "Đã đăng ký thành công".
    \begin{figure}[H]
    \centering
    \setlength{\fboxsep}{2pt}     
    \setlength{\fboxrule}{0.5pt}   
    \fbox{\includegraphics[scale=0.24]{Picture/UI/Tutor_CourseMan_OnlineOffline_Make-upClasses_Registered.png}}
    \caption{Giao diện thông báo đăng ký dạy bù thành công}
    \end{figure}
\end{itemize}
\pagebreak
\subsubsection*{Quản lý tài liệu học tập}
\begin{itemize}
    \item Tutor chọn chức năng quản lý tài liệu học tập (Hình 66), danh sách các môn học, tài liệu, record. Tutor xem, sửa, xóa tài liệu hoặc record.
    \begin{figure}[H]
    \centering
    \setlength{\fboxsep}{2pt}     
    \setlength{\fboxrule}{0.5pt}   
    \fbox{\includegraphics[scale=0.24]{Picture/UI/Tutor_Materials.png}}
    \caption{Giao diện quản lý tài liệu và record}
    \end{figure}
    \item Nếu Tutor chọn sửa ở tài liệu (Hình 89), Tutor có thể sửa tên của tài liệu đó.
    \begin{figure}[H]
    \centering
    \setlength{\fboxsep}{2pt}     
    \setlength{\fboxrule}{0.5pt}   
    \fbox{\includegraphics[scale=0.24]{Picture/UI/Tutor_Materials_ChangeFileName.png}}
    \caption{Giao diện đổi tên tài liệu}
    \end{figure}
    \pagebreak
    \item Nếu Tutor chọn xóa tài liệu (Hình 89), hệ thống sẽ gửi cảnh báo xác nhận.
    \begin{figure}[H]
    \centering
    \setlength{\fboxsep}{2pt}     
    \setlength{\fboxrule}{0.5pt}   
    \fbox{\includegraphics[scale=0.24]{Picture/UI/Tutor_Materials_DeleteFile.png}}
    \caption{Giao diện cảnh báo khi xóa tài liệu}
    \end{figure}
    \item Nếu Tutor chấp nhận xóa tài liệu, bấm Đồng ý. 
    \begin{figure}[H]
    \centering
    \setlength{\fboxsep}{2pt}     
    \setlength{\fboxrule}{0.5pt}   
    \fbox{\includegraphics[scale=0.24]{Picture/UI/Tutor_Materials_FileDeleted.png}}
    \caption{Giao diện sau khi xóa thành công tài liệu}
    \end{figure}
    \pagebreak
    \item Nếu Tutor chọn sửa record (Hình 89), Tutor có thể sửa tên record đó.
    \begin{figure}[H]
    \centering
    \setlength{\fboxsep}{2pt}     
    \setlength{\fboxrule}{0.5pt}   
    \fbox{\includegraphics[scale=0.24]{Picture/UI/Tutor_Materials_ChangeRecordName.png}}
    \caption{Giao diện đổi tên record}
    \end{figure}
    \item Nếu Tutor chọn xóa tài liệu (Hình 89), hệ thống sẽ gửi cảnh báo xác nhận.
    \begin{figure}[H]
    \centering
    \setlength{\fboxsep}{2pt}     
    \setlength{\fboxrule}{0.5pt}   
    \fbox{\includegraphics[scale=0.24]{Picture/UI/Tutor_Materials_DeleteRecord.png}}
    \caption{Giao diện cảnh báo xóa record}
    \end{figure}
    \pagebreak
    \item Nếu Tutor chấp nhận xóa tài liệu, bấm Đồng ý.
    \begin{figure}[H]
    \centering
    \setlength{\fboxsep}{2pt}     
    \setlength{\fboxrule}{0.5pt}   
    \fbox{\includegraphics[scale=0.24]{Picture/UI/Tutor_Materials_RecordDeleted.png}}
    \caption{Giao diện khi xóa thành công record}
    \end{figure}
\end{itemize}
\subsubsection*{Đánh giá sinh viên}
\begin{itemize}
    \item Tutor chọn chức năng đánh giá sinh viên (Hình 66), hiển thị danh sách môn học để Tutor lựa chọn, chọn đánh giá. 
    \begin{figure}[H]
    \centering
    \setlength{\fboxsep}{2pt}     
    \setlength{\fboxrule}{0.5pt}   
    \fbox{\includegraphics[scale=0.24]{Picture/UI/Tutor_Rating_ListOfCourse.png}}
    \caption{Giao diện các môn học đã đăng ký để đánh giá sinh viên}
    \end{figure}
    \pagebreak
    \item Tutor chọn đánh giá, sau đó chọn chi tiết từng sinh viên để dánh giá sinh viên đó.
    \begin{figure}[H]
    \centering
    \setlength{\fboxsep}{2pt}     
    \setlength{\fboxrule}{0.5pt}   
    \fbox{\includegraphics[scale=0.24]{Picture/UI/Tutor_Rating_ListOfStudent.png}}
    \caption{Giao diện danh sách sinh viên theo môn học}
    \end{figure}
    \item Mỗi sinh viên đều được Tutor đánh giá bằng số sao và nhận xét.
    \begin{figure}[H]
    \centering
    \setlength{\fboxsep}{2pt}     
    \setlength{\fboxrule}{0.5pt}   
    \fbox{\includegraphics[scale=0.24]{Picture/UI/Tutor_Rating_StudentDetail.png}}
    \caption{Giao diện đánh giá chi tiết sinh viên}
    \end{figure}
\end{itemize}