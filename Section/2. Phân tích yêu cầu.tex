\part*{2. Phân tích yêu cầu dự án}
\addcontentsline{toc}{part}{2. Phân tích yêu cầu dự án}
Trong chương này, nhóm em sẽ tiến hành xác định và phân tích các yêu cầu của hệ thống. Trước hết, việc nhận diện các tác nhân (Actors) tương tác với hệ thống là bước quan trọng để hiểu rõ các luồng chức năng và mục tiêu mà phần mềm cần đáp ứng.

%========================================================================================

\section*{2.1. Tác nhân}
\addcontentsline{toc}{section}{2.1. Tác nhân}
Tác nhân (Actor) là một thực thể bên ngoài, có thể là người dùng hoặc một hệ thống khác, tương tác trực tiếp với hệ thống để thực hiện một mục tiêu cụ thể. Dựa trên vai trò và cách thức tương tác, các tác nhân của "Hệ thống hỗ trợ Tutor" được phân loại thành tác nhân chính và tác nhân phụ.

%========================================================================================

\subsection*{2.1.1. Tác nhân chính}
\addcontentsline{toc}{subsection}{2.1.1. Tác nhân chính}
\subsubsection*{2.1.1.1. Tutor}
\addcontentsline{toc}{subsubsection}{2.1.1.1. Tutor}
Tutor có thể là sinh viên giỏi, NCS, hoặc giảng viên, được đăng ký vào hệ thống để hỗ trợ học tập. Họ có trách nhiệm quản lý hồ sơ, lịch rảnh, tổ chức buổi học và theo dõi tiến độ của sinh viên.
\begin{itemize}
    \item \textbf{Tạo và cập nhật lịch rảnh:}
    \begin{itemize}
        \item \textbf{Input:} Ngày, giờ, hình thức học (online/offline).
        \item \textbf{Process:} Hệ thống lưu lại khung giờ rảnh, đồng bộ với module đặt lịch.
        \item \textbf{Output:} Lịch rảnh hiển thị cho sinh viên để chọn.
        \item \textbf{Constraints:} Không được nhập trùng khung giờ.
        \item \textbf{Acceptance:} Sinh viên có thể đặt lịch học trong khung rảnh của Tutor.
        \item \textbf{Error handling:} Nếu nhập sai định dạng hoặc trùng lịch thì hệ thống báo lỗi.
    \end{itemize}
    
    \item \textbf{Mở (oneline/offline), hủy, đổi buổi học:}
    \begin{itemize}
        \item \textbf{Input:} Yêu cầu mở buổi học, hoặc yêu cầu huỷ/đổi.
        \item \textbf{Process:} Hệ thống kiểm tra lịch đã có sinh viên đặt chưa, xử lý cập nhật.
        \item \textbf{Output:} Buổi học được thêm/sửa/xoá trong hệ thống.
        \item \textbf{Constraints:} Hủy/đổi $\geq$ 3h trước giờ học
        \item \textbf{Acceptance:} Sinh viên và Tutor đều nhận được thông báo cập nhật.
        \item \textbf{Error handling:} Nếu yêu cầu đổi sát giờ hệ thống từ chối, báo lỗi.
    \end{itemize}
    
    \item \textbf{Nhận thông báo và nhắc nhở giờ dạy:}
    \begin{itemize}
        \item \textbf{Input:} Lịch học sắp diễn ra.
        \item \textbf{Process:} Hệ thống gửi thông báo (noti/email).
        \item \textbf{Output:} Tutor nhận được thông báo đúng hạn.
        \item \textbf{Constraints:} Thông báo phải được gửi $\geq$ 30 phút trước giờ học.
        \item \textbf{Acceptance:} Tutor xác nhận đã đọc thông báo.
        \item \textbf{Error handling:} Nếu gửi lỗi → hệ thống gửi lại lần 2 hoặc báo qua email dự phòng.
    \end{itemize}
    
    \item \textbf{Theo dõi tiến bộ sinh viên:}
    \begin{itemize}
        \item \textbf{Input:} Điểm, nhận xét, đánh giá sau buổi học.
        \item \textbf{Process:} Tutor nhập vào form đánh giá, hệ thống lưu lại.
        \item \textbf{Output:} Báo cáo tiến bộ gắn với hồ sơ sinh viên.
        \item \textbf{Constraints:} Chỉ Tutor đã dạy sinh viên đó mới được nhập.
        \item \textbf{Acceptance:} Khoa/bộ môn có thể truy cập báo cáo.
        \item \textbf{Error handling:} Nếu buổi học chưa hoàn tất thì từ chối ghi nhận.
    \end{itemize}
    
    \item \textbf{Điểm danh và record:}
    \begin{itemize}
        \item \textbf{Input:} ID sinh viên tham gia, mã buổi học.
        \item \textbf{Process:} Hệ thống điểm danh, ghi log tham dự.
        \item \textbf{Output:} Record buổi học (thời lượng, người tham gia).
        \item \textbf{Constraints:} Mỗi SV chỉ được điểm danh vào 1 buổi học tại 1 thời điểm.
        \item \textbf{Acceptance:} Log lưu thành công và hiển thị trong báo cáo.
        \item \textbf{Error handling:} Nếu trùng ID hệ thống từ chối, báo lỗi.
    \end{itemize}
    
    \item \textbf{Cập nhật trạng thái buổi học:}
    \begin{itemize}
        \item \textbf{Input:} Trạng thái (hoàn thành, huỷ, đang diễn ra).
        \item \textbf{Process:} Tutor xác nhận trạng thái, hệ thống lưu lại.
        \item \textbf{Output:} Buổi học hiển thị trạng thái mới.
        \item \textbf{Constraints:} Trạng thái chỉ được thay đổi bởi Tutor của buổi học.
        \item \textbf{Acceptance:} Sinh viên và khoa/bộ môn nhìn thấy trạng thái chính xác.
        \item \textbf{Error handling:} Nếu cập nhật sai thì hệ thống cho phép sửa lại trong 24h.
    \end{itemize}

    \item \textbf{Đăng nội dung bài học:}
    \begin{itemize}
        \item \textbf{Input:} File/tài liệu/note buổi học.
        \item \textbf{Process:} Upload vào hệ thống, lưu trữ trong HCMUT\_LIBRARY.
        \item \textbf{Output:} Sinh viên có thể tải xuống.
        \item \textbf{Constraints:} Dung lượng $\leq$ 50MB/file.
        \item \textbf{Acceptance:} Nội dung hiển thị đúng với sinh viên liên quan.
        \item \textbf{Error handling:} Nếu file hỏng thì báo lỗi, yêu cầu upload lại.
    \end{itemize}
\end{itemize}

%========================================================================================

\subsubsection*{2.1.1.2. Sinh viên}
\addcontentsline{toc}{subsubsection}{2.1.1.2. Sinh viên}
 Sinh viên là đối tượng cần hỗ trợ, tham gia hệ thống để tìm Tutor, đặt lịch học và nhận hỗ trợ học tập.
\begin{itemize}
    \item \textbf{Tạo tài khoản, hồ sơ cá nhân:}
    \begin{itemize}
        \item \textbf{Input:} Họ tên, MSSV, email, số điện thoại, thông tin học tập (GPA, môn cần hỗ trợ).
        \item \textbf{Process:} Hệ thống kiểm tra định dạng dữ liệu, đồng bộ với HCMUT\_DATACORE.
        \item \textbf{Output:} Hồ sơ cá nhân của SV được lưu và hiển thị trong hệ thống.
        \item \textbf{Constraints:} MSSV và email phải trùng khớp dữ liệu HCMUT.
        \item \textbf{Acceptance:} SV có thể đăng nhập và sử dụng các chức năng khác.
        \item \textbf{Error handling:} Nếu dữ liệu không hợp lệ → hệ thống báo lỗi, yêu cầu sửa.
    \end{itemize}

    \item \textbf{Đăng ký chương trình học:}
    \begin{itemize}
        \item \textbf{Input:} Môn học hoặc lĩnh vực cần hỗ trợ, nguyện vọng học tập.
        \item \textbf{Process:} Hệ thống ghi nhận nhu cầu, đồng bộ với dữ liệu đào tạo và gợi ý Tutor phù hợp.
        \item \textbf{Output:} Hồ sơ SV được cập nhật với chương trình đã đăng ký.
        \item \textbf{Constraints:} Chỉ được đăng ký trong danh sách môn/lĩnh vực mà hệ thống hỗ trợ.
        \item \textbf{Acceptance:} SV thấy chương trình học hiển thị trong hồ sơ.
        \item \textbf{Error handling:} Nếu môn/lĩnh vực không hợp lệ thì hệ thống báo lỗi, yêu cầu chọn lại.
    \end{itemize}

    \item \textbf{Lựa chọn Tutor / được ghép tự động:}
    \begin{itemize}
        \item \textbf{Input:} Nhu cầu hỗ trợ (môn, lịch, hình thức).
        \item \textbf{Process:} 
        \begin{itemize}
            \item \textbf{Thủ công:} SV chọn Tutor trong danh sách.
            \item \textbf{Tự động: } Hệ thống so khớp theo khoa/ngành, lịch rảnh, AI ranking.
        \end{itemize}
        \item \textbf{Output:} Ghép cặp Tutor – SV được xác lập.
        \item \textbf{Constraints:} Một SV chỉ có 1 Tutor chính tại một thời điểm.
        \item \textbf{Acceptance:} SV thấy thông tin Tutor trong hồ sơ.
        \item \textbf{Error handling:} Nếu lịch trùng thì yêu cầu chọn lại hoặc hệ thống gợi ý Tutor khác.
    \end{itemize}

    \item \textbf{Đặt lịch học (cảnh báo trùng lịch):}
    \begin{itemize}
        \item \textbf{Input:} Ngày, giờ, môn học.
        \item \textbf{Process:} Hệ thống kiểm tra lịch rảnh của Tutor và lịch của SV.
        \item \textbf{Output:} Lịch học mới được thêm.
        \item \textbf{Constraints:} Không được đặt trùng với lịch học hoặc lịch thi chính thức.
        \item \textbf{Acceptance:} Lịch hiển thị trong tài khoản SV và Tutor.
        \item \textbf{Error handling:} Nếu trùng lịch thì cảnh báo, từ chối đặt.
    \end{itemize}

    \item \textbf{Nhận thông báo và nhắc nhở giờ học:}
    \begin{itemize}
        \item \textbf{Input:} Lịch học sắp diễn ra.
        \item \textbf{Process:} Hệ thống gửi thông báo (noti/email).
        \item \textbf{Output:} SV nhận được thông báo.
        \item \textbf{Constraints:} Thông báo $\geq$ 30 phút trước giờ học.
        \item \textbf{Acceptance:} SV xác nhận thông báo trên hệ thống.
        \item \textbf{Error handling:} Nếu thông báo lỗi thì gửi lại qua email dự phòng.
    \end{itemize}

    \item \textbf{Phản hồi và đánh giá chất lượng buổi học:}
    \begin{itemize}
        \item \textbf{Input:}  Điểm (1–5 sao), bình luận nhận xét.
        \item \textbf{Process:} Hệ thống lưu đánh giá gắn với buổi học và Tutor.
        \item \textbf{Output:} Thông tin phản hồi hiển thị cho Tutor và khoa/bộ môn.
        \item \textbf{Constraints:} Chỉ được đánh giá sau khi buổi học hoàn thành.
        \item \textbf{Acceptance:} Đánh giá hiển thị trong báo cáo tổng hợp.
        \item \textbf{Error handling:} Nếu buổi học chưa hoàn tất → từ chối đánh giá.
    \end{itemize}
\end{itemize}

%========================================================================================

\subsection*{2.1.2. Tác nhân phụ}
\addcontentsline{toc}{subsection}{2.1.2. Tác nhân phụ}

%========================================================================================

\subsubsection*{2.1.2.1. Khoa/Bộ môn}
\addcontentsline{toc}{subsubsection}{2.1.2.1. Khoa/Bộ môn}
\begin{itemize}
    \item \textbf{Nhận đánh giá và tổng hợp kết quả của sinh viên:}
    \begin{itemize}
        \item \textbf{Input:} Đánh giá (điểm số, nhận xét) từ sinh viên sau buổi học.
        \item \textbf{Process:} Hệ thống tổng hợp các phản hồi, phân loại theo môn học/Tutor.
        \item \textbf{Output:} Báo cáo chất lượng buổi học theo lớp, môn, Tutor.
        \item \textbf{Constraints:} Chỉ sử dụng đánh giá từ các buổi học hợp lệ.
        \item \textbf{Acceptance:} Báo cáo được cập nhật định kỳ (theo tuần/tháng).
        \item \textbf{Error handling:} Nếu thiếu dữ liệu đánh giá thì hệ thống ghi chú “chưa có đủ dữ liệu”.
    \end{itemize}
    
    \item \textbf{Quản lý chất lượng Tutor và SV:}
    \begin{itemize}
        \item \textbf{Input:} Hồ sơ Tutor, hồ sơ SV, số buổi học, đánh giá.
        \item \textbf{Process:} Khoa theo dõi, so sánh chất lượng giảng dạy và mức độ tiến bộ của SV.
        \item \textbf{Output:} Bảng xếp hạng/đánh giá Tutor và tổng kết tiến độ SV.
        \item \textbf{Constraints:} Dữ liệu phải dựa trên lịch sử buổi học và đánh giá chính thức.
        \item \textbf{Acceptance:} Báo cáo thể hiện chính xác tình hình giảng dạy – học tập.
        \item \textbf{Error handling:} Nếu dữ liệu không đồng bộ → hệ thống tự động cảnh báo để kiểm tra. 
    \end{itemize}
    
    \item \textbf{Theo dõi tiến độ học tập của sinh viên:}
    \begin{itemize}
        \item \textbf{Input:} GPA trước/sau, kết quả môn học, log buổi học.
        \item \textbf{Process:} Hệ thống đối chiếu tiến độ, xác định sự cải thiện.
        \item \textbf{Output:} Báo cáo cá nhân/tập thể về tiến bộ của SV.
        \item \textbf{Constraints:} Chỉ tính các SV tham gia tối thiểu X buổi học.
        \item \textbf{Acceptance:} Báo cáo có thể dùng làm cơ sở xét khen thưởng hoặc hỗ trợ.
        \item \textbf{Error handling:} Nếu thiếu GPA hoặc dữ liệu học tập → báo cáo đánh dấu “khuyết dữ liệu”.
    \end{itemize}
\end{itemize}

%========================================================================================

\subsubsection*{2.1.2.2. Phòng Công tác sinh viên}
\addcontentsline{toc}{subsubsection}{2.1.2.2. Phòng Công tác sinh viên}
\begin{itemize}
    \item \textbf{Nắm bắt GPA sinh viên trước và sau khi tham gia:}
    \begin{itemize}
        \item \textbf{Input:} GPA ban đầu, GPA cập nhật sau kỳ học.
        \item \textbf{Process:} Hệ thống tự động lấy dữ liệu từ HCMUT\_DATACORE, đối chiếu kết quả trước/sau.
        \item \textbf{Output:} Báo cáo so sánh GPA từng sinh viên.
        \item \textbf{Constraints:} Dữ liệu GPA phải đồng bộ chính xác từ hệ thống đào tạo.
        \item \textbf{Acceptance:} PCTSV có thể tra cứu sự thay đổi kết quả học tập của SV.
        \item \textbf{Error handling:} Nếu thiếu dữ liệu GPA thì hệ thống báo lỗi, yêu cầu đồng bộ lại.
    \end{itemize}
    
    \item \textbf{Tổng hợp kết quả tham gia:}
    \begin{itemize}
        \item \textbf{Input:} Danh sách SV, log số buổi học, đánh giá từ Tutor.
        \item \textbf{Process:} Hệ thống thống kê tần suất tham gia và kết quả học tập.
        \item \textbf{Output:} Báo cáo mức độ tham gia của SV
        \item \textbf{Constraints:} hỉ tính những SV tham gia tối thiểu số buổi học quy định.
        \item \textbf{Acceptance:} Báo cáo được xuất file (Excel/PDF) và tích hợp vào hệ thống quản lý SV.
        \item \textbf{Error handling:} Nếu dữ liệu không đầy đủ thì báo cáo gắn cờ “chưa hoàn chỉnh”.
    \end{itemize}
    
    \item \textbf{Ghi nhận kết quả tham gia để xét điểm rèn luyện / học bổng:}
    \begin{itemize}
        \item \textbf{Input:} Báo cáo tổng hợp SV tham gia chương trình Tutor.
        \item \textbf{Process:} PCTSV đối chiếu với quy chế điểm rèn luyện, học bổng.
        \item \textbf{Output:} Điểm rèn luyện/học bổng của SV được cập nhật.
        \item \textbf{Constraints:} Chỉ SV có tham gia hợp lệ, đủ số buổi quy định mới được ghi nhận.
        \item \textbf{Acceptance:} Kết quả được tích hợp vào hệ thống xét điểm rèn luyện và học bổng.
        \item \textbf{Error handling:} Nếu báo cáo thiếu dữ liệu thì đánh dấu “pending” cho đến khi bổ sung.
    \end{itemize}
\end{itemize}

%========================================================================================

\subsubsection*{2.1.2.3. Phòng Đào tạo}
\addcontentsline{toc}{subsubsection}{2.1.2.3. Phòng Đào tạo}
\begin{itemize}
    \item \textbf{Quản lý và theo dõi hồ sơ Tutor:}
    \begin{itemize}
        \item \textbf{Input:} Hồ sơ cá nhân, chuyên môn, lịch rảnh của Tutor.
        \item \textbf{Process:} PĐT xem, kiểm tra và xác nhận hồ sơ Tutor.
        \item \textbf{Output:} Danh sách Tutor hợp lệ được duyệt.
        \item \textbf{Constraints:} Chỉ Tutor đủ điều kiện (ví dụ GPA $\geq$ 7.0, có chuyên môn rõ ràng) mới được phê duyệt.
        \item \textbf{Acceptance:} Hồ sơ hiển thị trong hệ thống cho SV lựa chọn.
        \item \textbf{Error handling:} Nếu hồ sơ không hợp lệ thì trả lại yêu cầu cập nhật.
    \end{itemize}

    \item \textbf{Theo dõi số lượng buổi học:}
    \begin{itemize}
        \item \textbf{Input:} Log buổi học từ hệ thống.
        \item \textbf{Process:} Hệ thống tổng hợp số buổi học theo Tutor, theo SV, theo môn.
        \item \textbf{Output:} Báo cáo thống kê buổi học (ngày, giờ, trạng thái, số lượng).
        \item \textbf{Constraints:} Chỉ tính các buổi học hợp lệ (có điểm danh).
        \item \textbf{Acceptance:} Báo cáo hiển thị chính xác cho quản lý đào tạo.
        \item \textbf{Error handling:} Nếu dữ liệu log thiếu thì hệ thống cảnh báo “incomplete data”.
    \end{itemize}

    \item \textbf{Tối ưu phân bổ nguồn lực giữa Tutor và SV:}
    \begin{itemize}
        \item \textbf{Input:} Danh sách Tutor, danh sách SV đăng ký, nhu cầu hỗ trợ.
        \item \textbf{Process:} Hệ thống gợi ý phân bổ Tutor cho SV (theo ngành, lịch rảnh, số lượng tối đa).
        \item \textbf{Output:} Bảng phân công Tutor – SV.
        \item \textbf{Constraints:} Một Tutor chỉ nhận tối đa số SV theo quy định (ví dụ $\leq$ 5 SV).
        \item \textbf{Acceptance:} Phân bổ hợp lý, không quá tải Tutor, đáp ứng nhu cầu SV.
        \item \textbf{Error handling:} Nếu số SV vượt quá khả năng phân bổ thì hệ thống cảnh báo, yêu cầu thêm Tutor.
    \end{itemize}
\end{itemize}


%========================================================================================

\section*{2.2. Sơ đồ usecase toàn hệ thống}
\addcontentsline{toc}{section}{2.2. Sơ đồ usecase toàn hệ thống}

%========================================================================================

\section*{2.3. User Stories}
\addcontentsline{toc}{section}{2.3. User Stories}
\subsection*{2.3.1. Module Quản lý Tài khoản và Hồ sơ}
\addcontentsline{toc}{subsection}{2.3.1. Module Quản lý Tài khoản và Hồ sơ} 
\subsubsection*{US-01: Đăng ký tài khoản} 
Là 1 user, tôi muốn đăng ký tài khoản trên hệ thống HCMUT SSO  bằng số điện thoại /email mình.
\subsubsection*{US-02: Đăng nhập hệ thống} 
Là  user đã có tài khoản, tôi muốn đăng nhập vào hệ thống một cách an toàn bằng email/SDT và mật khẩu, để truy cập dashboard phù hợp với vai trò của mình.
\subsubsection*{US-03: Cập nhật hồ sơ} 
Là 1 user, tôi muốn chỉnh sửa và cập nhật hồ sơ cá nhân (email, SĐT, chuyên môn,GPA) để thông tin luôn chính xác và mới nhất.

%========================================================================================

\subsection*{2.3.2. Module Đăng ký chương trình học}
\addcontentsline{toc}{subsection}{2.3.2. Module Đăng ký chương trình học}
\subsubsection*{US-04: Đăng ký chương trình học} 
Là một sinh viên, tôi muốn đăng kí môn học phù hợp chuyên ngành và tìm tutor phù hợp để có thể học tập và luyện thi.
\subsubsection*{US-05: Hủy đăng ký chương trình học} 
Là một sinh viên, tôi muốn hủy môn đã đăng ký chương trình học  khi cảm thấy không còn nhu cầu.

%========================================================================================

\subsection*{2.3.3. Module Ghép cặp Tutor – SV}
\addcontentsline{toc}{subsection}{2.3.3. Module Ghép cặp Tutor – SV}
\subsubsection*{US-06: Ghép thủ công (SV chọn Tutor)} 
Là một sinh viên, tôi muốn chọn tutor từ danh sách đề xuất phù hợp hợp chuyên ngành và thời gian biểu.

\subsubsection*{US-07: Ghép tự động (hệ thống đề xuất Tutor)} 
Là một sinh viên, tôi muốn hệ thống chọn tutor  dựa chuyên môn và thời gian biểu của tôi.

%========================================================================================

\subsection*{2.3.4. Module Quản lý lịch học}
\addcontentsline{toc}{subsection}{2.3.4. Module Quản lý lịch học}
\subsubsection*{US-08: Tạo lịch rảnh (Tutor)} 
Là tutor, tôi muốn tạo lịch rảnh để sinh viên có thể xem,sắp xếp và đăng ký.
\subsubsection*{US-09: Đặt lịch học (SV)} 
Là sinh viên, tôi muốn đặt lịch học với tutor phù hợp thời gian biểu của bản thân.
\subsubsection*{US-10: Hủy/Đổi lịch học} 
Là sinh viên, tôi muốn hủy hoặc đổi lịch học để phù hợp với thay đổi của bản thân.

%========================================================================================

\subsection*{2.3.5. Module Thông báo và nhắc nhở}
\addcontentsline{toc}{subsection}{2.3.5. Module Thông báo và nhắc nhở}
\subsubsection*{US-11: Gửi thông báo lịch học} 
Là user, tôi muốn nhận thông báo khi có  lịch học mới được đặt hoặc thay đổi.
\subsubsection*{US-12: Gửi nhắc nhở buổi học} 
Là user , tôi muốn nhận thông báo nhắc nhở trước buổi học để có thể chuẩn bị.

%========================================================================================

\subsection*{2.3.6. Module Quản lý buổi học và điểm danh}
\addcontentsline{toc}{subsection}{2.3.6. Module Quản lý buổi học và điểm danh}
\subsubsection*{US-13: Điểm danh sinh viên}
Là tutor, tôi muốn điểm danh sinh viên để có nắm rõ số lượng sinh viên tham gia buổi học.

\subsubsection*{US-14: Cập nhật trạng thái buổi học}
Là tutor, tôi muốn cập nhật trạng thái buổi học để sinh viên có thể nắm rõ tình hình.
%========================================================================================
 
\subsection*{2.3.7. Module Quản lý tài liệu học tập}
\addcontentsline{toc}{subsection}{2.3.7. Module Quản lý tài liệu học tập}
\subsubsection*{US-15: Tutor upload tài liệu}
Là tutor, tôi muốn đăng tải tài liệu học tập để sinh viên có thể sử dụng trong quá trình học tập.
\subsubsection*{US-16: SV tải tài liệu}
Là sinh viên, tôi muốn tải xuống tài liệu học tập được tutor chia sẽ để ôn tập.

%========================================================================================

\subsection*{2.3.8. Module Đánh giá và phản hồi}
\addcontentsline{toc}{subsection}{2.3.8. Module Đánh giá và phản hồi}
\subsubsection*{US-17: Sinh viên đánh giá Tutor} 
Là sinh viên, tôi muốn đánh giá tutor sau khi kết môn học để sinh viên khác biết về chất lượng giảng dạy.

\subsubsection*{US-18: Tutor đánh giá sinh viên} 
Là tutor, tôi muốn đánh thái độ học tập và mức độ tham gia của sinh viên để phản hồi cho khoa.

\subsubsection*{US-19: Khoa/BM tổng hợp đánh giá} 
Là Khoa, tôi muốn nhận tổng hợp đánh giá của sinh viên và tutor.

%========================================================================================

\subsection*{2.3.9. Module Thống kê và báo cáo}
\addcontentsline{toc}{subsection}{2.3.9. Module Thống kê và báo cáo}

\subsubsection*{US-20: Báo cáo kết quả học tập SV} 
Là khoa, tôi muốn tổng hợp và xác nhận kết quả học tập của sinh viên sau khi môn học kết thúc.

\subsubsection*{US-21: Báo cáo chất lượng Tutor} 
Là khoa, tôi muốn tổng hợp và xác nhận báo cáo chất lượng giảng dạy của tutor qua đánh giá của sinh viên và dữ liệu buổi học.

\subsubsection*{US-22: Báo cáo tổng hợp (Khoa/ PCTSV/PĐT)} 
Là Khoa, Phòng Công tác Sinh viên (PCTSV) và Phòng Đào tạo (PĐT),  tôi muốn có thể xem báo cáo tổng hợp toàn hệ thống sau khi dữ liệu từ các báo cáo con (từ Sinh viên, Tutor) đã được xác nhận và tổng hợp.

%========================================================================================

\subsection*{2.3.10. Module Chương trình học thuật và phi học thuật}
\addcontentsline{toc}{subsection}{2.3.9. Module Thống kê và báo cáo}

\subsubsection*{US-23: Tutor tạo chương trình học} 
Là tutor, tôi muốn tạo thêm chương trình học tập mới(học thuật hoặc phi học thuật) để sinh viên có học hỏi và tìm hiểu thêm kiến thức mới.

\subsubsection*{US-24: Sinh viên đăng ký chương trình học thuật} 
Là sinh viên, tôi muốn đăng ký tham gia chương trình học thuật để có thể học tập và trao dồi thêm kiến thức.

\subsubsection*{US-25: Sinh viên đăng ký chương trình phi học thuật} 
Là sinh viên, tôi muốn đăng ký tham gia chương trình phi học thuật để có thể nâng cao kỹ năng mềm và tham gia các hoạt động ngoại khóa.

%========================================================================================

% \newpage
\section*{2.4. Yêu cầu chức năng}
\addcontentsline{toc}{section}{2.4. Yêu cầu chức năng}
\subsection*{2.4.1. Module Quản lý Tài khoản và Hồ sơ}
\addcontentsline{toc}{subsection}{2.4.1. Module Quản lý Tài khoản và Hồ sơ}
\newpage
\subsubsection*{Use Case 01: Đăng ký tài khoản}
\begin{samepage}
\begin{table}[h!]
\begin{tabular}{|l|lll|l}
\cline{1-4}
\textbf{ID}                              & \multicolumn{3}{l|}{UC-01}                                                                                                                                                                                                                &  \\ \cline{1-4}
\textbf{Tên}                             & \multicolumn{3}{l|}{Đăng ký tài khoản}                                                                                                                                                                                                    &  \\ \cline{1-4}
\textbf{Mô tả}                           & \multicolumn{3}{l|}{Người dùng (SV hoặc Tutor) đăng ký tài khoản mới để tham gia hệ thống.}                                                                                                                                               &  \\ \cline{1-4}
\textbf{Actor chính}                     & \multicolumn{3}{l|}{Sinh viên, Tutor}                                                                                                                                                                                                     &  \\ \cline{1-4}
\textbf{Actor phụ}                       & \multicolumn{3}{l|}{Hệ thống xác thực OTP (Email/SMS), Admin PĐT}                                                                                                                                                                         &  \\ \cline{1-4}
\textbf{Tiền điều kiện}                  & \multicolumn{3}{l|}{\begin{tabular}[c]{@{}l@{}}Người dùng chưa có tài khoản\\ Có email hoặc số điện thoại hợp lệ\end{tabular}}                                                                                                            &  \\ \cline{1-4}
\textbf{Hậu điều kiện}                   & \multicolumn{3}{l|}{\begin{tabular}[c]{@{}l@{}}Tài khoản hợp lệ được tạo, có ID duy nhất.\\ Hồ sơ người dùng được lưu trong cơ sở dữ liệu\end{tabular}}                                                                                   &  \\ \cline{1-4}
\multirow{10}{*}{\textbf{Luồng sự kiện}} & \multicolumn{1}{l|}{\textbf{Bước}}      & \multicolumn{1}{l|}{\textbf{Thực hiện bởi}}     & \textbf{Mô tả}                                                                                                                                &  \\ \cline{2-4}
                                         & \multicolumn{1}{l|}{1}                  & \multicolumn{1}{l|}{Người dùng}                 & Chọn chức năng “Đăng ký”.                                                                                                                     &  \\ \cline{2-4}
                                         & \multicolumn{1}{l|}{2}                  & \multicolumn{1}{l|}{Người dùng}                 & \begin{tabular}[c]{@{}l@{}}Nhập thông tin cá nhân (SV: MSSV, GPA, khoa; \\ Tutor: chuyên môn, GPA $\geq$ 3.0 hoặc giấy xác minh).\end{tabular}     &  \\ \cline{2-4}
                                         & \multicolumn{1}{l|}{3}                  & \multicolumn{1}{l|}{Hệ thống}                   & Kiểm tra định dạng dữ liệu.                                                                                                                   &  \\ \cline{2-4}
                                         & \multicolumn{1}{l|}{4}                  & \multicolumn{1}{l|}{Hệ thống}                   & Gửi OTP xác thực qua email/SMS.                                                                                                               &  \\ \cline{2-4}
                                         & \multicolumn{1}{l|}{5}                  & \multicolumn{1}{l|}{Người dùng}                 & Nhập OTP nhận được.                                                                                                                           &  \\ \cline{2-4}
                                         & \multicolumn{1}{l|}{6}                  & \multicolumn{1}{l|}{Hệ thống}                   & Kiểm tra OTP.                                                                                                                                 &  \\ \cline{2-4}
                                         & \multicolumn{1}{l|}{7}                  & \multicolumn{1}{l|}{Hệ thống}                   & Kiểm tra trùng MSSV/email.                                                                                                                    &  \\ \cline{2-4}
                                         & \multicolumn{1}{l|}{8}                  & \multicolumn{1}{l|}{Hệ thống}                   & Lưu dữ liệu hợp lệ, gán UserID duy nhất.                                                                                                      &  \\ \cline{2-4}
                                         & \multicolumn{1}{l|}{9}                  & \multicolumn{1}{l|}{Hệ thống}                   & Hiển thị thông báo “Đăng ký thành công”                                                                                                       &  \\ \cline{1-4}
\multirow{4}{*}{\textbf{Luồng thay thế}} & \multicolumn{1}{l|}{\textbf{Bước}}      & \multicolumn{1}{l|}{\textbf{Thực hiện bởi}}     & \textbf{Mô tả}                                                                                                                                &  \\ \cline{2-4}
                                         & \multicolumn{1}{l|}{3a}                 & \multicolumn{1}{l|}{Người dùng}                 & \begin{tabular}[c]{@{}l@{}}Nhập email không hợp lệ → Hệ thống báo lỗi, \\ yêu cầu nhập lại (quay về bước 2).\end{tabular}                     &  \\ \cline{2-4}
                                         & \multicolumn{1}{l|}{5a}                 & \multicolumn{1}{l|}{Người dùng}                 & \begin{tabular}[c]{@{}l@{}}Người dùng nhập OTP sai → Hệ thống báo lỗi, \\ cho nhập lại tối đa 3 lần.\end{tabular}                             &  \\ \cline{2-4}
                                         & \multicolumn{1}{l|}{7a}                 & \multicolumn{1}{l|}{Hệ thống}                   & \begin{tabular}[c]{@{}l@{}}Phát hiện MSSV/email đã tồn tại → hiển thị \\ “Tài khoản đã có”.\end{tabular}                                      &  \\ \cline{1-4}
\textbf{Ngoại lệ}                        & \multicolumn{3}{l|}{\begin{tabular}[c]{@{}l@{}}Hệ thống không gửi được OTP (lỗi server/email/SMS) \\ → hiển thị “Vui lòng thử lại sau”. \\ Hệ thống lỗi khi lưu dữ liệu vào DB \\ → rollback, hiển thị “Thao tác thất bại”.\end{tabular}} &  \\ \cline{1-4}
\textbf{Business Rules}                  & \multicolumn{3}{l|}{\begin{tabular}[c]{@{}l@{}}Một MSSV/email chỉ được dùng cho 1 tài khoản. \\ Tutor phải được xác minh bởi PĐT trước khi tài khoản kích hoạt.\end{tabular}}                                                             &  \\ \cline{1-4}
\textbf{Data requirement}                & \multicolumn{3}{l|}{\begin{tabular}[c]{@{}l@{}}Users(userID, role, email, password, MSSV, GPA, faculty)\\ TutorProfile(tutorID, chuyên môn, giấy xác minh)\end{tabular}}                                                                  &  \\ \cline{1-4}
\end{tabular}
\caption{Bảng đặc tả chức năng đăng ký tài khoản}
\end{table}
\end{samepage}

%========================================================================================

\newpage
\subsubsection*{Use Case 02: Đăng nhập}
\begin{samepage}
\begin{table}[h!]
\begin{tabular}{|l|lll|l}
\cline{1-4}
\textbf{ID}                              & \multicolumn{3}{l|}{UC-02}                                                                                                                                                                                    &  \\ \cline{1-4}
\textbf{Tên}                             & \multicolumn{3}{l|}{Đăng nhập}                                                                                                                                                                                &  \\ \cline{1-4}
\textbf{Mô tả}                           & \multicolumn{3}{l|}{Người dùng đăng nhập để truy cập hệ thống bằng tài khoản đã đăng ký.}                                                                                                                     &  \\ \cline{1-4}
\textbf{Actor chính}                     & \multicolumn{3}{l|}{SV, Tutor, Admin}                                                                                                                                                                         &  \\ \cline{1-4}
\textbf{Actor phụ}                       & \multicolumn{3}{l|}{Hệ thống xác thực đăng nhập}                                                                                                                                                              &  \\ \cline{1-4}
\textbf{Tiền điều kiện}                  & \multicolumn{3}{l|}{Người dùng đã có tài khoản hợp lệ.}                                                                                                                                                       &  \\ \cline{1-4}
\textbf{Hậu điều kiện}                   & \multicolumn{3}{l|}{Người dùng truy cập được dashboard theo đúng quyền}                                                                                                                                       &  \\ \cline{1-4}
\multirow{6}{*}{\textbf{Luồng sự kiện}}  & \multicolumn{1}{l|}{\textbf{Bước}} & \multicolumn{1}{l|}{\textbf{Thực hiện bởi}} & \textbf{Mô tả}                                                                                                             &  \\ \cline{2-4}
                                         & \multicolumn{1}{l|}{1}             & \multicolumn{1}{l|}{Người dùng}             & Mở giao diện đăng nhập                                                                                                     &  \\ \cline{2-4}
                                         & \multicolumn{1}{l|}{2}             & \multicolumn{1}{l|}{Người dùng}             & Nhập email/MSSV và mật khẩu.                                                                                               &  \\ \cline{2-4}
                                         & \multicolumn{1}{l|}{3}             & \multicolumn{1}{l|}{Hệ thống}               & Kiểm tra định dạng dữ liệu đầu vào                                                                                         &  \\ \cline{2-4}
                                         & \multicolumn{1}{l|}{4}             & \multicolumn{1}{l|}{Hệ thống}               & Kiểm tra thông tin đăng nhập trong DB.                                                                                     &  \\ \cline{2-4}
                                         & \multicolumn{1}{l|}{5}             & \multicolumn{1}{l|}{Hệ thống}               & \begin{tabular}[c]{@{}l@{}}Xác nhận thông tin đúng → mở phiên đăng \\ nhập và chuyển đến giao diện chính.\end{tabular}     &  \\ \cline{1-4}
\multirow{3}{*}{\textbf{Luồng thay thế}} & \multicolumn{1}{l|}{\textbf{Bước}} & \multicolumn{1}{l|}{\textbf{Thực hiện bởi}} & \textbf{Mô tả}                                                                                                             &  \\ \cline{2-4}
                                         & \multicolumn{1}{l|}{2a}            & \multicolumn{1}{l|}{Người dùng}             & \begin{tabular}[c]{@{}l@{}}Nhập sai mật khẩu → Hệ thống báo lỗi \\ “Sai mật khẩu”, cho nhập lại tối đa 5 lần.\end{tabular} &  \\ \cline{2-4}
                                         & \multicolumn{1}{l|}{4a}            & \multicolumn{1}{l|}{Hệ thống}               & \begin{tabular}[c]{@{}l@{}}Phát hiện tài khoản bị khóa → hiển thị \\ “Tài khoản bị khóa, liên hệ Admin”..\end{tabular}     &  \\ \cline{1-4}
\textbf{Ngoại lệ}                        & \multicolumn{3}{l|}{\begin{tabular}[c]{@{}l@{}}Gặp lỗi khi truy vấn DB → hiển thị “Không thể đăng nhập lúc này, vui lòng \\ thử lại sau”.\end{tabular}}                                                       &  \\ \cline{1-4}
\textbf{Business Rules}                  & \multicolumn{3}{l|}{\begin{tabular}[c]{@{}l@{}}Nếu nhập sai mật khẩu quá 5 lần → Hệ thống tự động khóa tài khoản trong \\ 30 phút. Admin có quyền thiết lập lại mật khẩu cho người dùng.\end{tabular}}        &  \\ \cline{1-4}
\textbf{Data requirement}                & \multicolumn{3}{l|}{Users(userID, email, password, role, status, lastLogin)}                                                                                                                                  &  \\ \cline{1-4}
\end{tabular}
\caption{Bảng đặc tả chức năng đăng nhập}
\end{table}
\end{samepage}



%========================================================================================

\newpage
\subsubsection*{Use Case 03: Cập nhật hồ sơ}
\begin{samepage}
\begin{table}[h!]
\begin{tabular}{|l|lll|l}
\cline{1-4}
\textbf{ID}                              & \multicolumn{3}{l|}{UC-03}                                                                                                                                                                                &  \\ \cline{1-4}
\textbf{Tên}                             & \multicolumn{3}{l|}{Cập nhật hồ sơ}                                                                                                                                                                       &  \\ \cline{1-4}
\textbf{Mô tả}                           & \multicolumn{3}{l|}{Người dùng cập nhật thông tin hồ sơ cá nhân để đảm bảo dữ liệu mới nhất.}                                                                                                             &  \\ \cline{1-4}
\textbf{Actor chính}                     & \multicolumn{3}{l|}{SV, Tutor}                                                                                                                                                                            &  \\ \cline{1-4}
\textbf{Actor phụ}                       & \multicolumn{3}{l|}{Hệ thống, Admin (có quyền xem/sửa/kiểm tra Tutor profile) (PĐT, Khoa)}                                                                                                                &  \\ \cline{1-4}
\textbf{Tiền điều kiện}                  & \multicolumn{3}{l|}{Người dùng đã đăng nhập hệ thống.}                                                                                                                                                    &  \\ \cline{1-4}
\textbf{Hậu điều kiện}                   & \multicolumn{3}{l|}{Hồ sơ cập nhật thành công trong cơ sở dữ liệu.}                                                                                                                                       &  \\ \cline{1-4}
\multirow{6}{*}{\textbf{Luồng sự kiện}}  & \multicolumn{1}{l|}{\textbf{Bước}} & \multicolumn{1}{l|}{\textbf{Thực hiện bởi}} & \textbf{Mô tả}                                                                                                         &  \\ \cline{2-4}
                                         & \multicolumn{1}{l|}{1}             & \multicolumn{1}{l|}{Người dùng}             & Đăng nhập, chọn “Cập nhật hồ sơ”.                                                                                      &  \\ \cline{2-4}
                                         & \multicolumn{1}{l|}{2}             & \multicolumn{1}{l|}{Người dùng}             & \begin{tabular}[c]{@{}l@{}}Chỉnh sửa thông tin (SĐT, email, chuyên ngành, \\ mô tả năng lực…).\end{tabular}            &  \\ \cline{2-4}
                                         & \multicolumn{1}{l|}{3}             & \multicolumn{1}{l|}{Hệ thống}               & Kiểm tra định dạng dữ liệu.                                                                                            &  \\ \cline{2-4}
                                         & \multicolumn{1}{l|}{4}             & \multicolumn{1}{l|}{Hệ thống}               & Lưu thông tin mới vào DB.                                                                                              &  \\ \cline{2-4}
                                         & \multicolumn{1}{l|}{5}             & \multicolumn{1}{l|}{Hệ thống}               & Hiển thị thông báo “Cập nhật thành công”.                                                                              &  \\ \cline{1-4}
\multirow{3}{*}{\textbf{Luồng thay thế}} & \multicolumn{1}{l|}{\textbf{Bước}} & \multicolumn{1}{l|}{\textbf{Thực hiện bởi}} & \textbf{Mô tả}                                                                                                         &  \\ \cline{2-4}
                                         & \multicolumn{1}{l|}{2a}            & \multicolumn{1}{l|}{Người dùng}             & \begin{tabular}[c]{@{}l@{}}Bỏ trống trường bắt buộc (email, SĐT) \\ → Hệ thống báo lỗi, yêu cầu nhập lại.\end{tabular} &  \\ \cline{2-4}
                                         & \multicolumn{1}{l|}{3a}            & \multicolumn{1}{l|}{Người dùng}             & \begin{tabular}[c]{@{}l@{}}Nhập email/SĐT sai định dạng \\ → Hệ thống báo lỗi, yêu cầu nhập lại.\end{tabular}          &  \\ \cline{1-4}
\textbf{Ngoại lệ}                        & \multicolumn{3}{l|}{Hệ thống lỗi khi lưu dữ liệu vào DB → hiển thị “Cập nhật thất bại”.}                                                                                                                  &  \\ \cline{1-4}
\textbf{Business Rules}                  & \multicolumn{3}{l|}{\begin{tabular}[c]{@{}l@{}}Tutor phải cập nhật thông tin học thuật theo mẫu do PĐT quy định.\\ SV không được sửa MSSV.\end{tabular}}                                                  &  \\ \cline{1-4}
\textbf{Data requirement}                & \multicolumn{3}{l|}{\begin{tabular}[c]{@{}l@{}}Users(userID, email, phone, faculty, GPA, updatedAt)\\ TutorProfile(tutorID, chuyên môn, kinh nghiệm, updatedAt)\end{tabular}}                             &  \\ \cline{1-4}
\end{tabular}
\caption{Bảng đặc tả chức năng cập nhật hồ sơ}
\end{table}
\end{samepage}

%========================================================================================

\newpage
\subsection*{2.4.2. Module Đăng ký chương trình học}
\addcontentsline{toc}{subsection}{2.4.2. Module Đăng ký chương trình học}
% \newpage
\subsubsection*{Use Case 04: Đăng ký môn học}
\begin{samepage}
\begin{table}[h!]
\begin{tabular}{|l|lll|l}
\cline{1-4}
\textbf{ID} &
  \multicolumn{3}{l|}{UC-04} &
   \\ \cline{1-4}
\textbf{Tên} &
  \multicolumn{3}{l|}{Đăng ký môn học} &
   \\ \cline{1-4}
\textbf{Mô tả} &
  \multicolumn{3}{l|}{\begin{tabular}[c]{@{}l@{}}Sinh viên đăng ký môn học/lĩnh vực cần được Tutor hỗ trợ.\\ Mỗi lần đăng ký chỉ chọn 1 môn học, có thể lặp lại quy trình nhiều lần \\ (tối đa 4 môn).\end{tabular}} &
   \\ \cline{1-4}
\textbf{Actor chính} &
  \multicolumn{3}{l|}{Sinh viên} &
   \\ \cline{1-4}
\textbf{Actor phụ} &
  \multicolumn{3}{l|}{Hệ thống, Khoa/Bộ môn} &
   \\ \cline{1-4}
\textbf{Tiền điều kiện} &
  \multicolumn{3}{l|}{\begin{tabular}[c]{@{}l@{}}SV đã đăng ký tài khoản, hồ sơ hợp lệ.\\ Hệ thống đã có danh mục môn học/lĩnh vực hỗ trợ.\end{tabular}} &
   \\ \cline{1-4}
\textbf{Hậu điều kiện} &
  \multicolumn{3}{l|}{\begin{tabular}[c]{@{}l@{}}Thông tin chương trình học được lưu trong hồ sơ SV.\\ SV có thể tiếp tục đăng ký thêm môn khác, miễn là tổng số $\leq$ 4.\\ SV sẵn sàng cho bước ghép cặp với Tutor.\end{tabular}} &
   \\ \cline{1-4}
\multirow{9}{*}{\textbf{Luồng sự kiện}} &
  \multicolumn{1}{l|}{\textbf{Bước}} &
  \multicolumn{1}{l|}{\textbf{Thực hiện bởi}} &
  \textbf{Mô tả} &
   \\ \cline{2-4}
 &
  \multicolumn{1}{l|}{1} &
  \multicolumn{1}{l|}{Sinh viên} &
  Đăng nhập hệ thống &
   \\ \cline{2-4}
 &
  \multicolumn{1}{l|}{2} &
  \multicolumn{1}{l|}{Sinh viên} &
  Chọn chức năng “Đăng ký môn học”. &
   \\ \cline{2-4}
 &
  \multicolumn{1}{l|}{3} &
  \multicolumn{1}{l|}{Hệ thống} &
  \begin{tabular}[c]{@{}l@{}}Hiển thị danh mục môn học khả dụng trong \\ kỳ.\end{tabular} &
   \\ \cline{2-4}
 &
  \multicolumn{1}{l|}{4} &
  \multicolumn{1}{l|}{Sinh viên} &
  Chọn 1 môn học. &
   \\ \cline{2-4}
 &
  \multicolumn{1}{l|}{5} &
  \multicolumn{1}{l|}{Hệ thống} &
  \begin{tabular}[c]{@{}l@{}}Kiểm tra SV đã đăng ký bao nhiêu môn? \\ ($\leq$ 3 trước đó → cho phép; = 4 → từ chối).\end{tabular} &
   \\ \cline{2-4}
 &
  \multicolumn{1}{l|}{6} &
  \multicolumn{1}{l|}{Hệ thống} &
  Lưu thông tin đăng ký vào DB &
   \\ \cline{2-4}
 &
  \multicolumn{1}{l|}{7} &
  \multicolumn{1}{l|}{Hệ thống} &
  Hiển thị thông báo “Đăng ký thành công”. &
   \\ \cline{2-4}
 &
  \multicolumn{1}{l|}{8} &
  \multicolumn{1}{l|}{Sinh viên} &
  \begin{tabular}[c]{@{}l@{}}Có thể quay lại bước 2 để đăng ký thêm môn\\  khác\end{tabular} &
   \\ \cline{1-4}
\multirow{3}{*}{\textbf{Luồng thay thế}} &
  \multicolumn{1}{l|}{\textbf{Bước}} &
  \multicolumn{1}{l|}{\textbf{Thực hiện bởi}} &
  \textbf{Mô tả} &
   \\ \cline{2-4}
 &
  \multicolumn{1}{l|}{3a} &
  \multicolumn{1}{l|}{Hệ thống} &
  \begin{tabular}[c]{@{}l@{}}Nếu danh sách môn học khả dụng trống \\ (chưa có Tutor nào đăng ký dạy) → \\ Hệ thống hiển thị “Hiện chưa có môn học \\ nào khả dụng để đăng ký”.\end{tabular} &
   \\ \cline{2-4}
 &
  \multicolumn{1}{l|}{5a} &
  \multicolumn{1}{l|}{Sinh viên} &
  \begin{tabular}[c]{@{}l@{}}Nếu SV đã đăng ký đủ 4 môn → Hệ thống \\ hiển thị “Chỉ được đăng ký tối đa 4 môn”.\end{tabular} &
   \\ \cline{1-4}
\textbf{Ngoại lệ} &
  \multicolumn{3}{l|}{Hệ thống lỗi khi lưu vào DB → hiển thị “Đăng ký thất bại, thử lại sau”.} &
   \\ \cline{1-4}
\textbf{Business Rules} &
  \multicolumn{3}{l|}{\begin{tabular}[c]{@{}l@{}}SV chỉ có thể đăng ký tối đa 4 môn học cùng lúc.\\ Hệ thống chỉ hiển thị các môn học đã có ít nhất 1 Tutor đăng ký dạy.\\ SV có thể hủy môn học đã đăng ký (UC-05) để giải phóng slot trước khi \\ đăng ký mới.\\ Danh mục môn học khả dụng được Khoa/BM cập nhật theo từng kỳ học.\end{tabular}} &
   \\ \cline{1-4}
\textbf{Data requirement} &
  \multicolumn{3}{l|}{\begin{tabular}[c]{@{}l@{}}ProgramRegistration(regID, studentID, subjectID, purpose, regDate, \\ status)\end{tabular}} &
   \\ \cline{1-4}
\end{tabular}
\caption{Bảng đặc tả chức năng đăng ký môn học}
\end{table}
\end{samepage}

%========================================================================================

\newpage
\subsubsection*{Use Case 05: Hủy đăng ký môn học}
\begin{samepage}
\begin{table}[h!]
\begin{tabular}{|l|lll|l}
\cline{1-4}
\textbf{ID} &
  \multicolumn{3}{l|}{UC-05} &
   \\ \cline{1-4}
\textbf{Tên} &
  \multicolumn{3}{l|}{Hủy đăng ký môn học} &
   \\ \cline{1-4}
\textbf{Mô tả} &
  \multicolumn{3}{l|}{\begin{tabular}[c]{@{}l@{}}Sinh viên hủy môn học đã đăng ký nếu không còn nhu cầu hoặc muốn \\ đổi môn khác.\end{tabular}} &
   \\ \cline{1-4}
\textbf{Actor chính} &
  \multicolumn{3}{l|}{Sinh viên} &
   \\ \cline{1-4}
\textbf{Actor phụ} &
  \multicolumn{3}{l|}{Hệ thống} &
   \\ \cline{1-4}
\textbf{Tiền điều kiện} &
  \multicolumn{3}{l|}{\begin{tabular}[c]{@{}l@{}}Môn học chưa bắt đầu và còn ít nhất 1 tuần trước khi bắt đầu buổi \\ học đầu tiên\end{tabular}} &
   \\ \cline{1-4}
\textbf{Hậu điều kiện} &
  \multicolumn{3}{l|}{\begin{tabular}[c]{@{}l@{}}Thông tin đăng ký được cập nhật (status = Cancelled) trong DB. \\ Slot đăng ký được giải phóng, SV có thể đăng ký môn khác thay thế \\ (UC-04).\end{tabular}} &
   \\ \cline{1-4}
\multirow{9}{*}{\textbf{Luồng sự kiện}} &
  \multicolumn{1}{l|}{\textbf{Bước}} &
  \multicolumn{1}{l|}{\textbf{Thực hiện bởi}} &
  \textbf{Mô tả} &
   \\ \cline{2-4}
 &
  \multicolumn{1}{l|}{1} &
  \multicolumn{1}{l|}{Sinh viên} &
  Đăng nhập hệ thống &
   \\ \cline{2-4}
 &
  \multicolumn{1}{l|}{2} &
  \multicolumn{1}{l|}{Sinh viên} &
  Chọn chức năng “Hủy đăng ký môn học”. &
   \\ \cline{2-4}
 &
  \multicolumn{1}{l|}{3} &
  \multicolumn{1}{l|}{Hệ thống} &
  \begin{tabular}[c]{@{}l@{}}Hiển thị danh sách môn học mà sinh viên đã \\ đăng ký\end{tabular} &
   \\ \cline{2-4}
 &
  \multicolumn{1}{l|}{4} &
  \multicolumn{1}{l|}{Sinh viên} &
  Chọn 1 môn học muốn hủy &
   \\ \cline{2-4}
 &
  \multicolumn{1}{l|}{5} &
  \multicolumn{1}{l|}{Hệ thống} &
  Yêu cầu xác nhận thao tác &
   \\ \cline{2-4}
 &
  \multicolumn{1}{l|}{6} &
  \multicolumn{1}{l|}{Sinh viên} &
  Xác nhận hủy &
   \\ \cline{2-4}
 &
  \multicolumn{1}{l|}{7} &
  \multicolumn{1}{l|}{Hệ thống} &
  \begin{tabular}[c]{@{}l@{}}Cập nhật Database, đổi trạng thái đăng ký \\ thành "Cancelled"\end{tabular} &
   \\ \cline{2-4}
 &
  \multicolumn{1}{l|}{8} &
  \multicolumn{1}{l|}{Hệ thống} &
  Hiển thị thông báo "Hủy đăng ký thành công" &
   \\ \cline{1-4}
\multirow{3}{*}{\textbf{Luồng thay thế}} &
  \multicolumn{1}{l|}{\textbf{Bước}} &
  \multicolumn{1}{l|}{\textbf{Thực hiện bởi}} &
  \textbf{Mô tả} &
   \\ \cline{2-4}
 &
  \multicolumn{1}{l|}{3a} &
  \multicolumn{1}{l|}{Hệ thống} &
  \begin{tabular}[c]{@{}l@{}}Phát hiện SV chưa đăng ký môn nào \\ → hiển thị “Không có môn để hủy”.\end{tabular} &
   \\ \cline{2-4}
 &
  \multicolumn{1}{l|}{5a} &
  \multicolumn{1}{l|}{Sinh viên} &
  \begin{tabular}[c]{@{}l@{}}Bấm “Không đồng ý” khi xác nhận hủy → Hệ \\ thống quay lại danh sách môn đã đăng ký. \\ ( quay lại bước 3.)\end{tabular} &
   \\ \cline{1-4}
\textbf{Ngoại lệ} &
  \multicolumn{3}{l|}{Hệ thống lỗi khi cập nhật DB → hiển thị “Hủy thất bại, thử lại sau”.} &
   \\ \cline{1-4}
\textbf{Business Rules} &
  \multicolumn{3}{l|}{\begin{tabular}[c]{@{}l@{}}SV chỉ có thể hủy môn học chưa bắt đầu trước 1 tuần. Sau khi hủy, \\ SV có thể đăng ký môn khác miễn tổng số môn $\leq$ 4 (UC-04).\end{tabular}} &
   \\ \cline{1-4}
\textbf{Data requirement} &
  \multicolumn{3}{l|}{\begin{tabular}[c]{@{}l@{}}ProgramRegistration(regID, studentID, subjectID, purpose, regDate, \\ status) \\ (cập nhật status = “Cancelled”).\end{tabular}} &
   \\ \cline{1-4}
\end{tabular}
\caption{Bảng đặc tả chức năng hủy đăng ký môn học}
\end{table}
\end{samepage}

%========================================================================================

%========================================================================================

\newpage
\subsection*{2.4.3. Module Ghép cặp Tutor – SV}
\addcontentsline{toc}{subsection}{2.4.3. Module Ghép cặp Tutor – SV}
% \newpage
\subsubsection*{Use Case 06: Ghép thủ công (SV chọn Tutor)}
\begin{samepage}
\begin{table}[h!]
\begin{tabular}{|l|lll|l}
\cline{1-4}
\textbf{ID} &
  \multicolumn{3}{l|}{UC-06} &
   \\ \cline{1-4}
\textbf{Tên} &
  \multicolumn{3}{l|}{Ghép thủ công (SV chọn Tutor cho từng môn học)} &
   \\ \cline{1-4}
\textbf{Mô tả} &
  \multicolumn{3}{l|}{\begin{tabular}[c]{@{}l@{}}Sinh viên tự chọn Tutor từ danh sách đề xuất để tham gia chương trình \\ học.\end{tabular}} &
   \\ \cline{1-4}
\textbf{Actor chính} &
  \multicolumn{3}{l|}{Sinh viên} &
   \\ \cline{1-4}
\textbf{Actor phụ} &
  \multicolumn{3}{l|}{Tutor, hệ thống} &
   \\ \cline{1-4}
\textbf{Tiền điều kiện} &
  \multicolumn{3}{l|}{\begin{tabular}[c]{@{}l@{}}Sinh viên đã đăng ký ít nhất một môn học trong chương trình (UC-04)\\ Có ít nhất 1 Tutor phù hợp trong hệ thống cho môn đó\end{tabular}} &
   \\ \cline{1-4}
\textbf{Hậu điều kiện} &
  \multicolumn{3}{l|}{\begin{tabular}[c]{@{}l@{}}Cặp SV – Tutor được lưu trong DB với môn học cụ thể. \\ Một SV có thể có nhiều bản ghi ghép cho nhiều môn khác nhau. \\ Trạng thái ghép = “Đang hoạt động”.\end{tabular}} &
   \\ \cline{1-4}
\multirow{11}{*}{\textbf{Luồng sự kiện}} &
  \multicolumn{1}{l|}{\textbf{Bước}} &
  \multicolumn{1}{l|}{\textbf{Thực hiện bởi}} &
  \textbf{Mô tả} &
   \\ \cline{2-4}
 &
  \multicolumn{1}{l|}{1} &
  \multicolumn{1}{l|}{Sinh viên} &
  \begin{tabular}[c]{@{}l@{}}Đăng nhập, chọn chức năng “Chọn \\ Tutor thủ công”.\end{tabular} &
   \\ \cline{2-4}
 &
  \multicolumn{1}{l|}{2} &
  \multicolumn{1}{l|}{Hệ thống} &
  \begin{tabular}[c]{@{}l@{}}Hiển thị danh sách các môn học SV đã \\ đăng ký.\end{tabular} &
   \\ \cline{2-4}
 &
  \multicolumn{1}{l|}{3} &
  \multicolumn{1}{l|}{Sinh viên} &
  Chọn 1 môn học để tìm Tutor. &
   \\ \cline{2-4}
 &
  \multicolumn{1}{l|}{4} &
  \multicolumn{1}{l|}{Hệ thống} &
  \begin{tabular}[c]{@{}l@{}}Hiển thị danh sách Tutor phù hợp cho \\ môn học đó (theo lịch rảnh, slot còn trống, \\ chuyên môn).\end{tabular} &
   \\ \cline{2-4}
 &
  \multicolumn{1}{l|}{5} &
  \multicolumn{1}{l|}{Sinh viên} &
  Chọn một Tutor từ danh sách. &
   \\ \cline{2-4}
 &
  \multicolumn{1}{l|}{6} &
  \multicolumn{1}{l|}{Hệ thống} &
  \begin{tabular}[c]{@{}l@{}}Kiểm tra slot của Tutor (còn chỗ / chưa vượt \\ maxSV).\end{tabular} &
   \\ \cline{2-4}
 &
  \multicolumn{1}{l|}{7} &
  \multicolumn{1}{l|}{Hệ thống} &
  \begin{tabular}[c]{@{}l@{}}Nếu hợp lệ → Hệ thống lưu kết quả ghép \\ (SV – Tutor – Môn học) vào DB.\end{tabular} &
   \\ \cline{2-4}
 &
  \multicolumn{1}{l|}{8} &
  \multicolumn{1}{l|}{Hệ thống} &
  Hiển thị thông báo “Ghép thành công” cho SV. &
   \\ \cline{2-4}
 &
  \multicolumn{1}{l|}{9} &
  \multicolumn{1}{l|}{Hệ thống} &
  \begin{tabular}[c]{@{}l@{}}Gửi thông báo đến Tutor (Tutor chỉ nhận \\ thông báo, không được từ chối).\end{tabular} &
   \\ \cline{2-4}
 &
  \multicolumn{1}{l|}{10} &
  \multicolumn{1}{l|}{Sinh viên} &
  \begin{tabular}[c]{@{}l@{}}Có thể lặp lại quy trình cho các môn khác nếu \\ muốn.\end{tabular} &
   \\ \cline{1-4}
\multirow{4}{*}{\textbf{Luồng thay thế}} &
  \multicolumn{1}{l|}{\textbf{Bước}} &
  \multicolumn{1}{l|}{\textbf{Thực hiện bởi}} &
  \textbf{Mô tả} &
   \\ \cline{2-4}
 &
  \multicolumn{1}{l|}{3a} &
  \multicolumn{1}{l|}{Hệ thống} &
  \begin{tabular}[c]{@{}l@{}}Không tìm thấy Tutor phù hợp cho môn học đã \\ chọn → hiển thị “Chưa có Tutor khả dụng, vui \\ lòng thử lại sau”.\end{tabular} &
   \\ \cline{2-4}
 &
  \multicolumn{1}{l|}{6a} &
  \multicolumn{1}{l|}{Hệ thống} &
  \begin{tabular}[c]{@{}l@{}}Tutor đã full slot → hệ thống hiển thị “Tutor \\ đã đủ số lượng SV, vui lòng chọn Tutor khác”\end{tabular} &
   \\ \cline{2-4}
 &
  \multicolumn{1}{l|}{7a} &
  \multicolumn{1}{l|}{Hệ thống} &
  \begin{tabular}[c]{@{}l@{}}Nếu SV không hài lòng → chọn chức năng \\ “Hủy ghép” (UC-10) cho môn học đó, rồi \\ quay lại UC-06 để chạy lại matching.\end{tabular} &
   \\ \cline{1-4}
\textbf{Ngoại lệ} &
  \multicolumn{3}{l|}{\begin{tabular}[c]{@{}l@{}}Hệ thống lỗi kết nối khi gửi yêu cầu đến Tutor → hiển thị “Thao tác thất \\ bại, thử lại sau”.\end{tabular}} &
   \\ \cline{1-4}

% \end{tabular}
% \end{table}
% \end{samepage}

% \newpage
% \begin{samepage}
% \begin{table}[h!]
% \begin{tabular}{|l|lll|l}
\textbf{Business Rules} &
  \multicolumn{3}{l|}{\begin{tabular}[c]{@{}l@{}}Một SV có thể có nhiều Tutor khác nhau cho nhiều môn học khác nhau. \\ Với mỗi môn học, tại một thời điểm chỉ có 1 Tutor chính (thủ công \\ hoặc tự động). \\ Tutor chỉ nhận tối đa số SV theo quy định (VD: $\leq$ 10).\end{tabular}} &
   \\ \cline{1-4}
\textbf{Data requirement} &
  \multicolumn{3}{l|}{\begin{tabular}[c]{@{}l@{}}Matching(matchID,studentID,tutorID,subjectID,status,createdAt) \\ subjectID đảm bảo SV có thể được ghép nhiều lần (mỗi môn một Tutor).\end{tabular}} &
   \\ \cline{1-4}
\end{tabular}
\caption{Bảng đặc tả chức năng ghép thủ công (SV chọn Tutor cho từng môn học)}
\end{table}
\end{samepage}

%========================================================================================

\newpage
\subsubsection*{Use Case 07: Ghép tự động (Hệ thống đề xuất Tutor)}
\begin{samepage}
\begin{table}[h!]
\begin{tabular}{|l|lll|l}
\cline{1-4}
\textbf{ID} &
  \multicolumn{3}{l|}{UC-07} &
   \\ \cline{1-4}
\textbf{Tên} &
  \multicolumn{3}{l|}{Ghép tự động (hệ thống đề xuất Tutor theo từng môn học)} &
   \\ \cline{1-4}
\textbf{Mô tả} &
  \multicolumn{3}{l|}{\begin{tabular}[c]{@{}l@{}}Hệ thống tự động tìm và ghép Tutor phù hợp với SV cho từng môn học \\ dựa trên lịch rảnh của Tutor và ưu tiên mà SV chọn (VD: lịch 2-4-6, \\ buổi tối). \\ Một SV có thể được ghép với nhiều Tutor khác nhau, mỗi Tutor phụ \\ trách một môn riêng.\end{tabular}} &
   \\ \cline{1-4}
\textbf{Actor chính} &
  \multicolumn{3}{l|}{Sinh viên} &
   \\ \cline{1-4}
\textbf{Actor phụ} &
  \multicolumn{3}{l|}{Tutor, hệ thống} &
   \\ \cline{1-4}
\textbf{Tiền điều kiện} &
  \multicolumn{3}{l|}{\begin{tabular}[c]{@{}l@{}}SV đã đăng ký một hoặc nhiều môn học trong chương trình (UC-04). \\ Tutor đã cập nhật lịch rảnh và hồ sơ môn học (UC-08).\end{tabular}} &
   \\ \cline{1-4}
\textbf{Hậu điều kiện} &
  \multicolumn{3}{l|}{\begin{tabular}[c]{@{}l@{}}SV được ghép thành công với ít nhất 1 Tutor cho môn học đã chọn.\\ Nếu SV học nhiều môn, hệ thống có thể ghép nhiều Tutor khác nhau \\ (mỗi môn 1 Tutor). \\ Kết quả lưu vào DB.\\ SV có thể hủy ghép với 1 môn cụ thể và thử lại mà không ảnh hưởng \\ các môn khác.\end{tabular}} &
   \\ \cline{1-4}
\multirow{11}{*}{\textbf{Luồng sự kiện}} &
  \multicolumn{1}{l|}{\textbf{Bước}} &
  \multicolumn{1}{l|}{\textbf{Thực hiện bởi}} &
  \textbf{Mô tả} &
   \\ \cline{2-4}
 &
  \multicolumn{1}{l|}{1} &
  \multicolumn{1}{l|}{Sinh viên} &
  Đăng nhập, chọn chức năng “Ghép Tutor tự động”. &
   \\ \cline{2-4}
 &
  \multicolumn{1}{l|}{2} &
  \multicolumn{1}{l|}{Hệ thống} &
  Hiển thị danh sách các môn học SV đã đăng ký. &
   \\ \cline{2-4}
 &
  \multicolumn{1}{l|}{3} &
  \multicolumn{1}{l|}{Sinh viên} &
  Chọn 1 môn để thực hiện ghép Tutor &
   \\ \cline{2-4}
 &
  \multicolumn{1}{l|}{4} &
  \multicolumn{1}{l|}{Hệ thống} &
  \begin{tabular}[c]{@{}l@{}}Hiển thị form để SV chọn ưu tiên cho môn đó:\\ Khung lịch mong muốn (VD: 2-4-6, 3-5-7, \\ buổi sáng/chiều/tối).\end{tabular} &
   \\ \cline{2-4}
 &
  \multicolumn{1}{l|}{5} &
  \multicolumn{1}{l|}{Sinh viên} &
  Xác nhận lựa chọn. &
   \\ \cline{2-4}
 &
  \multicolumn{1}{l|}{6} &
  \multicolumn{1}{l|}{Hệ thống} &
  \begin{tabular}[c]{@{}l@{}}Lấy danh sách Tutor khả dụng từ DB theo tiêu chí:\\ Có lịch rảnh trùng với ưu tiên.\\ Có chuyên môn đúng môn học SV vừa chọn.\\ Còn slot trống.\end{tabular} &
   \\ \cline{2-4}
 &
  \multicolumn{1}{l|}{7} &
  \multicolumn{1}{l|}{Hệ thống} &
  \begin{tabular}[c]{@{}l@{}}Chạy thuật toán so khớp → chọn Tutor phù hợp \\ nhất cho môn học đó.\end{tabular} &
   \\ \cline{2-4}
 &
  \multicolumn{1}{l|}{8} &
  \multicolumn{1}{l|}{Hệ thống} &
  \begin{tabular}[c]{@{}l@{}}Lưu thông tin ghép cặp vào DB với subjectID \\ + tutorID.\end{tabular} &
   \\ \cline{2-4}
 &
  \multicolumn{1}{l|}{9} &
  \multicolumn{1}{l|}{Hệ thống} &
  Gửi thông báo kết quả ghép cho SV và Tutor. &
   \\ \cline{2-4}
 &
  \multicolumn{1}{l|}{10} &
  \multicolumn{1}{l|}{Sinh viên} &
  Muốn ghép thêm môn khác → lặp lại quy trình. &
   \\ \cline{1-4}
\multirow{3}{*}{\textbf{Luồng thay thế}} &
  \multicolumn{1}{l|}{\textbf{Bước}} &
  \multicolumn{1}{l|}{\textbf{Thực hiện bởi}} &
  \textbf{Mô tả} &
   \\ \cline{2-4}
 &
  \multicolumn{1}{l|}{6a} &
  \multicolumn{1}{l|}{Hệ thống} &
  \begin{tabular}[c]{@{}l@{}}Không tìm thấy Tutor phù hợp cho môn học → \\ hiển thị thông báo “Chưa có Tutor phù hợp, vui \\ lòng thử lại sau”.\end{tabular} &
   \\ \cline{2-4}
 &
  \multicolumn{1}{l|}{9a} &
  \multicolumn{1}{l|}{Hệ thống} &
  \begin{tabular}[c]{@{}l@{}}Nếu SV không hài lòng → chọn chức năng \\ “Hủy ghép” (UC-10) cho môn học đó, rồi \\ quay lại UC-07 để chạy lại matching.\end{tabular} &
   \\ \cline{1-4}
\end{tabular}
% \caption{Bảng đặc tả chức năng ghép tự động (hệ thống đề xuất Tutor theo từng môn học)}
\end{table}
\end{samepage}

\newpage
\begin{samepage}
\begin{table}[h!]
\begin{tabular}{|l|lll|l}
\textbf{Ngoại lệ} &
  \multicolumn{3}{l|}{\begin{tabular}[c]{@{}l@{}}Hệ thống lỗi kết nối khi gửi yêu cầu đến Tutor → hiển thị “Thao tác thất \\ bại, thử lại sau”.\end{tabular}} &
   \\ \cline{1-4}
\textbf{Business Rules} &
  \multicolumn{3}{l|}{\begin{tabular}[c]{@{}l@{}}Một SV có thể được ghép với nhiều Tutor khác nhau cho các môn khác \\ nhau. \\ Với mỗi môn học, tại 1 thời điểm chỉ có 1 Tutor (manual hoặc auto).\\ Một SV không được ghép song song 2 Tutor cho cùng một môn (trừ \\ khi hủy ghép trước đó). \\ Tutor không thể vượt quá số lượng SV tối đa cho môn mình phụ trách. \\ SV chỉ được chọn lịch nằm trong slot rảnh do Tutor khai báo.\end{tabular}} &
   \\ \cline{1-4}
\textbf{Data requirement} &
  \multicolumn{3}{l|}{\begin{tabular}[c]{@{}l@{}}Matching(matchID,studentID,tutorID,subjectID,method,status,createdAt\\ ,preferences) \\ method = "manual" hoặc "auto". \\ subjectID cho phép 1 SV có nhiều bản ghi ghép (mỗi môn 1 bản ghi, \\ mỗi bản ghi có 1 Tutor khác nhau).\end{tabular}} &
   \\ \cline{1-4}
\end{tabular}
\caption{Bảng đặc tả chức năng ghép tự động (hệ thống đề xuất Tutor theo từng môn học)}
\end{table}
\end{samepage}
%========================================================================================

%========================================================================================

\newpage
\subsection*{2.4.4. Module Quản lý lịch học}
\addcontentsline{toc}{subsection}{2.4.4. Module Quản lý lịch học}
\newpage
\subsubsection*{Use Case 08: Tạo lịch rảnh (Tutor)
}
\begin{samepage}

\end{samepage}

%========================================================================================

\newpage
\subsubsection*{Use Case 09: Đặt lịch học (SV)}
\begin{samepage}

\end{samepage}

%========================================================================================

\newpage
\subsubsection*{Use Case 10: Hủy/Đổi lịch học cố định}
\begin{samepage}

\end{samepage}

%========================================================================================

%========================================================================================

\newpage
\subsection*{2.4.5. Module Thông báo và nhắc nhở}
\addcontentsline{toc}{subsection}{2.4.5. Module Thông báo và nhắc nhở}
\newpage
\subsubsection*{Use Case 11: Gửi thông báo lịch học}
\begin{samepage}
\begin{table}[h!]
\begin{tabular}{|l|lll|l}
\cline{1-4}
\textbf{ID} &
  \multicolumn{3}{l|}{UC-11} &
   \\ \cline{1-4}
\textbf{Tên} &
  \multicolumn{3}{l|}{Gửi thông báo lịch học} &
   \\ \cline{1-4}
\textbf{Mô tả} &
  \multicolumn{3}{l|}{\begin{tabular}[c]{@{}l@{}}Hệ thống gửi thông báo đến SV và Tutor khi có lịch học mới được đặt \\ hoặc thay đổi.\end{tabular}} &
   \\ \cline{1-4}
\textbf{Actor chính} &
  \multicolumn{3}{l|}{Hệ thống} &
   \\ \cline{1-4}
\textbf{Actor phụ} &
  \multicolumn{3}{l|}{Sinh viên, Tutor} &
   \\ \cline{1-4}
\textbf{Tiền điều kiện} &
  \multicolumn{3}{l|}{\begin{tabular}[c]{@{}l@{}}Lịch học đã được tạo hoặc có thay đổi (UC-09, UC-10).\\ SV và Tutor đã kích hoạt tài khoản.\end{tabular}} &
   \\ \cline{1-4}
\textbf{Hậu điều kiện} &
  \multicolumn{3}{l|}{SV và Tutor nhận được thông báo về lịch học.} &
   \\ \cline{1-4}
\multirow{5}{*}{\textbf{Luồng sự kiện}} &
  \multicolumn{1}{l|}{\textbf{Bước}} &
  \multicolumn{1}{l|}{\textbf{Thực hiện bởi}} &
  \textbf{Mô tả} &
   \\ \cline{2-4}
 &
  \multicolumn{1}{l|}{1} &
  \multicolumn{1}{l|}{Hệ thống} &
  Phát hiện có lịch học mới/hủy/đổi. &
   \\ \cline{2-4}
 &
  \multicolumn{1}{l|}{2} &
  \multicolumn{1}{l|}{Hệ thống} &
  \begin{tabular}[c]{@{}l@{}}Tạo thông báo gồm: ngày, giờ, môn học, trạng \\ thái (mới/hủy/đổi).\end{tabular} &
   \\ \cline{2-4}
 &
  \multicolumn{1}{l|}{3} &
  \multicolumn{1}{l|}{Hệ thống} &
  Gửi thông báo qua email/app cho SV và Tutor. &
   \\ \cline{2-4}
 &
  \multicolumn{1}{l|}{4} &
  \multicolumn{1}{l|}{\begin{tabular}[c]{@{}l@{}}Người dùng\\ (SV/Tutor)\end{tabular}} &
  Nhận và đọc thông báo. &
   \\ \cline{1-4}
\multirow{2}{*}{\textbf{Luồng thay thế}} &
  \multicolumn{1}{l|}{\textbf{Bước}} &
  \multicolumn{1}{l|}{\textbf{Thực hiện bởi}} &
  \textbf{Mô tả} &
   \\ \cline{2-4}
 &
  \multicolumn{1}{l|}{4a} &
  \multicolumn{1}{l|}{\begin{tabular}[c]{@{}l@{}}Người dùng\\ (SV/Tutor)\end{tabular}} &
  \begin{tabular}[c]{@{}l@{}}chưa mở app → Hệ thống lưu thông báo trong \\ mục “Thông báo” để xem sau.\end{tabular} &
   \\ \cline{1-4}
\textbf{Ngoại lệ} &
  \multicolumn{3}{l|}{Hệ thống lỗi khi gửi email → chỉ lưu thông báo trong app.} &
   \\ \cline{1-4}
\textbf{Business Rules} &
  \multicolumn{3}{l|}{\begin{tabular}[c]{@{}l@{}}Tất cả thay đổi về lịch học phải được thông báo trong vòng 5 phút kể \\ từ khi cập nhật. \\ Nội dung thông báo phải có đủ thông tin: môn, ngày, giờ, hình thức học.\end{tabular}} &
   \\ \cline{1-4}
\textbf{Data requirement} &
  \multicolumn{3}{l|}{Notification(notifID, userID, type, content, createdAt, status)} &
   \\ \cline{1-4}
\end{tabular}
\caption{Bảng đặc tả chức năng gửi thông báo lịch học}
\end{table}
\end{samepage}

%========================================================================================

\newpage
\subsubsection*{Use Case 12: Gửi nhắc nhở buổi học}
\begin{samepage}

\end{samepage}

%========================================================================================



%========================================================================================

\newpage
\subsection*{2.4.6. Module Quản lý buổi học và điểm danh}
\addcontentsline{toc}{subsection}{2.4.6. Module Quản lý buổi học và điểm danh}
\newpage
\subsubsection*{Use Case 13: Điểm danh sinh viên}
\begin{samepage}
\begin{table}[h!]
\begin{tabular}{|l|lll|l}
\cline{1-4}
\textbf{ID} &
  \multicolumn{3}{l|}{UC-13} &
   \\ \cline{1-4}
\textbf{Tên} &
  \multicolumn{3}{l|}{Điểm danh sinh viên} &
   \\ \cline{1-4}
\textbf{Mô tả} &
  \multicolumn{3}{l|}{Tutor điểm danh sự tham gia của sinh viên trong buổi học.} &
   \\ \cline{1-4}
\textbf{Actor chính} &
  \multicolumn{3}{l|}{Sinh viên, Tutor} &
   \\ \cline{1-4}
\textbf{Actor phụ} &
  \multicolumn{3}{l|}{Hệ thống} &
   \\ \cline{1-4}
\textbf{Tiền điều kiện} &
  \multicolumn{3}{l|}{\begin{tabular}[c]{@{}l@{}}Buổi học đã được xác nhận và diễn ra đúng lịch.\\ SV và Tutor đã đăng nhập hệ thống.\end{tabular}} &
   \\ \cline{1-4}
\textbf{Hậu điều kiện} &
  \multicolumn{3}{l|}{Trạng thái điểm danh của SV được lưu trong DB.} &
   \\ \cline{1-4}
\multirow{7}{*}{\textbf{Luồng sự kiện}} &
  \multicolumn{1}{l|}{\textbf{Bước}} &
  \multicolumn{1}{l|}{\textbf{Thực hiện bởi}} &
  \textbf{Mô tả} &
   \\ \cline{2-4}
 &
  \multicolumn{1}{l|}{1} &
  \multicolumn{1}{l|}{Tutor} &
  Đăng nhập và mở danh sách buổi học trong ngày. &
   \\ \cline{2-4}
 &
  \multicolumn{1}{l|}{2} &
  \multicolumn{1}{l|}{Tutor} &
  Chọn buổi học cần điểm danh. &
   \\ \cline{2-4}
 &
  \multicolumn{1}{l|}{3} &
  \multicolumn{1}{l|}{Hệ thống} &
  Hiển thị danh sách SV đã đăng ký buổi học. &
   \\ \cline{2-4}
 &
  \multicolumn{1}{l|}{4} &
  \multicolumn{1}{l|}{Tutor} &
  Đánh dấu SV có mặt/vắng mặt./đi trễ &
   \\ \cline{2-4}
 &
  \multicolumn{1}{l|}{5} &
  \multicolumn{1}{l|}{Hệ thống} &
  Lưu kết quả điểm danh vào DB. &
   \\ \cline{2-4}
 &
  \multicolumn{1}{l|}{6} &
  \multicolumn{1}{l|}{Hệ thống} &
  Hiển thị thông báo “Điểm danh thành công”. &
   \\ \cline{1-4}
\multirow{2}{*}{\textbf{Luồng thay thế}} &
  \multicolumn{1}{l|}{\textbf{Bước}} &
  \multicolumn{1}{l|}{\textbf{Thực hiện bởi}} &
  \textbf{Mô tả} &
   \\ \cline{2-4}
 &
  \multicolumn{1}{l|}{3a} &
  \multicolumn{1}{l|}{Hệ thống} &
  \begin{tabular}[c]{@{}l@{}}Nếu buổi học không có SV nào đăng ký \\ → Hệ thống hiển thị “Không có SV tham gia”.\end{tabular} &
   \\ \cline{1-4}
\textbf{Ngoại lệ} &
  \multicolumn{3}{l|}{DB lỗi khi lưu → hiển thị “Điểm danh thất bại, thử lại sau} &
   \\ \cline{1-4}
\textbf{Business Rules} &
  \multicolumn{3}{l|}{\begin{tabular}[c]{@{}l@{}}Chỉ Tutor phụ trách buổi học mới có quyền điểm danh.\\ SV vắng mặt \textgreater 2 lần liên tiếp → hệ thống thông báo cho Khoa.\\ Điểm danh chỉ được thực hiện trong khoảng thời gian buổi học diễn ra.\\ Tutor không thể thay đổi điểm danh sau khi buổi học kết thúc \\ (trừ khi có xác nhận từ Khoa).\end{tabular}} &
   \\ \cline{1-4}
\textbf{Data requirement} &
  \multicolumn{3}{l|}{Attendance(attID, scheduleID, studentID, status, attTime)} &
   \\ \cline{1-4}
\end{tabular}
\caption{Bảng đặc tả chức năng điểm danh sinh viên}
\end{table}
\end{samepage}

%========================================================================================

\newpage
\subsubsection*{Use Case 14: Cập nhật trạng thái buổi học}
\begin{samepage}
\begin{table}[h!]
\begin{tabular}{|l|lll|l}
\cline{1-4}
\textbf{ID} &
  \multicolumn{3}{l|}{UC-14} &
   \\ \cline{1-4}
\textbf{Tên} &
  \multicolumn{3}{l|}{Cập nhật trạng thái buổi học} &
   \\ \cline{1-4}
\textbf{Mô tả} &
  \multicolumn{3}{l|}{\begin{tabular}[c]{@{}l@{}}Tutor hoặc hệ thống cập nhật trạng thái buổi học (chưa diễn ra, đang \\ diễn ra, đã hoàn tất, bị hủy).\end{tabular}} &
   \\ \cline{1-4}
\textbf{Actor chính} &
  \multicolumn{3}{l|}{Tutor, Hệ thống} &
   \\ \cline{1-4}
\textbf{Actor phụ} &
  \multicolumn{3}{l|}{Sinh viên} &
   \\ \cline{1-4}
\textbf{Tiền điều kiện} &
  \multicolumn{3}{l|}{Buổi học đã được xác nhận trong hệ thống.} &
   \\ \cline{1-4}
\textbf{Hậu điều kiện} &
  \multicolumn{3}{l|}{Trạng thái buổi học được cập nhật chính xác trong DB.} &
   \\ \cline{1-4}
\multirow{5}{*}{\textbf{Luồng sự kiện}} &
  \multicolumn{1}{l|}{\textbf{Bước}} &
  \multicolumn{1}{l|}{\textbf{Thực hiện bởi}} &
  \textbf{Mô tả} &
   \\ \cline{2-4}
 &
  \multicolumn{1}{l|}{1} &
  \multicolumn{1}{l|}{Hệ thống} &
  \begin{tabular}[c]{@{}l@{}}Trước giờ học, gán trạng thái buổi học là \\ “Sắp diễn ra”.\end{tabular} &
   \\ \cline{2-4}
 &
  \multicolumn{1}{l|}{2} &
  \multicolumn{1}{l|}{Tutor} &
  \begin{tabular}[c]{@{}l@{}}Chọn “Bắt đầu buổi học” → hệ thống \\ cập nhật trạng thái = “Đang diễn ra”.\end{tabular} &
   \\ \cline{2-4}
 &
  \multicolumn{1}{l|}{3} &
  \multicolumn{1}{l|}{Tutor} &
  \begin{tabular}[c]{@{}l@{}}Chọn “Hoàn tất buổi học” → hệ thống cập |\\ nhật trạng thái = “Đã hoàn tất”.\end{tabular} &
   \\ \cline{2-4}
 &
  \multicolumn{1}{l|}{4} &
  \multicolumn{1}{l|}{Hệ thống} &
  Lưu log thay đổi trạng thái vào DB. &
   \\ \cline{1-4}
\multirow{3}{*}{\textbf{Luồng thay thế}} &
  \multicolumn{1}{l|}{\textbf{Bước}} &
  \multicolumn{1}{l|}{\textbf{Thực hiện bởi}} &
  \textbf{Mô tả} &
   \\ \cline{2-4}
 &
  \multicolumn{1}{l|}{2a} &
  \multicolumn{1}{l|}{Tutor} &
  \begin{tabular}[c]{@{}l@{}}Nếu Tutor quên cập nhật → hệ thống tự động \\ đổi trạng thái sang “Đã hoàn tất” sau khi quá \\ giờ học.\end{tabular} &
   \\ \cline{2-4}
 &
  \multicolumn{1}{l|}{3a} &
  \multicolumn{1}{l|}{Tutor} &
  \begin{tabular}[c]{@{}l@{}}Nếu buổi học bị hủy trước khi bắt đầu → Hệ \\ thống cập nhập trạng thái = “Đã hủy”.\end{tabular} &
   \\ \cline{1-4}
\textbf{Ngoại lệ} &
  \multicolumn{3}{l|}{Hệ thống lỗi khi cập nhật DB → hiển thị “Thao tác thất bại, thử lại sau”.} &
   \\ \cline{1-4}
\textbf{Business Rules} &
  \multicolumn{3}{l|}{\begin{tabular}[c]{@{}l@{}}Trạng thái buổi học chỉ có thể thay đổi theo luồng hợp lệ:\\ “Sắp diễn ra” → “Đang diễn ra” → “Đã hoàn tất”\\ hoặc “Sắp diễn ra” → “Đã hủy”\\ SV chỉ xem được trạng thái, không được chỉnh sửa.\\ Nếu buổi học bị hủy, Hệ thống phải gửi thông báo ngay cho SV.\\ Trạng thái buổi học phải được cập nhật trong vòng 24h sau khi kết thúc.\end{tabular}} &
   \\ \cline{1-4}
\textbf{Data requirement} &
  \multicolumn{3}{l|}{\begin{tabular}[c]{@{}l@{}}Session(sessionID, tutorID, studentID, subjectID, date, startTime, \\ endTime, status, updatedAt)\end{tabular}} &
   \\ \cline{1-4}
\end{tabular}
\caption{Bảng đặc tả chức năng cập nhật trạng thái buổi học}
\end{table}
\end{samepage}

%========================================================================================



%========================================================================================

\newpage
\subsection*{2.4.7. Module Quản lý tài liệu học tập}
\addcontentsline{toc}{subsection}{2.4.7. Module Quản lý tài liệu học tập}
\newpage
\subsubsection*{Use Case 15: Quản lý tài liệu (Tutor)}
\begin{samepage}
\begin{table}[h!]
\begin{tabular}{|l|lll|l}
\cline{1-4}
\textbf{ID} &
  \multicolumn{3}{l|}{UC-15} &
   \\ \cline{1-4}
\textbf{Tên} &
  \multicolumn{3}{l|}{Quản lý tài liệu (Tutor)} &
   \\ \cline{1-4}
\textbf{Mô tả} &
  \multicolumn{3}{l|}{\begin{tabular}[c]{@{}l@{}}Tutor quản lý tài liệu học tập (slide, đề cương, bài tập) cho môn học hoặc \\ buổi học: upload, chỉnh sửa, xóa.\end{tabular}} &
   \\ \cline{1-4}
\textbf{Actor chính} &
  \multicolumn{3}{l|}{Tutor} &
   \\ \cline{1-4}
\textbf{Actor phụ} &
  \multicolumn{3}{l|}{Sinh viên, Hệ thống} &
   \\ \cline{1-4}
\textbf{Tiền điều kiện} &
  \multicolumn{3}{l|}{\begin{tabular}[c]{@{}l@{}}Tutor đã đăng nhập hệ thống.\\ Buổi học/chương trình học đã được xác nhận.\end{tabular}} &
   \\ \cline{1-4}
\textbf{Hậu điều kiện} &
  \multicolumn{3}{l|}{Tài liệu được lưu trữ trong hệ thống và hiển thị cho SV liên quan.} &
   \\ \cline{1-4}
\multirow{4}{*}{\textbf{Luồng sự kiện}} &
  \multicolumn{1}{l|}{\textbf{Bước}} &
  \multicolumn{1}{l|}{\textbf{Thực hiện bởi}} &
  \textbf{Mô tả} &
   \\ \cline{2-4}
 &
  \multicolumn{1}{l|}{1} &
  \multicolumn{1}{l|}{Tutor} &
  Đăng nhập và chọn “Quản lý tài liệu”. &
   \\ \cline{2-4}
 &
  \multicolumn{1}{l|}{2} &
  \multicolumn{1}{l|}{Tutor} &
  Chọn môn học/buổi học liên quan. &
   \\ \cline{2-4}
 &
  \multicolumn{1}{l|}{3} &
  \multicolumn{1}{l|}{Tutor} &
  \begin{tabular}[c]{@{}l@{}}Chọn thao tác:\\ Upload: chọn file (PDF, PPT, DOCX), nhập mô tả \\ → Hệ thống kiểm tra định dạng, dung lượng → \\ lưu file + metadata → hiển thị “Upload thành \\ công”.\\ Chỉnh sửa: chọn tài liệu → cập nhật tên/mô tả/file \\ mới → Hệ thống kiểm tra hợp lệ → cập nhật DB \\ → hiển thị “Cập nhật thành công”.\\ Xóa: chọn tài liệu → xác nhận → Hệ thống xóa \\ bản ghi và file → hiển thị “Xóa thành công”.\end{tabular} &
   \\ \cline{1-4}
\multirow{5}{*}{\textbf{Luồng thay thế}} &
  \multicolumn{1}{l|}{\textbf{Bước}} &
  \multicolumn{1}{l|}{\textbf{Thực hiện bởi}} &
  \textbf{Mô tả} &
   \\ \cline{2-4}
 &
  \multicolumn{1}{l|}{3a} &
  \multicolumn{1}{l|}{Tutor} &
  \begin{tabular}[c]{@{}l@{}}Không nhập mô tả khi upload → Hệ thống nhắc \\ nhập lại.\end{tabular} &
   \\ \cline{2-4}
 &
  \multicolumn{1}{l|}{3b} &
  \multicolumn{1}{l|}{Tutor} &
  File trùng tên → yêu cầu đổi tên &
   \\ \cline{2-4}
 &
  \multicolumn{1}{l|}{3c} &
  \multicolumn{1}{l|}{Tutor} &
  File sai định dạng → báo lỗi. &
   \\ \cline{2-4}
 &
  \multicolumn{1}{l|}{3d} &
  \multicolumn{1}{l|}{Tutor} &
  \begin{tabular}[c]{@{}l@{}}File vượt dung lượng tối đa (ví dụ \textgreater 50MB) \\ → từ chối upload.\end{tabular} &
   \\ \cline{1-4}
\textbf{Ngoại lệ} &
  \multicolumn{3}{l|}{Hệ thống lỗi khi lưu file/DB → hiển thị “Thao tác thất bại, thử lại sau”.} &
   \\ \cline{1-4}
\textbf{Business Rules} &
  \multicolumn{3}{l|}{\begin{tabular}[c]{@{}l@{}}Tài liệu phải gắn với một môn học hoặc buổi học cụ thể.\\ Tutor chỉ được chỉnh sửa/xóa tài liệu do chính mình upload.\end{tabular}} &
   \\ \cline{1-4}
\textbf{Data requirement} &
  \multicolumn{3}{l|}{\begin{tabular}[c]{@{}l@{}}Materials(materialID, tutorID, subjectID, sessionID, fileName, filePath, \\ fileSize, description, uploadedAt, updatedAt)\end{tabular}} &
   \\ \cline{1-4}
\end{tabular}
\caption{Bảng đặc tả chức năng quản lý tài liệu (Tutor)}
\end{table}
\end{samepage}

%========================================================================================

\newpage
\subsubsection*{Use Case 16: Sinh viên tải tài liệu}
\begin{samepage}
\begin{table}[h!]
\begin{tabular}{|l|lll|l}
\cline{1-4}
\textbf{ID}                              & \multicolumn{3}{l|}{UC-16}                                                                                                  &  \\ \cline{1-4}
\textbf{Tên}                             & \multicolumn{3}{l|}{SV tải tài liệu}                                                                                        &  \\ \cline{1-4}
\textbf{Mô tả}                           & \multicolumn{3}{l|}{Sinh viên tải tài liệu học tập do Tutor chia sẻ để phục vụ học tập.}                                    &  \\ \cline{1-4}
\textbf{Actor chính}                     & \multicolumn{3}{l|}{Sinh viên}                                                                                              &  \\ \cline{1-4}
\textbf{Actor phụ}                       & \multicolumn{3}{l|}{Tutor, Hệ thống}                                                                                        &  \\ \cline{1-4}
\textbf{Tiền điều kiện} &
  \multicolumn{3}{l|}{\begin{tabular}[c]{@{}l@{}}SV đã đăng nhập hệ thống.\\ Tutor đã upload ít nhất một tài liệu cho môn học. (UC-15).\end{tabular}} &
   \\ \cline{1-4}
\textbf{Hậu điều kiện}                   & \multicolumn{3}{l|}{SV tải xuống và sử dụng tài liệu học tập.}                                                              &  \\ \cline{1-4}
\multirow{7}{*}{\textbf{Luồng sự kiện}}  & \multicolumn{1}{l|}{\textbf{Bước}} & \multicolumn{1}{l|}{\textbf{Thực hiện bởi}} & \textbf{Mô tả}                           &  \\ \cline{2-4}
                                         & \multicolumn{1}{l|}{1}             & \multicolumn{1}{l|}{Sinh viên}              & Đăng nhập, chọn môn học.                 &  \\ \cline{2-4}
                                         & \multicolumn{1}{l|}{2}             & \multicolumn{1}{l|}{Hệ thống}               & Hiển thị danh sách tài liệu của môn học. &  \\ \cline{2-4}
                                         & \multicolumn{1}{l|}{3}             & \multicolumn{1}{l|}{Sinh viên}              & Chọn file muốn tải.                      &  \\ \cline{2-4}
 &
  \multicolumn{1}{l|}{4} &
  \multicolumn{1}{l|}{Hệ thống} &
  \begin{tabular}[c]{@{}l@{}}Kiểm tra quyền truy cập.\\ (SV thuộc lớp/buổi học đó).\end{tabular} &
   \\ \cline{2-4}
                                         & \multicolumn{1}{l|}{5}             & \multicolumn{1}{l|}{Hệ thống}               & Gửi file để SV tải về.                   &  \\ \cline{2-4}
                                         & \multicolumn{1}{l|}{6}             & \multicolumn{1}{l|}{Sinh viên}              & Nhận và lưu file vào máy.                &  \\ \cline{1-4}
\multirow{3}{*}{\textbf{Luồng thay thế}} & \multicolumn{1}{l|}{\textbf{Bước}} & \multicolumn{1}{l|}{\textbf{Thực hiện bởi}} & \textbf{Mô tả}                           &  \\ \cline{2-4}
 &
  \multicolumn{1}{l|}{2a} &
  \multicolumn{1}{l|}{Hệ thống} &
  \begin{tabular}[c]{@{}l@{}}Nếu chưa có tài liệu → Hệ thống hiển thị “Chưa \\ có tài liệu được chia sẻ”.\end{tabular} &
   \\ \cline{2-4}
 &
  \multicolumn{1}{l|}{4a} &
  \multicolumn{1}{l|}{Hệ thống} &
  \begin{tabular}[c]{@{}l@{}}Nếu SV không thuộc buổi học → Hệ thống hiển \\ thị “Bạn không có quyền tải tài liệu này”.\end{tabular} &
   \\ \cline{1-4}
\textbf{Ngoại lệ}                        & \multicolumn{3}{l|}{Hệ thống lỗi server khi gửi file → hiển thị “Tải xuống thất bại”.}                                      &  \\ \cline{1-4}
\textbf{Business Rules} &
  \multicolumn{3}{l|}{\begin{tabular}[c]{@{}l@{}}SV chỉ tải tài liệu thuộc môn học mình đã đăng ký.\\ Hệ thống ghi log mỗi lần SV tải tài liệu.\end{tabular}} &
   \\ \cline{1-4}
\textbf{Data requirement}                & \multicolumn{3}{l|}{DownloadLog(logID, studentID, materialID, downloadTime, status)}                                        &  \\ \cline{1-4}
\end{tabular}
\caption{Bảng đặc tả chức năng sinh viên tải tài liệu}
\end{table}
\end{samepage}

%========================================================================================



%========================================================================================

\newpage
\subsection*{2.4.8. Module Đánh giá và phản hồi}
\addcontentsline{toc}{subsection}{2.4.8. Module Đánh giá và phản hồi}
\newpage
\subsubsection*{Use Case 17: Sinh viên đánh giá Tutor}
\begin{samepage}
\begin{table}[h!]
\begin{tabular}{|l|lll|l}
\cline{1-4}
\textbf{ID} &
  \multicolumn{3}{l|}{UC-17} &
   \\ \cline{1-4}
\textbf{Tên} &
  \multicolumn{3}{l|}{SV đánh giá Tutor} &
   \\ \cline{1-4}
\textbf{Mô tả} &
  \multicolumn{3}{l|}{\begin{tabular}[c]{@{}l@{}}Sau khi môn học kết thúc, sinh viên có thể đánh giá chất lượng Tutor để \\ phản hồi về hiệu quả hỗ trợ.\end{tabular}} &
   \\ \cline{1-4}
\textbf{Actor chính} &
  \multicolumn{3}{l|}{Sinh viên} &
   \\ \cline{1-4}
\textbf{Actor phụ} &
  \multicolumn{3}{l|}{Tutor, Hệ thống, Khoa/BM} &
   \\ \cline{1-4}
\textbf{Tiền điều kiện} &
  \multicolumn{3}{l|}{\begin{tabular}[c]{@{}l@{}}Môn học đã hoàn tất.\\ SV đã tham gia (có điểm danh $\geq$ 1 buổi trong môn).\end{tabular}} &
   \\ \cline{1-4}
\textbf{Hậu điều kiện} &
  \multicolumn{3}{l|}{Feedback được lưu vào hệ thống và gắn với hồ sơ Tutor.} &
   \\ \cline{1-4}
\multirow{7}{*}{\textbf{Luồng sự kiện}} &
  \multicolumn{1}{l|}{\textbf{Bước}} &
  \multicolumn{1}{l|}{\textbf{Thực hiện bởi}} &
  \textbf{Mô tả} &
   \\ \cline{2-4}
 &
  \multicolumn{1}{l|}{1} &
  \multicolumn{1}{l|}{Sinh viên} &
  Đăng nhập và chọn “Đánh giá Tutor”. &
   \\ \cline{2-4}
 &
  \multicolumn{1}{l|}{2} &
  \multicolumn{1}{l|}{Hệ thống} &
  Hiển thị danh sách môn học đã hoàn tất. &
   \\ \cline{2-4}
 &
  \multicolumn{1}{l|}{3} &
  \multicolumn{1}{l|}{Sinh viên} &
  Chọn môn học muốn đánh giá. &
   \\ \cline{2-4}
 &
  \multicolumn{1}{l|}{4} &
  \multicolumn{1}{l|}{Sinh viên} &
  Nhập rating (1–5, bắt buộc) và nhận xét (tùy chọn). &
   \\ \cline{2-4}
 &
  \multicolumn{1}{l|}{5} &
  \multicolumn{1}{l|}{Hệ thống} &
  Lưu đánh giá vào DB. &
   \\ \cline{2-4}
 &
  \multicolumn{1}{l|}{6} &
  \multicolumn{1}{l|}{Hệ thống} &
  Hiển thị thông báo “Đánh giá thành công”. &
   \\ \cline{1-4}
\multirow{3}{*}{\textbf{Luồng thay thế}} &
  \multicolumn{1}{l|}{\textbf{Bước}} &
  \multicolumn{1}{l|}{\textbf{Thực hiện bởi}} &
  \textbf{Mô tả} &
   \\ \cline{2-4}
 &
  \multicolumn{1}{l|}{2a} &
  \multicolumn{1}{l|}{Hệ thống} &
  \begin{tabular}[c]{@{}l@{}}Nếu không có môn học nào đã hoàn tất → hiển \\ thị “Chưa có môn học để đánh giá”.\end{tabular} &
   \\ \cline{2-4}
 &
  \multicolumn{1}{l|}{4a} &
  \multicolumn{1}{l|}{Sinh viên} &
  Nếu SV bỏ trống rating → hệ thống nhắc nhập. &
   \\ \cline{1-4}
\textbf{Ngoại lệ} &
  \multicolumn{3}{l|}{Lỗi khi lưu DB → hiển thị “Đánh giá thất bại, thử lại sau”.} &
   \\ \cline{1-4}
\textbf{Business Rules} &
  \multicolumn{3}{l|}{\begin{tabular}[c]{@{}l@{}}Mỗi SV chỉ được đánh giá 1 lần cho mỗi môn.\\ Rating bắt buộc, nhận xét không bắt buộc.\\ Feedback SV gửi → Tutor \& Khoa đều xem được.\\ Không cho sửa sau khi gửi.\end{tabular}} &
   \\ \cline{1-4}
\textbf{Data requirement} &
  \multicolumn{3}{l|}{\begin{tabular}[c]{@{}l@{}}Feedback(feedbackID, fromUserID, toUserID, subjectID, roleFrom,\\  roleTo, rating, comment, createdAt)\\  (roleFrom = "SV", roleTo = "Tutor")\end{tabular}} &
   \\ \cline{1-4}
\end{tabular}
\caption{Bảng đặc tả chức năng sinh viên đánh giá Tutor}
\end{table}
\end{samepage}

%========================================================================================

\newpage
\subsubsection*{Use Case 18: Tutor đánh giá sinh viên}
\begin{samepage}
\begin{table}[h!]
\begin{tabular}{|l|lll|l}
\cline{1-4}
\textbf{ID} &
  \multicolumn{3}{l|}{UC-18} &
   \\ \cline{1-4}
\textbf{Tên} &
  \multicolumn{3}{l|}{Tutor đánh giá sinh viên} &
   \\ \cline{1-4}
\textbf{Mô tả} &
  \multicolumn{3}{l|}{\begin{tabular}[c]{@{}l@{}}Sau khi kết thúc môn học, Tutor có thể đánh giá mức độ tham gia, thái \\ độ học tập của SV để phản hồi cho Khoa/BM.\end{tabular}} &
   \\ \cline{1-4}
\textbf{Actor chính} &
  \multicolumn{3}{l|}{Tutor} &
   \\ \cline{1-4}
\textbf{Actor phụ} &
  \multicolumn{3}{l|}{SV, Hệ thống, Khoa/BM} &
   \\ \cline{1-4}
\textbf{Tiền điều kiện} &
  \multicolumn{3}{l|}{Môn học đã hoàn tất.SV đã tham gia ít nhất 1 buổi.} &
   \\ \cline{1-4}
\textbf{Hậu điều kiện} &
  \multicolumn{3}{l|}{\begin{tabular}[c]{@{}l@{}}Feedback của Tutor được lưu trong DB, gắn với hồ sơ SV.\\ Khoa/BM có thể xem dữ liệu này trong báo cáo tổng hợp.\end{tabular}} &
   \\ \cline{1-4}
\multirow{7}{*}{\textbf{Luồng sự kiện}} &
  \multicolumn{1}{l|}{\textbf{Bước}} &
  \multicolumn{1}{l|}{\textbf{Thực hiện bởi}} &
  \textbf{Mô tả} &
   \\ \cline{2-4}
 &
  \multicolumn{1}{l|}{1} &
  \multicolumn{1}{l|}{Tutor} &
  Đăng nhập, chọn “Đánh giá sinh viên”. &
   \\ \cline{2-4}
 &
  \multicolumn{1}{l|}{2} &
  \multicolumn{1}{l|}{Hệ thống} &
  Hiển thị danh sách SV đã dạy. &
   \\ \cline{2-4}
 &
  \multicolumn{1}{l|}{3} &
  \multicolumn{1}{l|}{Tutor} &
  Chọn SV cần đánh giá. &
   \\ \cline{2-4}
 &
  \multicolumn{1}{l|}{4} &
  \multicolumn{1}{l|}{Tutor} &
  \begin{tabular}[c]{@{}l@{}}Nhập:\\ Rating (1–5, bắt buộc)\\ Nhận xét (tùy chọn)\\ Tiêu chí (chuyên cần, thái độ, hợp tác–tùy chọn)\end{tabular} &
   \\ \cline{2-4}
 &
  \multicolumn{1}{l|}{5} &
  \multicolumn{1}{l|}{Hệ thống} &
  Lưu dữ liệu vào DB. &
   \\ \cline{2-4}
 &
  \multicolumn{1}{l|}{6} &
  \multicolumn{1}{l|}{Hệ thống} &
  Hiển thị “Đánh giá thành công”. &
   \\ \cline{1-4}
\multirow{3}{*}{\textbf{Luồng thay thế}} &
  \multicolumn{1}{l|}{\textbf{Bước}} &
  \multicolumn{1}{l|}{\textbf{Thực hiện bởi}} &
  \textbf{Mô tả} &
   \\ \cline{2-4}
 &
  \multicolumn{1}{l|}{2a} &
  \multicolumn{1}{l|}{Hệ thống} &
  \begin{tabular}[c]{@{}l@{}}Nếu chưa có SV nào đã học xong → hiển thị \\ “Không có sinh viên để đánh giá”.\end{tabular} &
   \\ \cline{2-4}
 &
  \multicolumn{1}{l|}{4a} &
  \multicolumn{1}{l|}{Tutor} &
  Nếu Tutor bỏ trống rating → hệ thống nhắc nhập. &
   \\ \cline{1-4}
\textbf{Ngoại lệ} &
  \multicolumn{3}{l|}{Lỗi DB → hiển thị “Đánh giá thất bại, thử lại sau”.} &
   \\ \cline{1-4}
\textbf{Business Rules} &
  \multicolumn{3}{l|}{\begin{tabular}[c]{@{}l@{}}Mỗi SV chỉ được Tutor đánh giá 1 lần/môn.\\ Rating bắt buộc, nhận xét/tiêu chí không bắt buộc.\\ Feedback của Tutor chỉ hiển thị cho Khoa/BM (không gửi cho SV).\\ Dùng cho mục đích quản lý chất lượng.\end{tabular}} &
   \\ \cline{1-4}
\textbf{Data requirement} &
  \multicolumn{3}{l|}{\begin{tabular}[c]{@{}l@{}}Feedback(feedbackID, fromUserID, toUserID, subjectID, roleFrom, \\ roleTo, rating, comment, criteria, createdAt)\\ (roleFrom = "Tutor", roleTo = "SV")\end{tabular}} &
   \\ \cline{1-4}
\end{tabular}
\caption{Bảng đặc tả chức năng Tutor đánh giá sinh viên}
\end{table}
\end{samepage}

%========================================================================================

\newpage
\subsubsection*{Use Case 19: Khoa/BM tổng hợp đánh giá}
\begin{samepage}

\end{samepage}

%========================================================================================


%========================================================================================

\newpage
\subsection*{2.4.9. Module Thống kê và báo cáo}
\addcontentsline{toc}{subsection}{2.4.9. Module Thống kê và báo cáo}
\newpage
\subsubsection*{Use Case 20: Báo cáo kết quả học tập SV}
\begin{samepage}
\begin{table}[h!]
\begin{tabular}{|l|lll|l}
\cline{1-4}
\textbf{ID} &
  \multicolumn{3}{l|}{UC-20} &
   \\ \cline{1-4}
\textbf{Tên} &
  \multicolumn{3}{l|}{Báo cáo kết quả học tập SV} &
   \\ \cline{1-4}
\textbf{Mô tả} &
  \multicolumn{3}{l|}{\begin{tabular}[c]{@{}l@{}}Khoa tổng hợp và xác nhận kết quả học tập của SV sau khi môn học \\ kết thúc, sau đó hệ thống hiển thị cho SV.\end{tabular}} &
   \\ \cline{1-4}
\textbf{Actor chính} &
  \multicolumn{3}{l|}{Khoa} &
   \\ \cline{1-4}
\textbf{Actor phụ} &
  \multicolumn{3}{l|}{Hệ thống, SV} &
   \\ \cline{1-4}
\textbf{Tiền điều kiện} &
  \multicolumn{3}{l|}{\begin{tabular}[c]{@{}l@{}}SV đã tham gia ít nhất 1 buổi học có điểm danh.\\ Dữ liệu điểm danh, trạng thái buổi học, đánh giá Tutor đã được cập nhật.\end{tabular}} &
   \\ \cline{1-4}
\textbf{Hậu điều kiện} &
  \multicolumn{3}{l|}{Báo cáo đã được xác nhận bởi Khoa và hiển thị cho SV.} &
   \\ \cline{1-4}
\multirow{6}{*}{\textbf{Luồng sự kiện}} &
  \multicolumn{1}{l|}{\textbf{Bước}} &
  \multicolumn{1}{l|}{\textbf{Thực hiện bởi}} &
  \textbf{Mô tả} &
   \\ \cline{2-4}
 &
  \multicolumn{1}{l|}{1} &
  \multicolumn{1}{l|}{Khoa} &
  Đăng nhập, chọn “Tổng hợp kết quả học tập SV”. &
   \\ \cline{2-4}
 &
  \multicolumn{1}{l|}{2} &
  \multicolumn{1}{l|}{Hệ thống} &
  \begin{tabular}[c]{@{}l@{}}Lấy dữ liệu điểm danh, trạng thái buổi học và \\ đánh giá Tutor.\end{tabular} &
   \\ \cline{2-4}
 &
  \multicolumn{1}{l|}{3} &
  \multicolumn{1}{l|}{Hệ thống} &
  \begin{tabular}[c]{@{}l@{}}Tạo báo cáo sơ bộ: số buổi đăng ký, số buổi tham \\ gia, số buổi vắng, \% tham gia, phản hồi từ Tutor.\end{tabular} &
   \\ \cline{2-4}
 &
  \multicolumn{1}{l|}{4} &
  \multicolumn{1}{l|}{Khoa} &
  Kiểm tra và xác nhận báo cáo. &
   \\ \cline{2-4}
 &
  \multicolumn{1}{l|}{5} &
  \multicolumn{1}{l|}{Hệ thống} &
  \begin{tabular}[c]{@{}l@{}}Hiển thị báo cáo cho SV và cho phép SV xuất \\ file (PDF/Excel).\end{tabular} &
   \\ \cline{1-4}
\multirow{2}{*}{\textbf{Luồng thay thế}} &
  \multicolumn{1}{l|}{\textbf{Bước}} &
  \multicolumn{1}{l|}{\textbf{Thực hiện bởi}} &
  \textbf{Mô tả} &
   \\ \cline{2-4}
 &
  \multicolumn{1}{l|}{2a} &
  \multicolumn{1}{l|}{Hệ thống} &
  \begin{tabular}[c]{@{}l@{}}Nếu SV chưa tham gia buổi học nào → Hệ \\ thống hiển thị “Chưa có dữ liệu để báo cáo”.\end{tabular} &
   \\ \cline{1-4}
\textbf{Ngoại lệ} &
  \multicolumn{3}{l|}{Lỗi khi truy xuất DB → hiển thị “Không thể tạo báo cáo”.} &
   \\ \cline{1-4}
\textbf{Business Rules} &
  \multicolumn{3}{l|}{\begin{tabular}[c]{@{}l@{}}SV chỉ xem được báo cáo của chính mình.\\ Khoa có quyền xem \& xác nhận báo cáo cho SV trong phạm vi \\ quản lý.\\ Báo cáo phải bao gồm: số buổi đăng ký, số buổi tham gia, số buổi \\ vắng, \% tham gia.\end{tabular}} &
   \\ \cline{1-4}
\textbf{Data requirement} &
  \multicolumn{3}{l|}{\begin{tabular}[c]{@{}l@{}}Attendance(attID, scheduleID, studentID, status)\\ Session(sessionID, status)\\ StudentReport(studentID, totalSessions, attendedSessions, \\ attendanceRate, lastUpdated, approvedBy, approvedAt)\end{tabular}} &
   \\ \cline{1-4}
\end{tabular}
\caption{Bảng đặc tả chức năng báo cáo kết quả học tập sinh viên}
\end{table}
\end{samepage}

%========================================================================================

\newpage
\subsubsection*{Use Case 21: Báo cáo chất lượng Tutor}
\begin{samepage}
\begin{table}[h!]
\begin{tabular}{|l|lll|l}
\cline{1-4}
\textbf{ID} &
  \multicolumn{3}{l|}{UC-21} &
   \\ \cline{1-4}
\textbf{Tên} &
  \multicolumn{3}{l|}{Báo cáo chất lượng Tutor} &
   \\ \cline{1-4}
\textbf{Mô tả} &
  \multicolumn{3}{l|}{\begin{tabular}[c]{@{}l@{}}Khoa tổng hợp và xác nhận báo cáo chất lượng giảng dạy của Tutor dựa \\ trên feedback từ SV và dữ liệu buổi học.\end{tabular}} &
   \\ \cline{1-4}
\textbf{Actor chính} &
  \multicolumn{3}{l|}{Khoa} &
   \\ \cline{1-4}
\textbf{Actor phụ} &
  \multicolumn{3}{l|}{Hệ thống, Tutor} &
   \\ \cline{1-4}
\textbf{Tiền điều kiện} &
  \multicolumn{3}{l|}{\begin{tabular}[c]{@{}l@{}}Tutor đã có ít nhất 1 buổi học hoàn tất.\\ Có feedback từ SV.\end{tabular}} &
   \\ \cline{1-4}
\textbf{Hậu điều kiện} &
  \multicolumn{3}{l|}{Báo cáo chất lượng được xác nhận và hiển thị cho Tutor.} &
   \\ \cline{1-4}
\multirow{6}{*}{\textbf{Luồng sự kiện}} &
  \multicolumn{1}{l|}{\textbf{Bước}} &
  \multicolumn{1}{l|}{\textbf{Thực hiện bởi}} &
  \textbf{Mô tả} &
   \\ \cline{2-4}
 &
  \multicolumn{1}{l|}{1} &
  \multicolumn{1}{l|}{Khoa} &
  Đăng nhập, chọn “Báo cáo chất lượng Tutor”. &
   \\ \cline{2-4}
 &
  \multicolumn{1}{l|}{2} &
  \multicolumn{1}{l|}{Hệ thống} &
  \begin{tabular}[c]{@{}l@{}}Lấy dữ liệu feedback, điểm danh, trạng thái \\ buổi học.\end{tabular} &
   \\ \cline{2-4}
 &
  \multicolumn{1}{l|}{3} &
  \multicolumn{1}{l|}{Hệ thống} &
  \begin{tabular}[c]{@{}l@{}}Tổng hợp báo cáo sơ bộ (điểm trung bình, số \\ SV đánh giá, tỷ lệ buổi thành công).\end{tabular} &
   \\ \cline{2-4}
 &
  \multicolumn{1}{l|}{4} &
  \multicolumn{1}{l|}{Khoa} &
  Kiểm tra và xác nhận báo cáo. &
   \\ \cline{2-4}
 &
  \multicolumn{1}{l|}{5} &
  \multicolumn{1}{l|}{Hệ thống} &
  \begin{tabular}[c]{@{}l@{}}Hiển thị báo cáo cho Tutor và cho phép SV xuất \\ file (PDF/Excel).\end{tabular} &
   \\ \cline{1-4}
\multirow{2}{*}{\textbf{Luồng thay thế}} &
  \multicolumn{1}{l|}{\textbf{Bước}} &
  \multicolumn{1}{l|}{\textbf{Thực hiện bởi}} &
  \textbf{Mô tả} &
   \\ \cline{2-4}
 &
  \multicolumn{1}{l|}{2a} &
  \multicolumn{1}{l|}{Hệ thống} &
  \begin{tabular}[c]{@{}l@{}}Nếu không có feedback → Hệ thống hiển thị \\ “Chưa có dữ liệu đánh giá”.\end{tabular} &
   \\ \cline{1-4}
\textbf{Ngoại lệ} &
  \multicolumn{3}{l|}{Lỗi khi truy xuất DB → hiển thị “Không thể tạo báo cáo”} &
   \\ \cline{1-4}
\textbf{Business Rules} &
  \multicolumn{3}{l|}{\begin{tabular}[c]{@{}l@{}}Tutor chỉ xem được báo cáo của chính mình.\\ Khoa có quyền xem báo cáo của tất cả Tutor trong khoa.\\ Báo cáo phải có trạng thái xác nhận trước khi công bố cho Tutor.\end{tabular}} &
   \\ \cline{1-4}
\textbf{Data requirement} &
  \multicolumn{3}{l|}{\begin{tabular}[c]{@{}l@{}}Feedback(feedbackID, tutorID, rating, comment)\\ TutorReport(tutorID, avgRating, totalFeedback, successRate, \\ approvedBy, approvedAt, generatedAt)\end{tabular}} &
   \\ \cline{1-4}
\end{tabular}
\caption{Bảng đặc tả chức năng báo cáo chất lượng Tutor}
\end{table}
\end{samepage}

%========================================================================================

\newpage
\subsubsection*{Use Case 22: Báo cáo tổng hợp (Khoa, PCTSV, PĐT)}
\begin{samepage}
\begin{table}[h!]
\begin{tabular}{|l|lll|l}
\cline{1-4}
\textbf{ID} &
  \multicolumn{3}{l|}{UC-22} &
   \\ \cline{1-4}
\textbf{Tên} &
  \multicolumn{3}{l|}{Báo cáo tổng hợp (Khoa, PCTSV, PĐT)} &
   \\ \cline{1-4}
\textbf{Mô tả} &
  \multicolumn{3}{l|}{\begin{tabular}[c]{@{}l@{}}Khoa, PCTSV và PĐT có thể xem báo cáo tổng hợp toàn hệ thống sau \\ khi dữ liệu từ các báo cáo con (SV, Tutor) đã được xác nhận.\end{tabular}} &
   \\ \cline{1-4}
\textbf{Actor chính} &
  \multicolumn{3}{l|}{Khoa / PCTSV / PĐT} &
   \\ \cline{1-4}
\textbf{Actor phụ} &
  \multicolumn{3}{l|}{Hệ thống} &
   \\ \cline{1-4}
\textbf{Tiền điều kiện} &
  \multicolumn{3}{l|}{\begin{tabular}[c]{@{}l@{}}Các báo cáo học tập SV (UC-20) và chất lượng Tutor (UC-21) đã \\ được xác nhận.\\ Người dùng có quyền xem báo cáo tổng hợp.\end{tabular}} &
   \\ \cline{1-4}
\textbf{Hậu điều kiện} &
  \multicolumn{3}{l|}{Báo cáo tổng hợp hiển thị cho cấp quản lý, có thể xuất file.} &
   \\ \cline{1-4}
\multirow{6}{*}{\textbf{Luồng sự kiện}} &
  \multicolumn{1}{l|}{\textbf{Bước}} &
  \multicolumn{1}{l|}{\textbf{Thực hiện bởi}} &
  \textbf{Mô tả} &
   \\ \cline{2-4}
 &
  \multicolumn{1}{l|}{1} &
  \multicolumn{1}{l|}{\begin{tabular}[c]{@{}l@{}}Người dùng \\ (Khoa/PCTSV\\ /PĐT)\end{tabular}} &
  Đăng nhập, chọn “Báo cáo tổng hợp”. &
   \\ \cline{2-4}
 &
  \multicolumn{1}{l|}{2} &
  \multicolumn{1}{l|}{Hệ thống} &
  \begin{tabular}[c]{@{}l@{}}Lấy dữ liệu từ các báo cáo đã xác nhận \\ (UC-20, UC-21).\end{tabular} &
   \\ \cline{2-4}
 &
  \multicolumn{1}{l|}{3} &
  \multicolumn{1}{l|}{Hệ thống} &
  \begin{tabular}[c]{@{}l@{}}Tổng hợp theo phạm vi:\\ Khoa: theo môn/Tutor/SV trong khoa.\\ PCTSV: tình hình hỗ trợ SV toàn trường.\\ PĐT: hiệu quả chương trình, đề xuất cải tiến.\end{tabular} &
   \\ \cline{2-4}
 &
  \multicolumn{1}{l|}{4} &
  \multicolumn{1}{l|}{Hệ thống} &
  Hiển thị báo cáo dạng bảng và biểu đồ. &
   \\ \cline{2-4}
 &
  \multicolumn{1}{l|}{5} &
  \multicolumn{1}{l|}{Người dùng} &
  Có thể export file Excel/PDF. &
   \\ \cline{1-4}
\multirow{2}{*}{\textbf{Luồng thay thế}} &
  \multicolumn{1}{l|}{\textbf{Bước}} &
  \multicolumn{1}{l|}{\textbf{Thực hiện bởi}} &
  \textbf{Mô tả} &
   \\ \cline{2-4}
 &
  \multicolumn{1}{l|}{2a} &
  \multicolumn{1}{l|}{Hệ thống} &
  \begin{tabular}[c]{@{}l@{}}Nếu một số báo cáo chưa xác nhận → Hệ \\ thống hiển thị cảnh báo nhưng vẫn hiển thị \\ dữ liệu đã có.\end{tabular} &
   \\ \cline{1-4}
\textbf{Ngoại lệ} &
  \multicolumn{3}{l|}{Lỗi khi truy xuất DB lớn → hiển thị “Không thể tạo báo cáo, thử lại sau”.} &
   \\ \cline{1-4}
\textbf{Business Rules} &
  \multicolumn{3}{l|}{\begin{tabular}[c]{@{}l@{}}Khoa chỉ thấy dữ liệu thuộc khoa.\\ PCTSV/PĐT thấy dữ liệu toàn trường.\\ Báo cáo tổng hợp được cập nhật theo kỳ (tuần/tháng) sau khi dữ liệu đã \\ được xác nhận.\end{tabular}} &
   \\ \cline{1-4}
\textbf{Data requirement} &
  \multicolumn{3}{l|}{\begin{tabular}[c]{@{}l@{}}StudentReport(reportID, studentID, subjectID, totalSessions,\\ attendedSessions, absentSessions, attendanceRate, tutorFeedback, \\ lastUpdated, approvedBy, approvedAt)\\ TutorReport(reportID, tutorID,totalSessions, completedSessions, \\ cancelledSessions, avgRating, totalFeedback, positiveRate, successRate, \\ lastUpdated, approvedBy, approvedAt )\\ SystemReport(reportID, type, scope, approvedBy, approvedAt, \\ generatedAt, filePath)\end{tabular}} &
   \\ \cline{1-4}
\end{tabular}
\caption{Bảng đặc tả chức năng báo cáo tổng hợp (Khoa, PCTSV, PĐT)}
\end{table}
\end{samepage}

%========================================================================================


%========================================================================================

\newpage
\subsection*{2.4.10. Module Chương trình học thuật và phi học thuật}
\addcontentsline{toc}{subsection}{2.4.10. Module Chương trình học thuật và phi học thuật}
\newpage
\subsubsection*{Use Case 23: Tutor tạo chương trình học}
\begin{samepage}
\begin{table}[h!]
\begin{tabular}{|l|lll|l}
\cline{1-4}
\textbf{ID} &
  \multicolumn{3}{l|}{UC-23} &
   \\ \cline{1-4}
\textbf{Tên} &
  \multicolumn{3}{l|}{Tutor tạo chương trình học} &
   \\ \cline{1-4}
\textbf{Mô tả} &
  \multicolumn{3}{l|}{\begin{tabular}[c]{@{}l@{}}Tutor tạo và công bố chương trình học mới (học thuật hoặc phi học thuật) \\ để SV đăng ký.\end{tabular}} &
   \\ \cline{1-4}
\textbf{Actor chính} &
  \multicolumn{3}{l|}{Tutor} &
   \\ \cline{1-4}
\textbf{Actor phụ} &
  \multicolumn{3}{l|}{Hệ thống, SV} &
   \\ \cline{1-4}
\textbf{Tiền điều kiện} &
  \multicolumn{3}{l|}{Tutor đã đăng nhập và được xác minh} &
   \\ \cline{1-4}
\textbf{Hậu điều kiện} &
  \multicolumn{3}{l|}{Chương trình học được hiển thị cho SV.} &
   \\ \cline{1-4}
\multirow{6}{*}{\textbf{Luồng sự kiện}} &
  \multicolumn{1}{l|}{\textbf{Bước}} &
  \multicolumn{1}{l|}{\textbf{Thực hiện bởi}} &
  \textbf{Mô tả} &
   \\ \cline{2-4}
 &
  \multicolumn{1}{l|}{1} &
  \multicolumn{1}{l|}{Tutor} &
  Đăng nhập, chọn “Tạo chương trình học”. &
   \\ \cline{2-4}
 &
  \multicolumn{1}{l|}{2} &
  \multicolumn{1}{l|}{Tutor} &
  \begin{tabular}[c]{@{}l@{}}Nhập thông tin chương trình: loại (học thuật/phi \\ học thuật), tên, mô tả, thời lượng.\end{tabular} &
   \\ \cline{2-4}
 &
  \multicolumn{1}{l|}{3} &
  \multicolumn{1}{l|}{Hệ thống} &
  Kiểm tra thông tin hợp lệ &
   \\ \cline{2-4}
 &
  \multicolumn{1}{l|}{4} &
  \multicolumn{1}{l|}{Hệ thống} &
  Lưu và hiển thị chương trình trên hệ thống. &
   \\ \cline{2-4}
 &
  \multicolumn{1}{l|}{5} &
  \multicolumn{1}{l|}{Sinh viên} &
  Có thể thấy chương trình để đăng ký. &
   \\ \cline{1-4}
\multirow{2}{*}{\textbf{Luồng thay thế}} &
  \multicolumn{1}{l|}{\textbf{Bước}} &
  \multicolumn{1}{l|}{\textbf{Thực hiện bởi}} &
  \textbf{Mô tả} &
   \\ \cline{2-4}
 &
  \multicolumn{1}{l|}{2a} &
  \multicolumn{1}{l|}{Tutor} &
  \begin{tabular}[c]{@{}l@{}}Bỏ trống thông tin bắt buộc → Hệ thống nhắc \\ nhập lại.\end{tabular} &
   \\ \cline{1-4}
\textbf{Ngoại lệ} &
  \multicolumn{3}{l|}{Hệ thống lỗi khi lưu → hiển thị “Tạo chương trình thất bại”.} &
   \\ \cline{1-4}
\textbf{Business Rules} &
  \multicolumn{3}{l|}{\begin{tabular}[c]{@{}l@{}}Tutor chỉ có thể tạo chương trình sau khi được Khoa phê duyệt.\\ Mỗi chương trình phải thuộc một loại: học thuật hoặc phi học thuật.\end{tabular}} &
   \\ \cline{1-4}
\textbf{Data requirement} &
  \multicolumn{3}{l|}{\begin{tabular}[c]{@{}l@{}}Program(programID, tutorID, type, name, description, duration, slots, \\ createdAt, approvedBy, approvedAt, status)\end{tabular}} &
   \\ \cline{1-4}
\end{tabular}
\caption{Bảng đặc tả chức năng Tutor tạo chương trình học}
\end{table}
\end{samepage}

%========================================================================================

\newpage
\subsubsection*{Use Case 24: SV đăng ký chương trình học thuật}
\begin{samepage}

\end{samepage}

%========================================================================================

\newpage
\subsubsection*{Use Case 25: SV đăng ký chương trình phi học thuật}
\begin{samepage}

\end{samepage}

%========================================================================================


%========================================================================================

\section*{2.5. Yêu cầu phi chức năng}
\addcontentsline{toc}{section}{2.5. Yêu cầu phi chức năng}
Để xây dựng một hệ thống kết nối Tutor và Sinh viên thực sự hiệu quả và đáng tin cậy, việc đáp ứng các yêu cầu về chức năng là chưa đủ. Yếu tố quyết định trải nghiệm người dùng và sự thành công lâu dài của dự án nằm ở các Yêu cầu phi chức năng (Non-Functional Requirements). Các tiêu chí này đặt ra những chuẩn mực về tốc độ, bảo mật, độ ổn định và tính dễ sử dụng của hệ thống. Những ràng buộc và tiêu chuẩn dưới đây sẽ là những yếu tố để đảm bảo hệ thống không chỉ hoàn thiện mà còn mang lại sự hài lòng và tin tưởng tuyệt đối cho mọi người dùng, từ sinh viên, Tutor đến các cấp quản lý.

\subsection*{2.5.1. Hiệu năng (Performance Requirements)}
\addcontentsline{toc}{subsection}{2.5.1. Hiệu năng (Performance Requirements)}
\begin{itemize}
    \item \textbf{Mô tả:} Hệ thống phải xử lý nhanh và ổn định cho nhiều người dùng đồng thời.
    \item \textbf{Constraints:}
    \begin{itemize}
        \item Hỗ trợ tối thiểu 500 người dùng đồng thời.
        \item Thời gian phản hồi cho thao tác chính $\leq$ 3 giây.
        \item Thuật toán ghép cặp Tutor–SV chạy trong $\leq$ 5 giây.
    \end{itemize}
    \item \textbf{Acceptance:} Kiểm thử tải (load test) cho thấy hệ thống đáp ứng $\geq$ 95\% request trong 3 giây
\end{itemize}

%========================================================================================

\subsection*{2.5.2. Bảo mật (Security Requirements)}
\addcontentsline{toc}{subsection}{2.5.2. Hiệu năng (Performance Requirements)}
\begin{itemize}
    \item \textbf{Mô tả:} Bảo vệ thông tin người dùng và dữ liệu hệ thống khỏi truy cập trái phép.
    \item \textbf{Constraints:}
    \begin{itemize}
        \item Mã hóa toàn bộ giao tiếp bằng HTTPS (TLS 1.3).
        \item Lưu mật khẩu bằng bcrypt/Argon2.
        \item Xác thực 2FA áp dụng cho Tutor và Admin. 
        \item Khóa tài khoản sau 5 lần nhập sai mật khẩu.
        \item Phân quyền theo role (SV, Tutor, Khoa, PCTSV, PĐT, Admin).
    \end{itemize}
    \item \textbf{Acceptance:} Thử nghiệm penetration test không phát hiện lỗ hổng nghiêm trọng.
\end{itemize}

%========================================================================================

\subsection*{2.5.3. Tính tin cậy \& sẵn sàng (Reliability \& Availability)}
\addcontentsline{toc}{subsection}{2.5.3. Tính tin cậy \& sẵn sàng (Reliability \& Availability)}
\begin{itemize}
    \item \textbf{Mô tả:} Hệ thống phải đảm bảo tính liên tục và phục hồi khi có sự cố.
    \item \textbf{Constraints:} 
    \begin{itemize}
        \item Thời gian uptime $\geq$ 99.5\%/tháng.
        \item Backup dữ liệu hàng ngày, phục hồi $\leq$ 2h.
        \item Retry khi gửi thông báo thất bại.
        \item Log toàn bộ giao dịch quan trọng.
    \end{itemize}
    \item \textbf{Acceptance:} DRP (Disaster Recovery Plan) kiểm thử thành công, phục hồi dữ liệu $\leq$ 2h.
\end{itemize}

%========================================================================================

\subsection*{2.5.4. Khả năng sử dụng (Usability)}
\addcontentsline{toc}{subsection}{2.5.4. Khả năng sử dụng (Usability)}
\begin{itemize}
    \item \textbf{Mô tả:} Giao diện thân thiện, dễ sử dụng cho tất cả loại người dùng.
    \item \textbf{Constraints:} 
    \begin{itemize}
        \item Hỗ trợ đa thiết bị (desktop, mobile, tablet).
        \item Ngôn ngữ: Tiếng Việt (mặc định), Tiếng Anh (tùy chọn).
        \item Người dùng mới có thể đăng ký, đặt lịch trong $\leq$ 5 phút.
        \item Có màn hình trợ giúp/hướng dẫn.
    \end{itemize}
    \item \textbf{Acceptance:} Khảo sát $\geq$ 80\% người dùng đánh giá giao diện “dễ sử dụng”.

\end{itemize}

%========================================================================================

\subsection*{2.5.5. Tính bảo trì \& mở rộng (Maintainability \& Extensibility)}
\addcontentsline{toc}{subsection}{2.5.5. Tính bảo trì \& mở rộng (Maintainability \& Extensibility)}
\begin{itemize}
    \item \textbf{Mô tả:} Hệ thống dễ bảo trì, nâng cấp mà không ảnh hưởng đến chức năng hiện có.
    \item \textbf{Constraints:}
    \begin{itemize}
        \item Tuân thủ mô hình MVC hoặc Microservices.
        \item Code phải có comment, tuân thủ coding convention.
        \item Thêm module mới không ảnh hưởng module cũ.
        \item Bug critical sau release $\leq$ 2\%.
    \end{itemize}
    \item \textbf{Acceptance:} Regression test cho thấy chức năng cũ không bị ảnh hưởng sau khi thêm module mới.
\end{itemize}

%========================================================================================

\subsection*{2.5.6. Khả năng tương thích (Compatibility)}
\addcontentsline{toc}{subsection}{2.5.6. Khả năng tương thích (Compatibility)}
\begin{itemize}
    \item \textbf{Mô tả:} Hệ thống chạy được trên nhiều nền tảng và dịch vụ tích hợp.
    \item \textbf{Constraints:}
    \begin{itemize}
        \item Web chạy trên Chrome, Firefox, Edge, Safari (phiên bản mới nhất).
        \item Mobile app chạy trên Android $\geq$ 10, iOS $\geq$ 13.
        \item Tích hợp email server.
    \end{itemize}
    \item \textbf{Acceptance:} Test cross-browser cho kết quả hiển thị đúng $\geq$ 95\%.
\end{itemize}

%========================================================================================

\subsection*{2.5.7. Ràng buộc kỹ thuật (Technical Constraints)}
\addcontentsline{toc}{subsection}{2.5.7. Ràng buộc kỹ thuật (Technical Constraints)}
\begin{itemize}
    \item \textbf{Mô tả:} Các công nghệ, công cụ và nền tảng bắt buộc sử dụng.
    \item \textbf{Constraints:}
    \begin{itemize}
        \item DB: MySQL hoặc PostgreSQL.
        \item Backend: Java Spring Boot hoặc Node.js.
        \item Frontend: ReactJS hoặc Angular.
        \item Triển khai trên Docker/Kubernetes.
    \end{itemize}
    \item \textbf{Acceptance:} Cấu hình hệ thống triển khai thành công trên môi trường staging/production.
\end{itemize}

%========================================================================================