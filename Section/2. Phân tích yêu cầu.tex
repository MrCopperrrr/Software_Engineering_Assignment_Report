\part*{2. Phân tích yêu cầu dự án}
\addcontentsline{toc}{part}{2. Phân tích yêu cầu dự án}
Trong chương này, nhóm em sẽ tiến hành xác định và phân tích các yêu cầu của hệ thống. Trước hết, việc nhận diện các tác nhân (Actors) tương tác với hệ thống là bước quan trọng để hiểu rõ các luồng chức năng và mục tiêu mà phần mềm cần đáp ứng.

%========================================================================================

\section*{2.1. Tác nhân}
\addcontentsline{toc}{section}{2.1. Tác nhân}
Tác nhân (Actor) là một thực thể bên ngoài, có thể là người dùng hoặc một hệ thống khác, tương tác trực tiếp với hệ thống để thực hiện một mục tiêu cụ thể. Dựa trên vai trò và cách thức tương tác, các tác nhân của "Hệ thống hỗ trợ Tutor" được phân loại thành tác nhân chính và tác nhân phụ.

%========================================================================================

\subsection*{2.1.1. Tác nhân chính}
\addcontentsline{toc}{subsection}{2.1.1. Tác nhân chính}
\subsubsection*{2.1.1.1. Tutor}
\addcontentsline{toc}{subsubsection}{2.1.1.1. Tutor}
Tutor có thể là sinh viên giỏi, NCS, hoặc giảng viên, được đăng ký vào hệ thống để hỗ trợ học tập. Họ có trách nhiệm quản lý hồ sơ, lịch rảnh, tổ chức buổi học và theo dõi tiến độ của sinh viên.
\begin{itemize}
    \item \textbf{Tạo và cập nhật lịch rảnh:}
    \begin{itemize}
        \item \textbf{Input:} Ngày, giờ, hình thức học (online/offline).
        \item \textbf{Process:} Hệ thống lưu lại khung giờ rảnh, đồng bộ với module đặt lịch.
        \item \textbf{Output:} Lịch rảnh hiển thị cho sinh viên để chọn.
        \item \textbf{Constraints:} Không được nhập trùng khung giờ.
        \item \textbf{Acceptance:} Sinh viên có thể đặt lịch học trong khung rảnh của Tutor.
        \item \textbf{Error handling:} Nếu nhập sai định dạng hoặc trùng lịch thì hệ thống báo lỗi.
    \end{itemize}
    
    \item \textbf{Mở (oneline/offline), hủy, đổi buổi học:}
    \begin{itemize}
        \item \textbf{Input:} Yêu cầu mở buổi học, hoặc yêu cầu huỷ/đổi.
        \item \textbf{Process:} Hệ thống kiểm tra lịch đã có sinh viên đặt chưa, xử lý cập nhật.
        \item \textbf{Output:} Buổi học được thêm/sửa/xoá trong hệ thống.
        \item \textbf{Constraints:} Hủy/đổi $\geq$ 3h trước giờ học
        \item \textbf{Acceptance:} Sinh viên và Tutor đều nhận được thông báo cập nhật.
        \item \textbf{Error handling:} Nếu yêu cầu đổi sát giờ hệ thống từ chối, báo lỗi.
    \end{itemize}
    
    \item \textbf{Nhận thông báo và nhắc nhở giờ dạy:}
    \begin{itemize}
        \item \textbf{Input:} Lịch học sắp diễn ra.
        \item \textbf{Process:} Hệ thống gửi thông báo (noti/email).
        \item \textbf{Output:} Tutor nhận được thông báo đúng hạn.
        \item \textbf{Constraints:} Thông báo phải được gửi $\geq$ 30 phút trước giờ học.
        \item \textbf{Acceptance:} Tutor xác nhận đã đọc thông báo.
        \item \textbf{Error handling:} Nếu gửi lỗi → hệ thống gửi lại lần 2 hoặc báo qua email dự phòng.
    \end{itemize}
    
    \item \textbf{Theo dõi tiến bộ sinh viên:}
    \begin{itemize}
        \item \textbf{Input:} Điểm, nhận xét, đánh giá sau buổi học.
        \item \textbf{Process:} Tutor nhập vào form đánh giá, hệ thống lưu lại.
        \item \textbf{Output:} Báo cáo tiến bộ gắn với hồ sơ sinh viên.
        \item \textbf{Constraints:} Chỉ Tutor đã dạy sinh viên đó mới được nhập.
        \item \textbf{Acceptance:} Khoa/bộ môn có thể truy cập báo cáo.
        \item \textbf{Error handling:} Nếu buổi học chưa hoàn tất thì từ chối ghi nhận.
    \end{itemize}
    
    \item \textbf{Điểm danh và record:}
    \begin{itemize}
        \item \textbf{Input:} ID sinh viên tham gia, mã buổi học.
        \item \textbf{Process:} Hệ thống điểm danh, ghi log tham dự.
        \item \textbf{Output:} Record buổi học (thời lượng, người tham gia).
        \item \textbf{Constraints:} Mỗi SV chỉ được điểm danh vào 1 buổi học tại 1 thời điểm.
        \item \textbf{Acceptance:} Log lưu thành công và hiển thị trong báo cáo.
        \item \textbf{Error handling:} Nếu trùng ID hệ thống từ chối, báo lỗi.
    \end{itemize}
    
    \item \textbf{Cập nhật trạng thái buổi học:}
    \begin{itemize}
        \item \textbf{Input:} Trạng thái (hoàn thành, huỷ, đang diễn ra).
        \item \textbf{Process:} Tutor xác nhận trạng thái, hệ thống lưu lại.
        \item \textbf{Output:} Buổi học hiển thị trạng thái mới.
        \item \textbf{Constraints:} Trạng thái chỉ được thay đổi bởi Tutor của buổi học.
        \item \textbf{Acceptance:} Sinh viên và khoa/bộ môn nhìn thấy trạng thái chính xác.
        \item \textbf{Error handling:} Nếu cập nhật sai thì hệ thống cho phép sửa lại trong 24h.
    \end{itemize}

    \item \textbf{Đăng nội dung bài học:}
    \begin{itemize}
        \item \textbf{Input:} File/tài liệu/note buổi học.
        \item \textbf{Process:} Upload vào hệ thống, lưu trữ trong HCMUT\_LIBRARY.
        \item \textbf{Output:} Sinh viên có thể tải xuống.
        \item \textbf{Constraints:} Dung lượng $\leq$ 50MB/file.
        \item \textbf{Acceptance:} Nội dung hiển thị đúng với sinh viên liên quan.
        \item \textbf{Error handling:} Nếu file hỏng thì báo lỗi, yêu cầu upload lại.
    \end{itemize}
\end{itemize}

%========================================================================================

\subsubsection*{2.1.1.2. Sinh viên}
\addcontentsline{toc}{subsubsection}{2.1.1.2. Sinh viên}
 Sinh viên là đối tượng cần hỗ trợ, tham gia hệ thống để tìm Tutor, đặt lịch học và nhận hỗ trợ học tập.
\begin{itemize}
    \item \textbf{Tạo tài khoản, hồ sơ cá nhân:}
    \begin{itemize}
        \item \textbf{Input:} Họ tên, MSSV, email, số điện thoại, thông tin học tập (GPA, môn cần hỗ trợ).
        \item \textbf{Process:} Hệ thống kiểm tra định dạng dữ liệu, đồng bộ với HCMUT\_DATACORE.
        \item \textbf{Output:} Hồ sơ cá nhân của SV được lưu và hiển thị trong hệ thống.
        \item \textbf{Constraints:} MSSV và email phải trùng khớp dữ liệu HCMUT.
        \item \textbf{Acceptance:} SV có thể đăng nhập và sử dụng các chức năng khác.
        \item \textbf{Error handling:} Nếu dữ liệu không hợp lệ → hệ thống báo lỗi, yêu cầu sửa.
    \end{itemize}

    \item \textbf{Đăng ký chương trình học:}
    \begin{itemize}
        \item \textbf{Input:} Môn học hoặc lĩnh vực cần hỗ trợ, nguyện vọng học tập.
        \item \textbf{Process:} Hệ thống ghi nhận nhu cầu, đồng bộ với dữ liệu đào tạo và gợi ý Tutor phù hợp.
        \item \textbf{Output:} Hồ sơ SV được cập nhật với chương trình đã đăng ký.
        \item \textbf{Constraints:} Chỉ được đăng ký trong danh sách môn/lĩnh vực mà hệ thống hỗ trợ.
        \item \textbf{Acceptance:} SV thấy chương trình học hiển thị trong hồ sơ.
        \item \textbf{Error handling:} Nếu môn/lĩnh vực không hợp lệ thì hệ thống báo lỗi, yêu cầu chọn lại.
    \end{itemize}

    \item \textbf{Lựa chọn Tutor / được ghép tự động:}
    \begin{itemize}
        \item \textbf{Input:} Nhu cầu hỗ trợ (môn, lịch, hình thức).
        \item \textbf{Process:} 
        \begin{itemize}
            \item \textbf{Thủ công:} SV chọn Tutor trong danh sách.
            \item \textbf{Tự động: } Hệ thống so khớp theo khoa/ngành, lịch rảnh, AI ranking.
        \end{itemize}
        \item \textbf{Output:} Ghép cặp Tutor – SV được xác lập.
        \item \textbf{Constraints:} Một SV chỉ có 1 Tutor chính tại một thời điểm.
        \item \textbf{Acceptance:} SV thấy thông tin Tutor trong hồ sơ.
        \item \textbf{Error handling:} Nếu lịch trùng thì yêu cầu chọn lại hoặc hệ thống gợi ý Tutor khác.
    \end{itemize}

    \item \textbf{Đặt lịch học (cảnh báo trùng lịch):}
    \begin{itemize}
        \item \textbf{Input:} Ngày, giờ, môn học.
        \item \textbf{Process:} Hệ thống kiểm tra lịch rảnh của Tutor và lịch của SV.
        \item \textbf{Output:} Lịch học mới được thêm.
        \item \textbf{Constraints:} Không được đặt trùng với lịch học hoặc lịch thi chính thức.
        \item \textbf{Acceptance:} Lịch hiển thị trong tài khoản SV và Tutor.
        \item \textbf{Error handling:} Nếu trùng lịch thì cảnh báo, từ chối đặt.
    \end{itemize}

    \item \textbf{Nhận thông báo và nhắc nhở giờ học:}
    \begin{itemize}
        \item \textbf{Input:} Lịch học sắp diễn ra.
        \item \textbf{Process:} Hệ thống gửi thông báo (noti/email).
        \item \textbf{Output:} SV nhận được thông báo.
        \item \textbf{Constraints:} Thông báo $\geq$ 30 phút trước giờ học.
        \item \textbf{Acceptance:} SV xác nhận thông báo trên hệ thống.
        \item \textbf{Error handling:} Nếu thông báo lỗi thì gửi lại qua email dự phòng.
    \end{itemize}

    \item \textbf{Phản hồi và đánh giá chất lượng buổi học:}
    \begin{itemize}
        \item \textbf{Input:}  Điểm (1–5 sao), bình luận nhận xét.
        \item \textbf{Process:} Hệ thống lưu đánh giá gắn với buổi học và Tutor.
        \item \textbf{Output:} Thông tin phản hồi hiển thị cho Tutor và khoa/bộ môn.
        \item \textbf{Constraints:} Chỉ được đánh giá sau khi buổi học hoàn thành.
        \item \textbf{Acceptance:} Đánh giá hiển thị trong báo cáo tổng hợp.
        \item \textbf{Error handling:} Nếu buổi học chưa hoàn tất → từ chối đánh giá.
    \end{itemize}
\end{itemize}

%========================================================================================

\subsection*{2.1.2. Tác nhân phụ}
\addcontentsline{toc}{subsection}{2.1.2. Tác nhân phụ}

%========================================================================================

\subsubsection*{2.1.2.1. Khoa/Bộ môn}
\addcontentsline{toc}{subsubsection}{2.1.2.1. Khoa/Bộ môn}
\begin{itemize}
    \item \textbf{Nhận đánh giá và tổng hợp kết quả của sinh viên:}
    \begin{itemize}
        \item \textbf{Input:} Đánh giá (điểm số, nhận xét) từ sinh viên sau buổi học.
        \item \textbf{Process:} Hệ thống tổng hợp các phản hồi, phân loại theo môn học/Tutor.
        \item \textbf{Output:} Báo cáo chất lượng buổi học theo lớp, môn, Tutor.
        \item \textbf{Constraints:} Chỉ sử dụng đánh giá từ các buổi học hợp lệ.
        \item \textbf{Acceptance:} Báo cáo được cập nhật định kỳ (theo tuần/tháng).
        \item \textbf{Error handling:} Nếu thiếu dữ liệu đánh giá thì hệ thống ghi chú “chưa có đủ dữ liệu”.
    \end{itemize}
    
    \item \textbf{Quản lý chất lượng Tutor và SV:}
    \begin{itemize}
        \item \textbf{Input:} Hồ sơ Tutor, hồ sơ SV, số buổi học, đánh giá.
        \item \textbf{Process:} Khoa theo dõi, so sánh chất lượng giảng dạy và mức độ tiến bộ của SV.
        \item \textbf{Output:} Bảng xếp hạng/đánh giá Tutor và tổng kết tiến độ SV.
        \item \textbf{Constraints:} Dữ liệu phải dựa trên lịch sử buổi học và đánh giá chính thức.
        \item \textbf{Acceptance:} Báo cáo thể hiện chính xác tình hình giảng dạy – học tập.
        \item \textbf{Error handling:} Nếu dữ liệu không đồng bộ → hệ thống tự động cảnh báo để kiểm tra. 
    \end{itemize}
    
    \item \textbf{Theo dõi tiến độ học tập của sinh viên:}
    \begin{itemize}
        \item \textbf{Input:} GPA trước/sau, kết quả môn học, log buổi học.
        \item \textbf{Process:} Hệ thống đối chiếu tiến độ, xác định sự cải thiện.
        \item \textbf{Output:} Báo cáo cá nhân/tập thể về tiến bộ của SV.
        \item \textbf{Constraints:} Chỉ tính các SV tham gia tối thiểu X buổi học.
        \item \textbf{Acceptance:} Báo cáo có thể dùng làm cơ sở xét khen thưởng hoặc hỗ trợ.
        \item \textbf{Error handling:} Nếu thiếu GPA hoặc dữ liệu học tập → báo cáo đánh dấu “khuyết dữ liệu”.
    \end{itemize}
\end{itemize}

%========================================================================================

\subsubsection*{2.1.2.2. Phòng Công tác sinh viên}
\addcontentsline{toc}{subsubsection}{2.1.2.2. Phòng Công tác sinh viên}
\begin{itemize}
    \item \textbf{Nắm bắt GPA sinh viên trước và sau khi tham gia:}
    \begin{itemize}
        \item \textbf{Input:} GPA ban đầu, GPA cập nhật sau kỳ học.
        \item \textbf{Process:} Hệ thống tự động lấy dữ liệu từ HCMUT\_DATACORE, đối chiếu kết quả trước/sau.
        \item \textbf{Output:} Báo cáo so sánh GPA từng sinh viên.
        \item \textbf{Constraints:} Dữ liệu GPA phải đồng bộ chính xác từ hệ thống đào tạo.
        \item \textbf{Acceptance:} PCTSV có thể tra cứu sự thay đổi kết quả học tập của SV.
        \item \textbf{Error handling:} Nếu thiếu dữ liệu GPA thì hệ thống báo lỗi, yêu cầu đồng bộ lại.
    \end{itemize}
    
    \item \textbf{Tổng hợp kết quả tham gia:}
    \begin{itemize}
        \item \textbf{Input:} Danh sách SV, log số buổi học, đánh giá từ Tutor.
        \item \textbf{Process:} Hệ thống thống kê tần suất tham gia và kết quả học tập.
        \item \textbf{Output:} Báo cáo mức độ tham gia của SV
        \item \textbf{Constraints:} hỉ tính những SV tham gia tối thiểu số buổi học quy định.
        \item \textbf{Acceptance:} Báo cáo được xuất file (Excel/PDF) và tích hợp vào hệ thống quản lý SV.
        \item \textbf{Error handling:} Nếu dữ liệu không đầy đủ thì báo cáo gắn cờ “chưa hoàn chỉnh”.
    \end{itemize}
    
    \item \textbf{Ghi nhận kết quả tham gia để xét điểm rèn luyện / học bổng:}
    \begin{itemize}
        \item \textbf{Input:} Báo cáo tổng hợp SV tham gia chương trình Tutor.
        \item \textbf{Process:} PCTSV đối chiếu với quy chế điểm rèn luyện, học bổng.
        \item \textbf{Output:} Điểm rèn luyện/học bổng của SV được cập nhật.
        \item \textbf{Constraints:} Chỉ SV có tham gia hợp lệ, đủ số buổi quy định mới được ghi nhận.
        \item \textbf{Acceptance:} Kết quả được tích hợp vào hệ thống xét điểm rèn luyện và học bổng.
        \item \textbf{Error handling:} Nếu báo cáo thiếu dữ liệu thì đánh dấu “pending” cho đến khi bổ sung.
    \end{itemize}
\end{itemize}

%========================================================================================

\subsubsection*{2.1.2.3. Phòng Đào tạo}
\addcontentsline{toc}{subsubsection}{2.1.2.3. Phòng Đào tạo}
\begin{itemize}
    \item \textbf{Quản lý và theo dõi hồ sơ Tutor:}
    \begin{itemize}
        \item \textbf{Input:} Hồ sơ cá nhân, chuyên môn, lịch rảnh của Tutor.
        \item \textbf{Process:} PĐT xem, kiểm tra và xác nhận hồ sơ Tutor.
        \item \textbf{Output:} Danh sách Tutor hợp lệ được duyệt.
        \item \textbf{Constraints:} Chỉ Tutor đủ điều kiện (ví dụ GPA $\geq$ 7.0, có chuyên môn rõ ràng) mới được phê duyệt.
        \item \textbf{Acceptance:} Hồ sơ hiển thị trong hệ thống cho SV lựa chọn.
        \item \textbf{Error handling:} Nếu hồ sơ không hợp lệ thì trả lại yêu cầu cập nhật.
    \end{itemize}

    \item \textbf{Theo dõi số lượng buổi học:}
    \begin{itemize}
        \item \textbf{Input:} Log buổi học từ hệ thống.
        \item \textbf{Process:} Hệ thống tổng hợp số buổi học theo Tutor, theo SV, theo môn.
        \item \textbf{Output:} Báo cáo thống kê buổi học (ngày, giờ, trạng thái, số lượng).
        \item \textbf{Constraints:} Chỉ tính các buổi học hợp lệ (có điểm danh).
        \item \textbf{Acceptance:} Báo cáo hiển thị chính xác cho quản lý đào tạo.
        \item \textbf{Error handling:} Nếu dữ liệu log thiếu thì hệ thống cảnh báo “incomplete data”.
    \end{itemize}

    \item \textbf{Tối ưu phân bổ nguồn lực giữa Tutor và SV:}
    \begin{itemize}
        \item \textbf{Input:} Danh sách Tutor, danh sách SV đăng ký, nhu cầu hỗ trợ.
        \item \textbf{Process:} Hệ thống gợi ý phân bổ Tutor cho SV (theo ngành, lịch rảnh, số lượng tối đa).
        \item \textbf{Output:} Bảng phân công Tutor – SV.
        \item \textbf{Constraints:} Một Tutor chỉ nhận tối đa số SV theo quy định (ví dụ $\leq$ 5 SV).
        \item \textbf{Acceptance:} Phân bổ hợp lý, không quá tải Tutor, đáp ứng nhu cầu SV.
        \item \textbf{Error handling:} Nếu số SV vượt quá khả năng phân bổ thì hệ thống cảnh báo, yêu cầu thêm Tutor.
    \end{itemize}
\end{itemize}


%========================================================================================

\section*{2.2. Sơ đồ usecase toàn hệ thống}
\addcontentsline{toc}{section}{2.2. Sơ đồ usecase toàn hệ thống}

%========================================================================================

\section*{2.3. User Stories}
\addcontentsline{toc}{section}{2.3. User Stories}
\subsection*{2.3.1. Module Quản lý Tài khoản và Hồ sơ}
\addcontentsline{toc}{subsection}{2.3.1. Module Quản lý Tài khoản và Hồ sơ} 
\subsubsection*{US-01: Đăng ký tài khoản} 
Là 1 user, tôi muốn đăng ký tài khoản trên hệ thống HCMUT\_SSO bằng số điện thoại và email của mình.
\subsubsection*{US-02: Đăng nhập hệ thống} 
Là 1 user đã có tài khoản, tôi muốn đăng nhập vào hệ thống một cách an toàn bằng email/SĐT và mật khẩu, để truy cập dashboard phù hợp với vai trò của mình.
\subsubsection*{US-03: Cập nhật hồ sơ} 
Là 1 user, tôi muốn chỉnh sửa và cập nhật hồ sơ cá nhân (email, SĐT, chuyên môn, GPA) để thông tin luôn chính xác và mới nhất.

%========================================================================================

\subsection*{2.3.2. Module Đăng ký chương trình học}
\addcontentsline{toc}{subsection}{2.3.2. Module Đăng ký chương trình học}
\subsubsection*{US-04: Đăng ký chương trình học} 
Là một sinh viên, tôi muốn đăng ký môn học phù hợp chuyên ngành và tìm Tutor phù hợp để có thể học tập và luyện thi.
\subsubsection*{US-05: Hủy đăng ký chương trình học} 
Là một sinh viên, tôi muốn hủy môn đã đăng ký chương trình học khi cảm thấy không còn nhu cầu.

%========================================================================================

\subsection*{2.3.3. Module Ghép cặp Tutor – SV}
\addcontentsline{toc}{subsection}{2.3.3. Module Ghép cặp Tutor – SV}
\subsubsection*{US-06: Ghép thủ công (SV chọn Tutor)} 
Là một sinh viên, tôi muốn chọn Tutor từ danh sách đề xuất phù hợp với chuyên ngành và thời gian biểu của mình.

\subsubsection*{US-07: Ghép tự động (hệ thống đề xuất Tutor)} 
Là một sinh viên, tôi muốn hệ thống tự động chọn Tutor dựa trên chuyên môn và thời gian biểu để tiết kiệm thời gian.

%========================================================================================

\subsection*{2.3.4. Module Quản lý lịch học}
\addcontentsline{toc}{subsection}{2.3.4. Module Quản lý lịch học}
\subsubsection*{US-08: Tạo lịch rảnh (Tutor)} 
Là một tutor, tôi muốn tạo lịch rảnh để sinh viên có thể xem, sắp xếp và đăng ký.

\subsubsection*{US-09: Đặt lịch học (SV)} 
Là một sinh viên, tôi muốn đặt lịch học với tutor phù hợp thời gian biểu của bản thân.

\subsubsection*{US-10: Hủy/Đổi lịch học} 
Là một sinh viên, tôi muốn hủy hoặc đổi lịch học để phù hợp với thay đổi của bản thân.

%========================================================================================

\subsection*{2.3.5. Module Thông báo và nhắc nhở}
\addcontentsline{toc}{subsection}{2.3.5. Module Thông báo và nhắc nhở}
\subsubsection*{US-11: Gửi thông báo lịch học} 
Là một user, tôi muốn nhận thông báo khi có lịch học mới được đặt hoặc thay đổi.

\subsubsection*{US-12: Gửi nhắc nhở buổi học} 
Là một user, tôi muốn nhận thông báo nhắc nhở trước buổi học để có thể chuẩn bị.

%========================================================================================

\subsection*{2.3.6. Module Quản lý buổi học và điểm danh}
\addcontentsline{toc}{subsection}{2.3.6. Module Quản lý buổi học và điểm danh}
\subsubsection*{US-13: Điểm danh sinh viên}
Là một tutor, tôi muốn điểm danh sinh viên để nắm rõ số lượng sinh viên tham gia buổi học.

\subsubsection*{US-14: Cập nhật trạng thái buổi học}
Là một tutor, tôi muốn cập nhật trạng thái buổi học để sinh viên có thể nắm rõ tình hình.
%========================================================================================
 
\subsection*{2.3.7. Module Quản lý tài liệu học tập}
\addcontentsline{toc}{subsection}{2.3.7. Module Quản lý tài liệu học tập}
\subsubsection*{US-15: Tutor upload tài liệu}
Là một tutor, tôi muốn đăng tải tài liệu học tập để sinh viên có thể sử dụng trong quá trình học tập.

\subsubsection*{US-16: SV tải tài liệu}
Là một sinh viên, tôi muốn tải xuống tài liệu học tập được tutor chia sẻ để ôn tập.
%========================================================================================

\subsection*{2.3.8. Module Đánh giá và phản hồi}
\addcontentsline{toc}{subsection}{2.3.8. Module Đánh giá và phản hồi}
\subsubsection*{US-17: SV đánh giá Tutor} 
Là một sinh viên, tôi muốn đánh giá tutor sau buổi học để sinh viên khác biết về chất lượng giảng dạy.

\subsubsection*{US-18: Khoa/BM tổng hợp đánh giá} 
Là Khoa/Bộ môn, tôi muốn nắm rõ và tổng hợp thông tin đánh giá của sinh viên về tutor.

%========================================================================================

\subsection*{2.3.9. Module Thống kê và báo cáo}
\addcontentsline{toc}{subsection}{2.3.9. Module Thống kê và báo cáo}
\subsubsection*{US-19: Báo cáo kết quả học tập SV} 
Là Khoa, tôi muốn xem báo cáo kết quả học tập và tình trạng tham gia tiết học của sinh viên theo từng môn học để nắm tình hình học tập, kịp thời hỗ trợ khi cần thiết.

\subsubsection*{US-20: Báo cáo chất lượng Tutor} 
Là Khoa, tôi muốn tạo báo cáo tổng hợp của Tutor để đánh giá.

\subsubsection*{US-21: Báo cáo tổng hợp (Khoa, PCTSV, PĐT)} 
Là Khoa, tôi muốn xem báo cáo tổng hợp của sinh viên để cộng điểm rèn luyện hoặc xét học bổng chính xác, công bằng.

\subsubsection*{US-25: SV liên hệ Tutor khi cần hỗ trợ} 
Là một sinh viên, tôi muốn liên hệ với Tutor khi cần hỗ trợ để giải đáp thắc mắc hoặc nhận hướng dẫn thêm.
%========================================================================================

\section*{2.4. Yêu cầu chức năng}
\addcontentsline{toc}{section}{2.4. Yêu cầu chức năng}
\subsection*{2.4.1. Module Quản lý Tài khoản và Hồ sơ}
\addcontentsline{toc}{subsection}{2.4.1. Module Quản lý Tài khoản và Hồ sơ}
\newpage
\subsubsection*{Use Case 01: Đăng ký tài khoản}
\begin{samepage}
\begin{table}[h!]
\begin{tabular}{|l|lll|l}
\cline{1-4}
\textbf{ID}                              & \multicolumn{3}{l|}{UC-01}                                                                                                                                                                                                                &  \\ \cline{1-4}
\textbf{Tên}                             & \multicolumn{3}{l|}{Đăng ký tài khoản}                                                                                                                                                                                                    &  \\ \cline{1-4}
\textbf{Mô tả}                           & \multicolumn{3}{l|}{Người dùng (SV hoặc Tutor) đăng ký tài khoản mới để tham gia hệ thống.}                                                                                                                                               &  \\ \cline{1-4}
\textbf{Actor chính}                     & \multicolumn{3}{l|}{Sinh viên, Tutor}                                                                                                                                                                                                     &  \\ \cline{1-4}
\textbf{Actor phụ}                       & \multicolumn{3}{l|}{Hệ thống xác thực OTP (Email/SMS), Admin PĐT}                                                                                                                                                                         &  \\ \cline{1-4}
\textbf{Tiền điều kiện}                  & \multicolumn{3}{l|}{\begin{tabular}[c]{@{}l@{}}Người dùng chưa có tài khoản\\ Có email hoặc số điện thoại hợp lệ\end{tabular}}                                                                                                            &  \\ \cline{1-4}
\textbf{Hậu điều kiện}                   & \multicolumn{3}{l|}{\begin{tabular}[c]{@{}l@{}}Tài khoản hợp lệ được tạo, có ID duy nhất.\\ Hồ sơ người dùng được lưu trong cơ sở dữ liệu\end{tabular}}                                                                                   &  \\ \cline{1-4}
\multirow{10}{*}{\textbf{Luồng sự kiện}} & \multicolumn{1}{l|}{\textbf{Bước}}      & \multicolumn{1}{l|}{\textbf{Thực hiện bởi}}     & \textbf{Mô tả}                                                                                                                                &  \\ \cline{2-4}
                                         & \multicolumn{1}{l|}{1}                  & \multicolumn{1}{l|}{Người dùng}                 & Chọn chức năng “Đăng ký”.                                                                                                                     &  \\ \cline{2-4}
                                         & \multicolumn{1}{l|}{2}                  & \multicolumn{1}{l|}{Người dùng}                 & \begin{tabular}[c]{@{}l@{}}Nhập thông tin cá nhân (SV: MSSV, GPA, khoa; \\ Tutor: chuyên môn, GPA $\geq$ 3.0 hoặc giấy xác minh).\end{tabular}     &  \\ \cline{2-4}
                                         & \multicolumn{1}{l|}{3}                  & \multicolumn{1}{l|}{Hệ thống}                   & Kiểm tra định dạng dữ liệu.                                                                                                                   &  \\ \cline{2-4}
                                         & \multicolumn{1}{l|}{4}                  & \multicolumn{1}{l|}{Hệ thống}                   & Gửi OTP xác thực qua email/SMS.                                                                                                               &  \\ \cline{2-4}
                                         & \multicolumn{1}{l|}{5}                  & \multicolumn{1}{l|}{Người dùng}                 & Nhập OTP nhận được.                                                                                                                           &  \\ \cline{2-4}
                                         & \multicolumn{1}{l|}{6}                  & \multicolumn{1}{l|}{Hệ thống}                   & Kiểm tra OTP.                                                                                                                                 &  \\ \cline{2-4}
                                         & \multicolumn{1}{l|}{7}                  & \multicolumn{1}{l|}{Hệ thống}                   & Kiểm tra trùng MSSV/email.                                                                                                                    &  \\ \cline{2-4}
                                         & \multicolumn{1}{l|}{8}                  & \multicolumn{1}{l|}{Hệ thống}                   & Lưu dữ liệu hợp lệ, gán UserID duy nhất.                                                                                                      &  \\ \cline{2-4}
                                         & \multicolumn{1}{l|}{9}                  & \multicolumn{1}{l|}{Hệ thống}                   & Hiển thị thông báo “Đăng ký thành công”                                                                                                       &  \\ \cline{1-4}
\multirow{4}{*}{\textbf{Luồng thay thế}} & \multicolumn{1}{l|}{\textbf{Bước}}      & \multicolumn{1}{l|}{\textbf{Thực hiện bởi}}     & \textbf{Mô tả}                                                                                                                                &  \\ \cline{2-4}
                                         & \multicolumn{1}{l|}{3a}                 & \multicolumn{1}{l|}{Người dùng}                 & \begin{tabular}[c]{@{}l@{}}Nhập email không hợp lệ → Hệ thống báo lỗi, \\ yêu cầu nhập lại (quay về bước 2).\end{tabular}                     &  \\ \cline{2-4}
                                         & \multicolumn{1}{l|}{5a}                 & \multicolumn{1}{l|}{Người dùng}                 & \begin{tabular}[c]{@{}l@{}}Người dùng nhập OTP sai → Hệ thống báo lỗi, \\ cho nhập lại tối đa 3 lần.\end{tabular}                             &  \\ \cline{2-4}
                                         & \multicolumn{1}{l|}{7a}                 & \multicolumn{1}{l|}{Hệ thống}                   & \begin{tabular}[c]{@{}l@{}}Phát hiện MSSV/email đã tồn tại → hiển thị \\ “Tài khoản đã có”.\end{tabular}                                      &  \\ \cline{1-4}
\textbf{Ngoại lệ}                        & \multicolumn{3}{l|}{\begin{tabular}[c]{@{}l@{}}Hệ thống không gửi được OTP (lỗi server/email/SMS) \\ → hiển thị “Vui lòng thử lại sau”. \\ Hệ thống lỗi khi lưu dữ liệu vào DB \\ → rollback, hiển thị “Thao tác thất bại”.\end{tabular}} &  \\ \cline{1-4}
\textbf{Business Rules}                  & \multicolumn{3}{l|}{\begin{tabular}[c]{@{}l@{}}Một MSSV/email chỉ được dùng cho 1 tài khoản. \\ Tutor phải được xác minh bởi PĐT trước khi tài khoản kích hoạt.\end{tabular}}                                                             &  \\ \cline{1-4}
\textbf{Data requirement}                & \multicolumn{3}{l|}{\begin{tabular}[c]{@{}l@{}}Users(userID, role, email, password, MSSV, GPA, faculty)\\ TutorProfile(tutorID, chuyên môn, giấy xác minh)\end{tabular}}                                                                  &  \\ \cline{1-4}
\end{tabular}
\caption{Bảng đặc tả chức năng đăng ký tài khoản}
\end{table}
\end{samepage}

%========================================================================================

\newpage
\subsubsection*{Use Case 02: Đăng nhập}
\begin{samepage}
\begin{table}[h!]
\begin{tabular}{|l|lll|l}
\cline{1-4}
\textbf{ID}                              & \multicolumn{3}{l|}{UC-02}                                                                                                                                                                                    &  \\ \cline{1-4}
\textbf{Tên}                             & \multicolumn{3}{l|}{Đăng nhập}                                                                                                                                                                                &  \\ \cline{1-4}
\textbf{Mô tả}                           & \multicolumn{3}{l|}{Người dùng đăng nhập để truy cập hệ thống bằng tài khoản đã đăng ký.}                                                                                                                     &  \\ \cline{1-4}
\textbf{Actor chính}                     & \multicolumn{3}{l|}{SV, Tutor, Admin}                                                                                                                                                                         &  \\ \cline{1-4}
\textbf{Actor phụ}                       & \multicolumn{3}{l|}{Hệ thống xác thực đăng nhập}                                                                                                                                                              &  \\ \cline{1-4}
\textbf{Tiền điều kiện}                  & \multicolumn{3}{l|}{Người dùng đã có tài khoản hợp lệ.}                                                                                                                                                       &  \\ \cline{1-4}
\textbf{Hậu điều kiện}                   & \multicolumn{3}{l|}{Người dùng truy cập được dashboard theo đúng quyền}                                                                                                                                       &  \\ \cline{1-4}
\multirow{6}{*}{\textbf{Luồng sự kiện}}  & \multicolumn{1}{l|}{\textbf{Bước}} & \multicolumn{1}{l|}{\textbf{Thực hiện bởi}} & \textbf{Mô tả}                                                                                                             &  \\ \cline{2-4}
                                         & \multicolumn{1}{l|}{1}             & \multicolumn{1}{l|}{Người dùng}             & Mở giao diện đăng nhập                                                                                                     &  \\ \cline{2-4}
                                         & \multicolumn{1}{l|}{2}             & \multicolumn{1}{l|}{Người dùng}             & Nhập email/MSSV và mật khẩu.                                                                                               &  \\ \cline{2-4}
                                         & \multicolumn{1}{l|}{3}             & \multicolumn{1}{l|}{Hệ thống}               & Kiểm tra định dạng dữ liệu đầu vào                                                                                         &  \\ \cline{2-4}
                                         & \multicolumn{1}{l|}{4}             & \multicolumn{1}{l|}{Hệ thống}               & Kiểm tra thông tin đăng nhập trong DB.                                                                                     &  \\ \cline{2-4}
                                         & \multicolumn{1}{l|}{5}             & \multicolumn{1}{l|}{Hệ thống}               & \begin{tabular}[c]{@{}l@{}}Xác nhận thông tin đúng → mở phiên đăng \\ nhập và chuyển đến giao diện chính.\end{tabular}     &  \\ \cline{1-4}
\multirow{3}{*}{\textbf{Luồng thay thế}} & \multicolumn{1}{l|}{\textbf{Bước}} & \multicolumn{1}{l|}{\textbf{Thực hiện bởi}} & \textbf{Mô tả}                                                                                                             &  \\ \cline{2-4}
                                         & \multicolumn{1}{l|}{2a}            & \multicolumn{1}{l|}{Người dùng}             & \begin{tabular}[c]{@{}l@{}}Nhập sai mật khẩu → Hệ thống báo lỗi \\ “Sai mật khẩu”, cho nhập lại tối đa 5 lần.\end{tabular} &  \\ \cline{2-4}
                                         & \multicolumn{1}{l|}{4a}            & \multicolumn{1}{l|}{Hệ thống}               & \begin{tabular}[c]{@{}l@{}}Phát hiện tài khoản bị khóa → hiển thị \\ “Tài khoản bị khóa, liên hệ Admin”..\end{tabular}     &  \\ \cline{1-4}
\textbf{Ngoại lệ}                        & \multicolumn{3}{l|}{\begin{tabular}[c]{@{}l@{}}Gặp lỗi khi truy vấn DB → hiển thị “Không thể đăng nhập lúc này, vui lòng \\ thử lại sau”.\end{tabular}}                                                       &  \\ \cline{1-4}
\textbf{Business Rules}                  & \multicolumn{3}{l|}{\begin{tabular}[c]{@{}l@{}}Nếu nhập sai mật khẩu quá 5 lần → Hệ thống tự động khóa tài khoản trong \\ 30 phút. Admin có quyền thiết lập lại mật khẩu cho người dùng.\end{tabular}}        &  \\ \cline{1-4}
\textbf{Data requirement}                & \multicolumn{3}{l|}{Users(userID, email, password, role, status, lastLogin)}                                                                                                                                  &  \\ \cline{1-4}
\end{tabular}
\caption{Bảng đặc tả chức năng đăng nhập}
\end{table}
\end{samepage}



%========================================================================================

\newpage
\subsubsection*{Use Case 03: Cập nhật hồ sơ}
\begin{samepage}
\begin{table}[h!]
\begin{tabular}{|l|lll|l}
\cline{1-4}
\textbf{ID}                              & \multicolumn{3}{l|}{UC-03}                                                                                                                                                                                &  \\ \cline{1-4}
\textbf{Tên}                             & \multicolumn{3}{l|}{Cập nhật hồ sơ}                                                                                                                                                                       &  \\ \cline{1-4}
\textbf{Mô tả}                           & \multicolumn{3}{l|}{Người dùng cập nhật thông tin hồ sơ cá nhân để đảm bảo dữ liệu mới nhất.}                                                                                                             &  \\ \cline{1-4}
\textbf{Actor chính}                     & \multicolumn{3}{l|}{SV, Tutor}                                                                                                                                                                            &  \\ \cline{1-4}
\textbf{Actor phụ}                       & \multicolumn{3}{l|}{Hệ thống, Admin (có quyền xem/sửa/kiểm tra Tutor profile) (PĐT, Khoa)}                                                                                                                &  \\ \cline{1-4}
\textbf{Tiền điều kiện}                  & \multicolumn{3}{l|}{Người dùng đã đăng nhập hệ thống.}                                                                                                                                                    &  \\ \cline{1-4}
\textbf{Hậu điều kiện}                   & \multicolumn{3}{l|}{Hồ sơ cập nhật thành công trong cơ sở dữ liệu.}                                                                                                                                       &  \\ \cline{1-4}
\multirow{6}{*}{\textbf{Luồng sự kiện}}  & \multicolumn{1}{l|}{\textbf{Bước}} & \multicolumn{1}{l|}{\textbf{Thực hiện bởi}} & \textbf{Mô tả}                                                                                                         &  \\ \cline{2-4}
                                         & \multicolumn{1}{l|}{1}             & \multicolumn{1}{l|}{Người dùng}             & Đăng nhập, chọn “Cập nhật hồ sơ”.                                                                                      &  \\ \cline{2-4}
                                         & \multicolumn{1}{l|}{2}             & \multicolumn{1}{l|}{Người dùng}             & \begin{tabular}[c]{@{}l@{}}Chỉnh sửa thông tin (SĐT, email, chuyên ngành, \\ mô tả năng lực…).\end{tabular}            &  \\ \cline{2-4}
                                         & \multicolumn{1}{l|}{3}             & \multicolumn{1}{l|}{Hệ thống}               & Kiểm tra định dạng dữ liệu.                                                                                            &  \\ \cline{2-4}
                                         & \multicolumn{1}{l|}{4}             & \multicolumn{1}{l|}{Hệ thống}               & Lưu thông tin mới vào DB.                                                                                              &  \\ \cline{2-4}
                                         & \multicolumn{1}{l|}{5}             & \multicolumn{1}{l|}{Hệ thống}               & Hiển thị thông báo “Cập nhật thành công”.                                                                              &  \\ \cline{1-4}
\multirow{3}{*}{\textbf{Luồng thay thế}} & \multicolumn{1}{l|}{\textbf{Bước}} & \multicolumn{1}{l|}{\textbf{Thực hiện bởi}} & \textbf{Mô tả}                                                                                                         &  \\ \cline{2-4}
                                         & \multicolumn{1}{l|}{2a}            & \multicolumn{1}{l|}{Người dùng}             & \begin{tabular}[c]{@{}l@{}}Bỏ trống trường bắt buộc (email, SĐT) \\ → Hệ thống báo lỗi, yêu cầu nhập lại.\end{tabular} &  \\ \cline{2-4}
                                         & \multicolumn{1}{l|}{3a}            & \multicolumn{1}{l|}{Người dùng}             & \begin{tabular}[c]{@{}l@{}}Nhập email/SĐT sai định dạng \\ → Hệ thống báo lỗi, yêu cầu nhập lại.\end{tabular}          &  \\ \cline{1-4}
\textbf{Ngoại lệ}                        & \multicolumn{3}{l|}{Hệ thống lỗi khi lưu dữ liệu vào DB → hiển thị “Cập nhật thất bại”.}                                                                                                                  &  \\ \cline{1-4}
\textbf{Business Rules}                  & \multicolumn{3}{l|}{\begin{tabular}[c]{@{}l@{}}Tutor phải cập nhật thông tin học thuật theo mẫu do PĐT quy định.\\ SV không được sửa MSSV.\end{tabular}}                                                  &  \\ \cline{1-4}
\textbf{Data requirement}                & \multicolumn{3}{l|}{\begin{tabular}[c]{@{}l@{}}Users(userID, email, phone, faculty, GPA, updatedAt)\\ TutorProfile(tutorID, chuyên môn, kinh nghiệm, updatedAt)\end{tabular}}                             &  \\ \cline{1-4}
\end{tabular}
\caption{Bảng đặc tả chức năng cập nhật hồ sơ}
\end{table}
\end{samepage}

%========================================================================================

\subsection*{2.4.2. Module Đăng ký chương trình học}
\addcontentsline{toc}{subsection}{2.4.2. Module Đăng ký chương trình học}

%========================================================================================

\subsection*{2.4.3. Module Ghép cặp Tutor – SV}
\addcontentsline{toc}{subsection}{2.4.3. Module Ghép cặp Tutor – SV}

%========================================================================================

\subsection*{2.4.4. Module Quản lý lịch học}
\addcontentsline{toc}{subsection}{2.4.4. Module Quản lý lịch học}

%========================================================================================

\subsection*{2.4.5. Module Thông báo và nhắc nhở}
\addcontentsline{toc}{subsection}{2.4.5. Module Thông báo và nhắc nhở}

%========================================================================================

\subsection*{2.4.6. Module Quản lý buổi học và điểm danh}
\addcontentsline{toc}{subsection}{2.4.6. Module Quản lý buổi học và điểm danh}

%========================================================================================

\subsection*{2.4.7. Module Quản lý tài liệu học tập}
\addcontentsline{toc}{subsection}{2.4.7. Module Quản lý tài liệu học tập}

%========================================================================================

\subsection*{2.4.8. Module Đánh giá và phản hồi}
\addcontentsline{toc}{subsection}{2.4.8. Module Đánh giá và phản hồi}

%========================================================================================

\subsection*{2.4.9. Module Thống kê và báo cáo}
\addcontentsline{toc}{subsection}{2.4.9. Module Thống kê và báo cáo}

%========================================================================================

\subsection*{2.4.10. Module Chương trình học thuật và phi học thuật}
\addcontentsline{toc}{subsection}{2.4.10. Module Chương trình học thuật và phi học thuật}

%========================================================================================

\section*{2.5. Yêu cầu phi chức năng}
\addcontentsline{toc}{section}{2.5. Yêu cầu phi chức năng}
Để xây dựng một hệ thống kết nối Tutor và Sinh viên thực sự hiệu quả và đáng tin cậy, việc đáp ứng các yêu cầu về chức năng là chưa đủ. Yếu tố quyết định trải nghiệm người dùng và sự thành công lâu dài của dự án nằm ở các Yêu cầu phi chức năng (Non-Functional Requirements). Các tiêu chí này đặt ra những chuẩn mực về tốc độ, bảo mật, độ ổn định và tính dễ sử dụng của hệ thống. Những ràng buộc và tiêu chuẩn dưới đây sẽ là những yếu tố để đảm bảo hệ thống không chỉ hoàn thiện mà còn mang lại sự hài lòng và tin tưởng tuyệt đối cho mọi người dùng, từ sinh viên, Tutor đến các cấp quản lý.

\subsection*{2.5.1. Hiệu năng (Performance Requirements)}
\addcontentsline{toc}{subsection}{2.5.1. Hiệu năng (Performance Requirements)}
\begin{itemize}
    \item \textbf{Mô tả:} Hệ thống phải xử lý nhanh và ổn định cho nhiều người dùng đồng thời.
    \item \textbf{Constraints:}
    \begin{itemize}
        \item Hỗ trợ tối thiểu 500 người dùng đồng thời.
        \item Thời gian phản hồi cho thao tác chính $\leq$ 3 giây.
        \item Thuật toán ghép cặp Tutor–SV chạy trong $\leq$ 5 giây.
    \end{itemize}
    \item \textbf{Acceptance:} Kiểm thử tải (load test) cho thấy hệ thống đáp ứng $\geq$ 95\% request trong 3 giây
\end{itemize}

%========================================================================================

\subsection*{2.5.2. Bảo mật (Security Requirements)}
\addcontentsline{toc}{subsection}{2.5.2. Hiệu năng (Performance Requirements)}
\begin{itemize}
    \item \textbf{Mô tả:} Bảo vệ thông tin người dùng và dữ liệu hệ thống khỏi truy cập trái phép.
    \item \textbf{Constraints:}
    \begin{itemize}
        \item Mã hóa toàn bộ giao tiếp bằng HTTPS (TLS 1.3).
        \item Lưu mật khẩu bằng bcrypt/Argon2.
        \item Xác thực 2FA áp dụng cho Tutor và Admin. 
        \item Khóa tài khoản sau 5 lần nhập sai mật khẩu.
        \item Phân quyền theo role (SV, Tutor, Khoa, PCTSV, PĐT, Admin).
    \end{itemize}
    \item \textbf{Acceptance:} Thử nghiệm penetration test không phát hiện lỗ hổng nghiêm trọng.
\end{itemize}

%========================================================================================

\subsection*{2.5.3. Tính tin cậy \& sẵn sàng (Reliability \& Availability)}
\addcontentsline{toc}{subsection}{2.5.3. Tính tin cậy \& sẵn sàng (Reliability \& Availability)}
\begin{itemize}
    \item \textbf{Mô tả:} Hệ thống phải đảm bảo tính liên tục và phục hồi khi có sự cố.
    \item \textbf{Constraints:} 
    \begin{itemize}
        \item Thời gian uptime $\geq$ 99.5\%/tháng.
        \item Backup dữ liệu hàng ngày, phục hồi $\leq$ 2h.
        \item Retry khi gửi thông báo thất bại.
        \item Log toàn bộ giao dịch quan trọng.
    \end{itemize}
    \item \textbf{Acceptance:} DRP (Disaster Recovery Plan) kiểm thử thành công, phục hồi dữ liệu $\leq$ 2h.
\end{itemize}

%========================================================================================

\subsection*{2.5.4. Khả năng sử dụng (Usability)}
\addcontentsline{toc}{subsection}{2.5.4. Khả năng sử dụng (Usability)}
\begin{itemize}
    \item \textbf{Mô tả:} Giao diện thân thiện, dễ sử dụng cho tất cả loại người dùng.
    \item \textbf{Constraints:} 
    \begin{itemize}
        \item Hỗ trợ đa thiết bị (desktop, mobile, tablet).
        \item Ngôn ngữ: Tiếng Việt (mặc định), Tiếng Anh (tùy chọn).
        \item Người dùng mới có thể đăng ký, đặt lịch trong $\leq$ 5 phút.
        \item Có màn hình trợ giúp/hướng dẫn.
    \end{itemize}
    \item \textbf{Acceptance:} Khảo sát $\geq$ 80\% người dùng đánh giá giao diện “dễ sử dụng”.

\end{itemize}

%========================================================================================

\subsection*{2.5.5. Tính bảo trì \& mở rộng (Maintainability \& Extensibility)}
\addcontentsline{toc}{subsection}{2.5.5. Tính bảo trì \& mở rộng (Maintainability \& Extensibility)}
\begin{itemize}
    \item \textbf{Mô tả:} Hệ thống dễ bảo trì, nâng cấp mà không ảnh hưởng đến chức năng hiện có.
    \item \textbf{Constraints:}
    \begin{itemize}
        \item Tuân thủ mô hình MVC hoặc Microservices.
        \item Code phải có comment, tuân thủ coding convention.
        \item Thêm module mới không ảnh hưởng module cũ.
        \item Bug critical sau release $\leq$ 2\%.
    \end{itemize}
    \item \textbf{Acceptance:} Regression test cho thấy chức năng cũ không bị ảnh hưởng sau khi thêm module mới.
\end{itemize}

%========================================================================================

\subsection*{2.5.6. Khả năng tương thích (Compatibility)}
\addcontentsline{toc}{subsection}{2.5.6. Khả năng tương thích (Compatibility)}
\begin{itemize}
    \item \textbf{Mô tả:} Hệ thống chạy được trên nhiều nền tảng và dịch vụ tích hợp.
    \item \textbf{Constraints:}
    \begin{itemize}
        \item Web chạy trên Chrome, Firefox, Edge, Safari (phiên bản mới nhất).
        \item Mobile app chạy trên Android $\geq$ 10, iOS $\geq$ 13.
        \item Tích hợp email server và SMS gateway.
    \end{itemize}
    \item \textbf{Acceptance:} Test cross-browser cho kết quả hiển thị đúng $\geq$ 95\%.
\end{itemize}

%========================================================================================

\subsection*{2.5.7. Ràng buộc kỹ thuật (Technical Constraints)}
\addcontentsline{toc}{subsection}{2.5.7. Ràng buộc kỹ thuật (Technical Constraints)}
\begin{itemize}
    \item \textbf{Mô tả:} Các công nghệ, công cụ và nền tảng bắt buộc sử dụng.
    \item \textbf{Constraints:}
    \begin{itemize}
        \item DB: MySQL hoặc PostgreSQL.
        \item Backend: Java Spring Boot hoặc Node.js.
        \item Frontend: ReactJS hoặc Angular.
        \item Triển khai trên Docker/Kubernetes.
    \end{itemize}
    \item \textbf{Acceptance:} Cấu hình hệ thống triển khai thành công trên môi trường staging/production.
\end{itemize}

%========================================================================================