\part*{2. Phân tích yêu cầu dự án}
\addcontentsline{toc}{part}{2. Phân tích yêu cầu dự án}
Trong chương này, nhóm em sẽ tiến hành xác định và phân tích các yêu cầu của hệ thống. Trước hết, việc nhận diện các tác nhân (Actors) tương tác với hệ thống là bước quan trọng để hiểu rõ các luồng chức năng và mục tiêu mà phần mềm cần đáp ứng.

%========================================================================================

\section*{2.1. Tác nhân}
\addcontentsline{toc}{section}{2.1. Tác nhân}
Tác nhân (Actor) là một thực thể bên ngoài, có thể là người dùng hoặc một hệ thống khác, tương tác trực tiếp với hệ thống để thực hiện một mục tiêu cụ thể. Dựa trên vai trò và cách thức tương tác, các tác nhân của "Hệ thống hỗ trợ Tutor" được phân loại thành tác nhân chính và tác nhân phụ.

%========================================================================================

\subsection*{2.1.1. Tác nhân chính}
\addcontentsline{toc}{subsection}{2.1.1. Tác nhân chính}
\subsubsection*{2.1.1.1. Tutor}
\addcontentsline{toc}{subsubsection}{2.1.1.1. Tutor}
Tutor có thể là sinh viên giỏi, NCS, hoặc giảng viên, được đăng ký vào hệ thống để hỗ trợ học tập. Họ có trách nhiệm quản lý hồ sơ, lịch rảnh, tổ chức buổi học và theo dõi tiến độ của sinh viên.
\begin{itemize}
    \item \textbf{Tạo và cập nhật lịch rảnh:}
    \begin{itemize}
        \item \textbf{Input:} Ngày, giờ, hình thức học (online/offline).
        \item \textbf{Process:} Hệ thống lưu lại khung giờ rảnh, đồng bộ với module đặt lịch.
        \item \textbf{Output:} Lịch rảnh hiển thị cho sinh viên để chọn.
        \item \textbf{Constraints:} Không được nhập trùng khung giờ.
        \item \textbf{Acceptance:} Sinh viên có thể đặt lịch học trong khung rảnh của Tutor.
        \item \textbf{Error handling:} Nếu nhập sai định dạng hoặc trùng lịch thì hệ thống báo lỗi.
    \end{itemize}
    
    \item \textbf{Mở (oneline/offline), hủy, đổi buổi học:}
    \begin{itemize}
        \item \textbf{Input:} Yêu cầu mở buổi học, hoặc yêu cầu huỷ/đổi.
        \item \textbf{Process:} Hệ thống kiểm tra lịch đã có sinh viên đặt chưa, xử lý cập nhật.
        \item \textbf{Output:} Buổi học được thêm/sửa/xoá trong hệ thống.
        \item \textbf{Constraints:} Hủy/đổi $\geq$ 3h trước giờ học
        \item \textbf{Acceptance:} Sinh viên và Tutor đều nhận được thông báo cập nhật.
        \item \textbf{Error handling:} Nếu yêu cầu đổi sát giờ hệ thống từ chối, báo lỗi.
    \end{itemize}
    
    \item \textbf{Nhận thông báo và nhắc nhở giờ dạy:}
    \begin{itemize}
        \item \textbf{Input:} Lịch học sắp diễn ra.
        \item \textbf{Process:} Hệ thống gửi thông báo (noti/email).
        \item \textbf{Output:} Tutor nhận được thông báo đúng hạn.
        \item \textbf{Constraints:} Thông báo phải được gửi $\geq$ 30 phút trước giờ học.
        \item \textbf{Acceptance:} Tutor xác nhận đã đọc thông báo.
        \item \textbf{Error handling:} Nếu gửi lỗi → hệ thống gửi lại lần 2 hoặc báo qua email dự phòng.
    \end{itemize}
    
    \item \textbf{Theo dõi tiến bộ sinh viên:}
    \begin{itemize}
        \item \textbf{Input:} Điểm, nhận xét, đánh giá sau buổi học.
        \item \textbf{Process:} Tutor nhập vào form đánh giá, hệ thống lưu lại.
        \item \textbf{Output:} Báo cáo tiến bộ gắn với hồ sơ sinh viên.
        \item \textbf{Constraints:} Chỉ Tutor đã dạy sinh viên đó mới được nhập.
        \item \textbf{Acceptance:} Khoa/bộ môn có thể truy cập báo cáo.
        \item \textbf{Error handling:} Nếu buổi học chưa hoàn tất thì từ chối ghi nhận.
    \end{itemize}
    
    \item \textbf{Điểm danh và record:}
    \begin{itemize}
        \item \textbf{Input:} ID sinh viên tham gia, mã buổi học.
        \item \textbf{Process:} Hệ thống điểm danh, ghi log tham dự.
        \item \textbf{Output:} Record buổi học (thời lượng, người tham gia).
        \item \textbf{Constraints:} Mỗi SV chỉ được điểm danh vào 1 buổi học tại 1 thời điểm.
        \item \textbf{Acceptance:} Log lưu thành công và hiển thị trong báo cáo.
        \item \textbf{Error handling:} Nếu trùng ID hệ thống từ chối, báo lỗi.
    \end{itemize}
    
    \item \textbf{Cập nhật trạng thái buổi học:}
    \begin{itemize}
        \item \textbf{Input:} Trạng thái (hoàn thành, huỷ, đang diễn ra).
        \item \textbf{Process:} Tutor xác nhận trạng thái, hệ thống lưu lại.
        \item \textbf{Output:} Buổi học hiển thị trạng thái mới.
        \item \textbf{Constraints:} Trạng thái chỉ được thay đổi bởi Tutor của buổi học.
        \item \textbf{Acceptance:} Sinh viên và khoa/bộ môn nhìn thấy trạng thái chính xác.
        \item \textbf{Error handling:} Nếu cập nhật sai thì hệ thống cho phép sửa lại trong 24h.
    \end{itemize}

    \item \textbf{Đăng nội dung bài học:}
    \begin{itemize}
        \item \textbf{Input:} File/tài liệu/note buổi học.
        \item \textbf{Process:} Upload vào hệ thống, lưu trữ trong HCMUT\_LIBRARY.
        \item \textbf{Output:} Sinh viên có thể tải xuống.
        \item \textbf{Constraints:} Dung lượng $\leq$ 50MB/file.
        \item \textbf{Acceptance:} Nội dung hiển thị đúng với sinh viên liên quan.
        \item \textbf{Error handling:} Nếu file hỏng thì báo lỗi, yêu cầu upload lại.
    \end{itemize}
\end{itemize}

%========================================================================================

\subsubsection*{2.1.1.2. Sinh viên}
\addcontentsline{toc}{subsubsection}{2.1.1.2. Sinh viên}
 Sinh viên là đối tượng cần hỗ trợ, tham gia hệ thống để tìm Tutor, đặt lịch học và nhận hỗ trợ học tập.
\begin{itemize}
    \item \textbf{Tạo tài khoản, hồ sơ cá nhân:}
    \begin{itemize}
        \item \textbf{Input:} Họ tên, MSSV, email, số điện thoại, thông tin học tập (GPA, môn cần hỗ trợ).
        \item \textbf{Process:} Hệ thống kiểm tra định dạng dữ liệu, đồng bộ với HCMUT\_DATACORE.
        \item \textbf{Output:} Hồ sơ cá nhân của SV được lưu và hiển thị trong hệ thống.
        \item \textbf{Constraints:} MSSV và email phải trùng khớp dữ liệu HCMUT.
        \item \textbf{Acceptance:} SV có thể đăng nhập và sử dụng các chức năng khác.
        \item \textbf{Error handling:} Nếu dữ liệu không hợp lệ → hệ thống báo lỗi, yêu cầu sửa.
    \end{itemize}

    \item \textbf{Đăng ký chương trình học:}
    \begin{itemize}
        \item \textbf{Input:} Môn học hoặc lĩnh vực cần hỗ trợ, nguyện vọng học tập.
        \item \textbf{Process:} Hệ thống ghi nhận nhu cầu, đồng bộ với dữ liệu đào tạo và gợi ý Tutor phù hợp.
        \item \textbf{Output:} Hồ sơ SV được cập nhật với chương trình đã đăng ký.
        \item \textbf{Constraints:} Chỉ được đăng ký trong danh sách môn/lĩnh vực mà hệ thống hỗ trợ.
        \item \textbf{Acceptance:} SV thấy chương trình học hiển thị trong hồ sơ.
        \item \textbf{Error handling:} Nếu môn/lĩnh vực không hợp lệ thì hệ thống báo lỗi, yêu cầu chọn lại.
    \end{itemize}

    \item \textbf{Lựa chọn Tutor / được ghép tự động:}
    \begin{itemize}
        \item \textbf{Input:} Nhu cầu hỗ trợ (môn, lịch, hình thức).
        \item \textbf{Process:} 
        \begin{itemize}
            \item \textbf{Thủ công:} SV chọn Tutor trong danh sách.
            \item \textbf{Tự động: } Hệ thống so khớp theo khoa/ngành, lịch rảnh, AI ranking.
        \end{itemize}
        \item \textbf{Output:} Ghép cặp Tutor – SV được xác lập.
        \item \textbf{Constraints:} Một SV chỉ có 1 Tutor chính tại một thời điểm.
        \item \textbf{Acceptance:} SV thấy thông tin Tutor trong hồ sơ.
        \item \textbf{Error handling:} Nếu lịch trùng thì yêu cầu chọn lại hoặc hệ thống gợi ý Tutor khác.
    \end{itemize}

    \item \textbf{Đặt lịch học (cảnh báo trùng lịch):}
    \begin{itemize}
        \item \textbf{Input:} Ngày, giờ, môn học.
        \item \textbf{Process:} Hệ thống kiểm tra lịch rảnh của Tutor và lịch của SV.
        \item \textbf{Output:} Lịch học mới được thêm.
        \item \textbf{Constraints:} Không được đặt trùng với lịch học hoặc lịch thi chính thức.
        \item \textbf{Acceptance:} Lịch hiển thị trong tài khoản SV và Tutor.
        \item \textbf{Error handling:} Nếu trùng lịch thì cảnh báo, từ chối đặt.
    \end{itemize}

    \item \textbf{Nhận thông báo và nhắc nhở giờ học:}
    \begin{itemize}
        \item \textbf{Input:} Lịch học sắp diễn ra.
        \item \textbf{Process:} Hệ thống gửi thông báo (noti/email).
        \item \textbf{Output:} SV nhận được thông báo.
        \item \textbf{Constraints:} Thông báo $\geq$ 30 phút trước giờ học.
        \item \textbf{Acceptance:} SV xác nhận thông báo trên hệ thống.
        \item \textbf{Error handling:} Nếu thông báo lỗi thì gửi lại qua email dự phòng.
    \end{itemize}

    \item \textbf{Phản hồi và đánh giá chất lượng buổi học:}
    \begin{itemize}
        \item \textbf{Input:}  Điểm (1–5 sao), bình luận nhận xét.
        \item \textbf{Process:} Hệ thống lưu đánh giá gắn với buổi học và Tutor.
        \item \textbf{Output:} Thông tin phản hồi hiển thị cho Tutor và khoa/bộ môn.
        \item \textbf{Constraints:} Chỉ được đánh giá sau khi buổi học hoàn thành.
        \item \textbf{Acceptance:} Đánh giá hiển thị trong báo cáo tổng hợp.
        \item \textbf{Error handling:} Nếu buổi học chưa hoàn tất → từ chối đánh giá.
    \end{itemize}
\end{itemize}

%========================================================================================

\subsection*{2.1.2. Tác nhân phụ}
\addcontentsline{toc}{subsection}{2.1.2. Tác nhân phụ}

%========================================================================================

\subsubsection*{2.1.2.1. Khoa/Bộ môn}
\addcontentsline{toc}{subsubsection}{2.1.2.1. Khoa/Bộ môn}
\begin{itemize}
    \item \textbf{Nhận đánh giá và tổng hợp kết quả của sinh viên:}
    \begin{itemize}
        \item \textbf{Input:} Đánh giá (điểm số, nhận xét) từ sinh viên sau buổi học.
        \item \textbf{Process:} Hệ thống tổng hợp các phản hồi, phân loại theo môn học/Tutor.
        \item \textbf{Output:} Báo cáo chất lượng buổi học theo lớp, môn, Tutor.
        \item \textbf{Constraints:} Chỉ sử dụng đánh giá từ các buổi học hợp lệ.
        \item \textbf{Acceptance:} Báo cáo được cập nhật định kỳ (theo tuần/tháng).
        \item \textbf{Error handling:} Nếu thiếu dữ liệu đánh giá thì hệ thống ghi chú “chưa có đủ dữ liệu”.
    \end{itemize}
    
    \item \textbf{Quản lý chất lượng Tutor và SV:}
    \begin{itemize}
        \item \textbf{Input:} Hồ sơ Tutor, hồ sơ SV, số buổi học, đánh giá.
        \item \textbf{Process:} Khoa theo dõi, so sánh chất lượng giảng dạy và mức độ tiến bộ của SV.
        \item \textbf{Output:} Bảng xếp hạng/đánh giá Tutor và tổng kết tiến độ SV.
        \item \textbf{Constraints:} Dữ liệu phải dựa trên lịch sử buổi học và đánh giá chính thức.
        \item \textbf{Acceptance:} Báo cáo thể hiện chính xác tình hình giảng dạy – học tập.
        \item \textbf{Error handling:} Nếu dữ liệu không đồng bộ → hệ thống tự động cảnh báo để kiểm tra. 
    \end{itemize}
    
    \item \textbf{Theo dõi tiến độ học tập của sinh viên:}
    \begin{itemize}
        \item \textbf{Input:} GPA trước/sau, kết quả môn học, log buổi học.
        \item \textbf{Process:} Hệ thống đối chiếu tiến độ, xác định sự cải thiện.
        \item \textbf{Output:} Báo cáo cá nhân/tập thể về tiến bộ của SV.
        \item \textbf{Constraints:} Chỉ tính các SV tham gia tối thiểu X buổi học.
        \item \textbf{Acceptance:} Báo cáo có thể dùng làm cơ sở xét khen thưởng hoặc hỗ trợ.
        \item \textbf{Error handling:} Nếu thiếu GPA hoặc dữ liệu học tập → báo cáo đánh dấu “khuyết dữ liệu”.
    \end{itemize}
\end{itemize}

%========================================================================================

\subsubsection*{2.1.2.2. Phòng Công tác sinh viên}
\addcontentsline{toc}{subsubsection}{2.1.2.2. Phòng Công tác sinh viên}
\begin{itemize}
    \item \textbf{Nắm bắt GPA sinh viên trước và sau khi tham gia:}
    \begin{itemize}
        \item \textbf{Input:} GPA ban đầu, GPA cập nhật sau kỳ học.
        \item \textbf{Process:} Hệ thống tự động lấy dữ liệu từ HCMUT\_DATACORE, đối chiếu kết quả trước/sau.
        \item \textbf{Output:} Báo cáo so sánh GPA từng sinh viên.
        \item \textbf{Constraints:} Dữ liệu GPA phải đồng bộ chính xác từ hệ thống đào tạo.
        \item \textbf{Acceptance:} PCTSV có thể tra cứu sự thay đổi kết quả học tập của SV.
        \item \textbf{Error handling:} Nếu thiếu dữ liệu GPA thì hệ thống báo lỗi, yêu cầu đồng bộ lại.
    \end{itemize}
    
    \item \textbf{Tổng hợp kết quả tham gia:}
    \begin{itemize}
        \item \textbf{Input:} Danh sách SV, log số buổi học, đánh giá từ Tutor.
        \item \textbf{Process:} Hệ thống thống kê tần suất tham gia và kết quả học tập.
        \item \textbf{Output:} Báo cáo mức độ tham gia của SV
        \item \textbf{Constraints:} hỉ tính những SV tham gia tối thiểu số buổi học quy định.
        \item \textbf{Acceptance:} Báo cáo được xuất file (Excel/PDF) và tích hợp vào hệ thống quản lý SV.
        \item \textbf{Error handling:} Nếu dữ liệu không đầy đủ thì báo cáo gắn cờ “chưa hoàn chỉnh”.
    \end{itemize}
    
    \item \textbf{Ghi nhận kết quả tham gia để xét điểm rèn luyện / học bổng:}
    \begin{itemize}
        \item \textbf{Input:} Báo cáo tổng hợp SV tham gia chương trình Tutor.
        \item \textbf{Process:} PCTSV đối chiếu với quy chế điểm rèn luyện, học bổng.
        \item \textbf{Output:} Điểm rèn luyện/học bổng của SV được cập nhật.
        \item \textbf{Constraints:} Chỉ SV có tham gia hợp lệ, đủ số buổi quy định mới được ghi nhận.
        \item \textbf{Acceptance:} Kết quả được tích hợp vào hệ thống xét điểm rèn luyện và học bổng.
        \item \textbf{Error handling:} Nếu báo cáo thiếu dữ liệu thì đánh dấu “pending” cho đến khi bổ sung.
    \end{itemize}
\end{itemize}

%========================================================================================

\subsubsection*{2.1.2.3. Phòng Đào tạo}
\addcontentsline{toc}{subsubsection}{2.1.2.3. Phòng Đào tạo}
\begin{itemize}
    \item \textbf{Quản lý và theo dõi hồ sơ Tutor:}
    \begin{itemize}
        \item \textbf{Input:} Hồ sơ cá nhân, chuyên môn, lịch rảnh của Tutor.
        \item \textbf{Process:} PĐT xem, kiểm tra và xác nhận hồ sơ Tutor.
        \item \textbf{Output:} Danh sách Tutor hợp lệ được duyệt.
        \item \textbf{Constraints:} Chỉ Tutor đủ điều kiện (ví dụ GPA $\geq$ 7.0, có chuyên môn rõ ràng) mới được phê duyệt.
        \item \textbf{Acceptance:} Hồ sơ hiển thị trong hệ thống cho SV lựa chọn.
        \item \textbf{Error handling:} Nếu hồ sơ không hợp lệ thì trả lại yêu cầu cập nhật.
    \end{itemize}

    \item \textbf{Theo dõi số lượng buổi học:}
    \begin{itemize}
        \item \textbf{Input:} Log buổi học từ hệ thống.
        \item \textbf{Process:} Hệ thống tổng hợp số buổi học theo Tutor, theo SV, theo môn.
        \item \textbf{Output:} Báo cáo thống kê buổi học (ngày, giờ, trạng thái, số lượng).
        \item \textbf{Constraints:} Chỉ tính các buổi học hợp lệ (có điểm danh).
        \item \textbf{Acceptance:} Báo cáo hiển thị chính xác cho quản lý đào tạo.
        \item \textbf{Error handling:} Nếu dữ liệu log thiếu thì hệ thống cảnh báo “incomplete data”.
    \end{itemize}

    \item \textbf{Tối ưu phân bổ nguồn lực giữa Tutor và SV:}
    \begin{itemize}
        \item \textbf{Input:} Danh sách Tutor, danh sách SV đăng ký, nhu cầu hỗ trợ.
        \item \textbf{Process:} Hệ thống gợi ý phân bổ Tutor cho SV (theo ngành, lịch rảnh, số lượng tối đa).
        \item \textbf{Output:} Bảng phân công Tutor – SV.
        \item \textbf{Constraints:} Một Tutor chỉ nhận tối đa số SV theo quy định (ví dụ $\leq$ 5 SV).
        \item \textbf{Acceptance:} Phân bổ hợp lý, không quá tải Tutor, đáp ứng nhu cầu SV.
        \item \textbf{Error handling:} Nếu số SV vượt quá khả năng phân bổ thì hệ thống cảnh báo, yêu cầu thêm Tutor.
    \end{itemize}
\end{itemize}


%========================================================================================

\section*{2.2. Sơ đồ usecase toàn hệ thống}
\addcontentsline{toc}{section}{2.2. Sơ đồ usecase toàn hệ thống}


%========================================================================================

\section*{2.3. Yêu cầu chức năng}
\addcontentsline{toc}{section}{2.3. Yêu cầu chức năng}


%========================================================================================

\section*{2.4. Yêu cầu phi chức năng}
\addcontentsline{toc}{section}{2.4. Yêu cầu phi chức năng}
Để xây dựng một hệ thống kết nối Tutor và Sinh viên thực sự hiệu quả và đáng tin cậy, việc đáp ứng các yêu cầu về chức năng là chưa đủ. Yếu tố quyết định trải nghiệm người dùng và sự thành công lâu dài của dự án nằm ở các Yêu cầu phi chức năng (Non-Functional Requirements). Các tiêu chí này đặt ra những chuẩn mực về tốc độ, bảo mật, độ ổn định và tính dễ sử dụng của hệ thống. Những ràng buộc và tiêu chuẩn dưới đây sẽ là những yếu tố để đảm bảo hệ thống không chỉ hoàn thiện mà còn mang lại sự hài lòng và tin tưởng tuyệt đối cho mọi người dùng, từ sinh viên, Tutor đến các cấp quản lý.

\subsection*{2.4.1. Hiệu năng (Performance Requirements)}
\addcontentsline{toc}{subsection}{2.4.1. Hiệu năng (Performance Requirements)}
\begin{itemize}
    \item \textbf{Mô tả:}
    \item \textbf{Constraints:}
    \item \textbf{Acceptance:}
\end{itemize}

%========================================================================================

\subsection*{2.4.2. Bảo mật (Security Requirements)}
\addcontentsline{toc}{subsection}{2.4.2. Hiệu năng (Performance Requirements)}
\begin{itemize}
    \item \textbf{Mô tả:}
    \item \textbf{Constraints:}
    \item \textbf{Acceptance:}
\end{itemize}

%========================================================================================

\subsection*{2.4.3. Tính tin cậy \& sẵn sàng (Reliability \& Availability)}
\addcontentsline{toc}{subsection}{2.4.3. Tính tin cậy \& sẵn sàng (Reliability \& Availability)}
\begin{itemize}
    \item \textbf{Mô tả:}
    \item \textbf{Constraints:}
    \item \textbf{Acceptance:}
\end{itemize}

%========================================================================================

\subsection*{2.4.4. Khả năng sử dụng (Usability)}
\addcontentsline{toc}{subsection}{2.4.4. Khả năng sử dụng (Usability)}
\begin{itemize}
    \item \textbf{Mô tả:}
    \item \textbf{Constraints:}
    \item \textbf{Acceptance:}
\end{itemize}

%========================================================================================

\subsection*{2.4.5. Tính bảo trì \& mở rộng (Maintainability \& Extensibility)}
\addcontentsline{toc}{subsection}{2.4.5. Tính bảo trì \& mở rộng (Maintainability \& Extensibility)}
\begin{itemize}
    \item \textbf{Mô tả:}
    \item \textbf{Constraints:}
    \item \textbf{Acceptance:}
\end{itemize}

%========================================================================================

\subsection*{2.4.6. Khả năng tương thích (Compatibility)}
\addcontentsline{toc}{subsection}{2.4.6. Khả năng tương thích (Compatibility)}
\begin{itemize}
    \item \textbf{Mô tả:}
    \item \textbf{Constraints:}
    \item \textbf{Acceptance:}
\end{itemize}

%========================================================================================

\subsection*{2.4.7. Ràng buộc kỹ thuật (Technical Constraints)}
\addcontentsline{toc}{subsection}{2.4.7. Ràng buộc kỹ thuật (Technical Constraints)}
\begin{itemize}
    \item \textbf{Mô tả:}
    \item \textbf{Constraints:}
    \item \textbf{Acceptance:}
\end{itemize}

%========================================================================================