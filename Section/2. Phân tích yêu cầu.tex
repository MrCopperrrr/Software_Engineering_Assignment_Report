\part*{2. Phân tích yêu cầu dự án}
\addcontentsline{toc}{part}{2. Phân tích yêu cầu dự án}
Trong chương này, nhóm em sẽ tiến hành xác định và phân tích các yêu cầu của hệ thống. Trước hết, việc nhận diện các tác nhân (Actors) tương tác với hệ thống là bước quan trọng để hiểu rõ các luồng chức năng và mục tiêu mà phần mềm cần đáp ứng.

%========================================================================================

\section*{2.1. Tác nhân}
\addcontentsline{toc}{section}{2.1. Tác nhân}
Tác nhân (Actor) là một thực thể bên ngoài, có thể là người dùng hoặc một hệ thống khác, tương tác trực tiếp với hệ thống để thực hiện một mục tiêu cụ thể. Dựa trên vai trò và cách thức tương tác, các tác nhân của "Hệ thống hỗ trợ Tutor" được phân loại thành tác nhân chính và tác nhân phụ.

%========================================================================================

\subsection*{2.1.1. Tác nhân chính}
\addcontentsline{toc}{subsection}{2.1.1. Tác nhân chính}
\subsubsection*{2.1.1.1. Tutor}
Tutor có thể là sinh viên giỏi, NCS, hoặc giảng viên, được đăng ký vào hệ thống để hỗ trợ học tập. Họ có trách nhiệm quản lý hồ sơ, lịch rảnh, tổ chức buổi học và theo dõi tiến độ của sinh viên.
\begin{itemize}
    \item \textbf{Tạo và cập nhật lịch rảnh:}
    \begin{itemize}
        \item \textbf{Input:} Ngày, giờ, hình thức học (online/offline).
        \item \textbf{Process:} Hệ thống lưu lại khung giờ rảnh, đồng bộ với module đặt lịch.
        \item \textbf{Output:} Lịch rảnh hiển thị cho sinh viên để chọn.
        \item \textbf{Constraints:} Không được nhập trùng khung giờ.
        \item \textbf{Acceptance:} Sinh viên có thể đặt lịch học trong khung rảnh của Tutor.
        \item \textbf{Error handling:} Nếu nhập sai định dạng hoặc trùng lịch thì hệ thống báo lỗi.
    \end{itemize}
    
    \item \textbf{Mở (oneline/offline), hủy, đổi buổi học:}
    \begin{itemize}
        \item \textbf{Input:} Yêu cầu mở buổi học, hoặc yêu cầu huỷ/đổi.
        \item \textbf{Process:} Hệ thống kiểm tra lịch đã có sinh viên đặt chưa, xử lý cập nhật.
        \item \textbf{Output:} Buổi học được thêm/sửa/xoá trong hệ thống.
        \item \textbf{Constraints:} Hủy/đổi $\geq$ 3h trước giờ học
        \item \textbf{Acceptance:} Sinh viên và Tutor đều nhận được thông báo cập nhật.
        \item \textbf{Error handling:} Nếu yêu cầu đổi sát giờ hệ thống từ chối, báo lỗi.
    \end{itemize}
    
    \item \textbf{Nhận thông báo và nhắc nhở giờ dạy:}
    \begin{itemize}
        \item \textbf{Input:} Lịch học sắp diễn ra.
        \item \textbf{Process:} Hệ thống gửi thông báo (noti/email).
        \item \textbf{Output:} Tutor nhận được thông báo đúng hạn.
        \item \textbf{Constraints:} Thông báo phải được gửi $\geq$ 30 phút trước giờ học.
        \item \textbf{Acceptance:} Tutor xác nhận đã đọc thông báo.
        \item \textbf{Error handling:} Nếu gửi lỗi → hệ thống gửi lại lần 2 hoặc báo qua email dự phòng.
    \end{itemize}
    
    \item \textbf{Theo dõi tiến bộ sinh viên:}
    \begin{itemize}
        \item \textbf{Input:} Điểm, nhận xét, đánh giá sau buổi học.
        \item \textbf{Process:} Tutor nhập vào form đánh giá, hệ thống lưu lại.
        \item \textbf{Output:} Báo cáo tiến bộ gắn với hồ sơ sinh viên.
        \item \textbf{Constraints:} Chỉ Tutor đã dạy sinh viên đó mới được nhập.
        \item \textbf{Acceptance:} Khoa/bộ môn có thể truy cập báo cáo.
        \item \textbf{Error handling:} Nếu buổi học chưa hoàn tất thì từ chối ghi nhận.
    \end{itemize}
    
    \item \textbf{Điểm danh và record:}
    \begin{itemize}
        \item \textbf{Input:} ID sinh viên tham gia, mã buổi học.
        \item \textbf{Process:} Hệ thống điểm danh, ghi log tham dự.
        \item \textbf{Output:} Record buổi học (thời lượng, người tham gia).
        \item \textbf{Constraints:} Mỗi SV chỉ được điểm danh vào 1 buổi học tại 1 thời điểm.
        \item \textbf{Acceptance:} Log lưu thành công và hiển thị trong báo cáo.
        \item \textbf{Error handling:} Nếu trùng ID hệ thống từ chối, báo lỗi.
    \end{itemize}
    
    \item \textbf{Cập nhật trạng thái buổi học:}
    \begin{itemize}
        \item \textbf{Input:} Trạng thái (hoàn thành, huỷ, đang diễn ra).
        \item \textbf{Process:} Tutor xác nhận trạng thái, hệ thống lưu lại.
        \item \textbf{Output:} Buổi học hiển thị trạng thái mới.
        \item \textbf{Constraints:} Trạng thái chỉ được thay đổi bởi Tutor của buổi học.
        \item \textbf{Acceptance:} Sinh viên và khoa/bộ môn nhìn thấy trạng thái chính xác.
        \item \textbf{Error handling:} Nếu cập nhật sai thì hệ thống cho phép sửa lại trong 24h.
    \end{itemize}

    \item \textbf{Đăng nội dung bài học:}
    \begin{itemize}
        \item \textbf{Input:} File/tài liệu/note buổi học.
        \item \textbf{Process:} Upload vào hệ thống, lưu trữ trong HCMUT\_LIBRARY.
        \item \textbf{Output:} Sinh viên có thể tải xuống.
        \item \textbf{Constraints:} Dung lượng $\leq$ 50MB/file.
        \item \textbf{Acceptance:} Nội dung hiển thị đúng với sinh viên liên quan.
        \item \textbf{Error handling:} Nếu file hỏng thì báo lỗi, yêu cầu upload lại.
    \end{itemize}
\end{itemize}

%========================================================================================

\subsubsection*{2.1.1.2. Sinh viên}
 Sinh viên là đối tượng cần hỗ trợ, tham gia hệ thống để tìm Tutor, đặt lịch học và nhận hỗ trợ học tập.
\begin{itemize}
    \item \textbf{Tạo tài khoản, hồ sơ cá nhân:}
    \begin{itemize}
        \item \textbf{Input:} Họ tên, MSSV, email, số điện thoại, thông tin học tập (GPA, môn cần hỗ trợ).
        \item \textbf{Process:} Hệ thống kiểm tra định dạng dữ liệu, đồng bộ với HCMUT\_DATACORE.
        \item \textbf{Output:} Hồ sơ cá nhân của SV được lưu và hiển thị trong hệ thống.
        \item \textbf{Constraints:} MSSV và email phải trùng khớp dữ liệu HCMUT.
        \item \textbf{Acceptance:} SV có thể đăng nhập và sử dụng các chức năng khác.
        \item \textbf{Error handling:} Nếu dữ liệu không hợp lệ → hệ thống báo lỗi, yêu cầu sửa.
    \end{itemize}

    \item \textbf{Đăng ký chương trình học:}
    \begin{itemize}
        \item \textbf{Input:} Môn học hoặc lĩnh vực cần hỗ trợ, nguyện vọng học tập.
        \item \textbf{Process:} Hệ thống ghi nhận nhu cầu, đồng bộ với dữ liệu đào tạo và gợi ý Tutor phù hợp.
        \item \textbf{Output:} Hồ sơ SV được cập nhật với chương trình đã đăng ký.
        \item \textbf{Constraints:} Chỉ được đăng ký trong danh sách môn/lĩnh vực mà hệ thống hỗ trợ.
        \item \textbf{Acceptance:} SV thấy chương trình học hiển thị trong hồ sơ.
        \item \textbf{Error handling:} Nếu môn/lĩnh vực không hợp lệ thì hệ thống báo lỗi, yêu cầu chọn lại.
    \end{itemize}

    \item \textbf{Lựa chọn Tutor / được ghép tự động:}
    \begin{itemize}
        \item \textbf{Input:} Nhu cầu hỗ trợ (môn, lịch, hình thức).
        \item \textbf{Process:} 
        \begin{itemize}
            \item \textbf{Thủ công:} SV chọn Tutor trong danh sách.
            \item \textbf{Tự động: } Hệ thống so khớp theo khoa/ngành, lịch rảnh, AI ranking.
        \end{itemize}
        \item \textbf{Output:} Ghép cặp Tutor – SV được xác lập.
        \item \textbf{Constraints:} Một SV chỉ có 1 Tutor chính tại một thời điểm.
        \item \textbf{Acceptance:} SV thấy thông tin Tutor trong hồ sơ.
        \item \textbf{Error handling:} Nếu lịch trùng thì yêu cầu chọn lại hoặc hệ thống gợi ý Tutor khác.
    \end{itemize}

    \item \textbf{Đặt lịch học (cảnh báo trùng lịch):}
    \begin{itemize}
        \item \textbf{Input:} Ngày, giờ, môn học.
        \item \textbf{Process:} Hệ thống kiểm tra lịch rảnh của Tutor và lịch của SV.
        \item \textbf{Output:} Lịch học mới được thêm.
        \item \textbf{Constraints:} Không được đặt trùng với lịch học hoặc lịch thi chính thức.
        \item \textbf{Acceptance:} Lịch hiển thị trong tài khoản SV và Tutor.
        \item \textbf{Error handling:} Nếu trùng lịch thì cảnh báo, từ chối đặt.
    \end{itemize}

    \item \textbf{Nhận thông báo và nhắc nhở giờ học:}
    \begin{itemize}
        \item \textbf{Input:}
        \item \textbf{Process:}
        \item \textbf{Output:}
        \item \textbf{Constraints:}
        \item \textbf{Acceptance:}
        \item \textbf{Error handling:}
    \end{itemize}

    \item \textbf{Phản hồi và đánh giá chất lượng buổi học:}
    \begin{itemize}
        \item \textbf{Input:}
        \item \textbf{Process:}
        \item \textbf{Output:}
        \item \textbf{Constraints:}
        \item \textbf{Acceptance:}
        \item \textbf{Error handling:}
    \end{itemize}
\end{itemize}

%========================================================================================

\subsection*{2.1.2. Tác nhân phụ}
\addcontentsline{toc}{subsection}{2.1.2. Tác nhân phụ}

%========================================================================================

\subsubsection*{2.1.2.1. Khoa/Bộ môn}
\begin{itemize}
    \item \textbf{Nhận đánh giá và tổng hợp kết quả của sinh viên:}
    \item \textbf{Quản lý chất lượng Tutor và SV:}
    \item \textbf{Theo dõi tiến độ học tập của sinh viên:}
\end{itemize}

%========================================================================================

\subsubsection*{2.1.2.2. Phòng Công tác sinh viên}
\begin{itemize}
    \item \textbf{Nắm bắt GPA sinh viên trước và sau khi tham gia:}
    \item \textbf{Tổng hợp kết quả tham gia:}
    \item \textbf{Ghi nhận kết quả tham gia để xét điểm rèn luyện / học bổng:}
\end{itemize}

%========================================================================================

\subsubsection*{2.1.2.3. Phòng Đào tạo}
\begin{itemize}
    \item \textbf{Quản lý và theo dõi hồ sơ Tutor:}
    \item \textbf{Theo dõi số lượng buổi học:}
    \item \textbf{Tối ưu phân bổ nguồn lực giữa Tutor và SV:}
\end{itemize}


%========================================================================================

\section*{2.2. Yêu cầu chức năng}
\addcontentsline{toc}{section}{2.2. Yêu cầu chức năng}


%========================================================================================

\section*{2.3. Yêu cầu phi chức năng}
\addcontentsline{toc}{section}{2.3. Yêu cầu phi chức năng}


%========================================================================================

\section*{2.4. Sơ đồ usecase toàn hệ thống}
\addcontentsline{toc}{section}{2.4. Sơ đồ usecase toàn hệ thống}


%========================================================================================

\section*{2.5. Đặc tả usecase}
\addcontentsline{toc}{section}{2.5. Đặc tả usecase}
