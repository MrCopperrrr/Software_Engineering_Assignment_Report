\part*{4. Triển khai hệ thống}
\addcontentsline{toc}{part}{4. Triển khai hệ thống}

%========================================================================================
\section*{4.1. Sơ đồ triển khai (Deployment View)}
\addcontentsline{toc}{section}{4.1. Sơ đồ triển khai (Deployment View)}

\subsection*{4.1.1 Giới thiệu}
\addcontentsline{toc}{subsection}{4.1.1 Giới thiệu}
Sơ đồ triển khai (Deployment Diagram) mô tả kiến trúc vật lý của "Hệ thống hỗ trợ Tutor", thể hiện cách các thành phần phần mềm được phân bổ và vận hành trên các nút (node) phần cứng. Sơ đồ này cung cấp một cái nhìn tổng quan về môi trường thực thi của hệ thống, bao gồm các máy chủ, cơ sở dữ liệu, và sự tương tác giữa chúng.

Kiến trúc được lựa chọn là mô hình \textbf{client-server ba lớp (3-tier)} hiện đại, hoạt động hoàn toàn \textbf{độc lập (Standalone)}, bao gồm:

\begin{itemize}
	\item \textbf{Presentation Tier (Client)}: Giao diện người dùng trên trình duyệt web.
	\item \textbf{Application Tier (Server)}: Máy chủ ứng dụng xử lý logic nghiệp vụ và cung cấp API.
	\item \textbf{Data Tier (Database)}: Máy chủ cơ sở dữ liệu lưu trữ toàn bộ thông tin hệ thống.
\end{itemize}

\subsection*{4.1.2. Sơ đồ triển khai hệ thống}
\addcontentsline{toc}{subsection}{4.1.2. Sơ đồ triển khai hệ thống}
Sơ đồ dưới đây được tạo bằng PlantUML, mô tả chi tiết các thành phần và luồng tương tác.

\begin{figure}[H]
	\centering
	\setlength{\fboxsep}{2pt}     
	\setlength{\fboxrule}{0.5pt}   
	\fbox{\includegraphics[scale=0.3]{Picture/Deploymentview.png}}
	\caption{Sơ đồ triển khai hệ thống hỗ trợ Tutor (Standalone)}
\end{figure}

\subsection*{4.1.3. Mô tả các thành phần}
\addcontentsline{toc}{subsection}{4.1.3. Mô tả các thành phần}

\begin{table}[H]
	\centering
	\begin{tabular}{|c|p{3.5cm}|p{6cm}|p{3.5cm}|}
		\hline
		\textbf{STT} & \textbf{Tên thành phần} & \textbf{Mô tả} & \textbf{Công nghệ/Phần mềm} \\
		\hline
		1 & Người dùng & Các tác nhân (Sinh viên, Tutor, Admin) tương tác với hệ thống thông qua trình duyệt web. & Trình duyệt Web (Chrome, Firefox, Safari) \\
		\hline
		2 & Web Server (Nginx) & Đóng vai trò là Reverse Proxy, tiếp nhận yêu cầu từ Internet và chuyển tiếp đến Application Server. & Nginx \\
		\hline
		3 & Application Server & Chứa toàn bộ logic nghiệp vụ, module xác thực và GraphQL Resolvers. Đóng gói trong Docker container. & \textbf{Node.js} (Express + Apollo Server) \\
		\hline
		4 & Database Server & Chịu trách nhiệm lưu trữ và quản lý toàn bộ dữ liệu của hệ thống (User, Course, Class). & \textbf{MySQL} (Thư viện mysql2 pool) \\
		\hline
		5 & Email Service & Dịch vụ bên ngoài dùng để gửi mã xác thực OTP qua email. & \textbf{SendGrid API} \\
		\hline
		6 & CI/CD Pipeline & Hệ thống tự động hóa quy trình kiểm thử và triển khai mã nguồn. & \textbf{Azure Pipelines} \\
		\hline
	\end{tabular}
	\caption{Các thành phần trong sơ đồ triển khai hệ thống hỗ trợ Tutor}
\end{table}

\subsection*{4.1.4. Luồng hoạt động}
\addcontentsline{toc}{subsection}{4.1.4. Luồng hoạt động}
Mô tả chi tiết luồng hoạt động cho các kịch bản tiêu biểu trong môi trường độc lập:

\begin{enumerate}[label=\Alph*.]
	\item \textbf{Kịch bản 1: Người dùng đăng ký và xác thực tài khoản}
	\begin{enumerate}[label*=\arabic*.]
		\item Người dùng nhập email và thông tin đăng ký trên giao diện Client.
		\item Yêu cầu gửi mã OTP (GraphQL Mutation) được gửi đến \textbf{Application Server}.
		\item \textbf{Application Server} sinh mã OTP ngẫu nhiên, lưu tạm thời vào bộ nhớ RAM (In-memory) và gọi API đến \textbf{SendGrid} để gửi email cho người dùng.
		\item Người dùng nhập mã OTP để xác thực. \textbf{Application Server} kiểm tra mã, nếu đúng sẽ tạo tài khoản mới trong \textbf{Database Server} và trả về JWT Token.
	\end{enumerate}
	
	\item \textbf{Kịch bản 2: Tutor mở lớp học mới (Open Class)}
	\begin{enumerate}[label*=\arabic*.]
		\item \textbf{Tutor} đăng nhập, gọi API \texttt{openClass} với thông tin: Ngày, Giờ bắt đầu/kết thúc, Hình thức (Online/Offline).
		\item \textbf{Application Server} xác thực Token của Tutor và kiểm tra tính hợp lệ của dữ liệu (ví dụ: giờ bắt đầu < giờ kết thúc).
		\item Nếu hợp lệ, Server thực hiện câu lệnh \texttt{INSERT} vào bảng \texttt{Class} trong \textbf{Database Server}.
		\item Hệ thống trả về thông tin lớp học vừa tạo để hiển thị lên giao diện Tutor.
	\end{enumerate}
	
	\item \textbf{Kịch bản 3: Sinh viên đăng ký môn học}
	\begin{enumerate}[label*=\arabic*.]
		\item Sinh viên xem danh sách các môn học khả dụng (được lấy từ bảng \texttt{Course} trong Database).
		\item Sinh viên chọn môn học và nhấn "Đăng ký". Client gửi Mutation \texttt{enrollCourse} đến Server.
		\item \textbf{Application Server} thực hiện kiểm tra kép:
		\begin{itemize}
			\item Môn học có tồn tại không? (Query bảng \texttt{Course})
			\item Sinh viên đã đăng ký môn này chưa? (Query bảng \texttt{CourseRegistration})
		\end{itemize}
		\item Nếu thỏa mãn, Server tạo bản ghi mới trong bảng \texttt{CourseRegistration} và trả về thông báo thành công.
	\end{enumerate}
\end{enumerate}
%========================================================================================
\section*{4.2. Sơ đồ phát triển (Development View)}
\addcontentsline{toc}{section}{4.2. Sơ đồ phát triển (Development View)}

\subsection*{4.2.1 Giới thiệu}
\addcontentsline{toc}{subsection}{4.2.1 Giới thiệu}
Sơ đồ phát triển, hay \textbf{Sơ đồ gói (Package Diagram)}, mô tả cấu trúc tĩnh và cách tổ chức mã nguồn của hệ thống từ góc nhìn của đội ngũ phát triển.

Hệ thống được thiết kế theo kiến trúc \textbf{phân lớp (Layered Architecture)} trên nền tảng \textbf{Node.js}, sử dụng giao thức \textbf{GraphQL}. Các lớp chính được ánh xạ vào cấu trúc thư mục của dự án như sau:

\begin{itemize}
	\item \textbf{Interface Layer (GraphQL Resolvers)}: Tiếp nhận các truy vấn (Query) và thay đổi dữ liệu (Mutation) từ phía Client.
	\item \textbf{Logic Layer (Services/Utils)}: Chứa các hàm xử lý logic nghiệp vụ như gửi mail, tạo OTP.
	\item \textbf{Data Access Layer (Models)}: Chứa các hàm tương tác trực tiếp với cơ sở dữ liệu MySQL (thư mục \texttt{models/}).
	\item \textbf{Data Objects}: Các đối tượng dữ liệu được định nghĩa trong GraphQL Schema (\texttt{typeDefs}).
\end{itemize}

Các gói hỗ trợ xuyên suốt:
\begin{itemize}
	\item \textbf{Security}: Module xác thực JWT và mã hóa mật khẩu (\texttt{bcryptjs}).
	\item \textbf{Configuration}: Cấu hình kết nối Database và biến môi trường (\texttt{dotenv}).
	\item \textbf{Integration}: Module tích hợp gửi Email qua SendGrid (\texttt{MailSender}).
\end{itemize}

\subsection*{4.2.2. Sơ đồ gói (Package Diagram)}
\addcontentsline{toc}{subsection}{4.2.2. Sơ đồ gói (Package Diagram)}

\begin{figure}[H]
	\centering
	\setlength{\fboxsep}{2pt}     
	\setlength{\fboxrule}{0.5pt}   
	\fbox{\includegraphics[scale=0.45]{Picture/Developmentview.png}}
	\caption{Sơ đồ tổ chức các gói của hệ thống (Logical View)}
\end{figure}

\subsection*{4.2.3. Mô tả các gói}
\addcontentsline{toc}{subsection}{4.2.3. Mô tả các gói}
Bảng dưới đây giải thích chức năng của từng gói trong kiến trúc thực tế:

\begin{table}[H]
	\centering
	\begin{tabular}{|c|p{4cm}|p{9cm}|}
		\hline
		\textbf{STT} & \textbf{Tên gói (Logic)} & \textbf{Ánh xạ trong Code (Thực tế)} \\
		\hline
		1 & Controller/Resolver Layer & Thư mục \texttt{resolvers/}: Chứa các hàm xử lý đầu vào từ GraphQL (Query/Mutation), gọi xuống tầng Model để lấy dữ liệu. \\
		\hline
		2 & Service/Utils Layer & Thư mục \texttt{MailSender/} và các hàm tiện ích: Xử lý logic nghiệp vụ đặc thù như sinh mã OTP, gửi email. \\
		\hline
		3 & Repository/Model Layer & Thư mục \texttt{models/} (\texttt{userModel.js}, \texttt{courseModel.js}...): Chứa các câu lệnh SQL để truy xuất và thao tác dữ liệu trực tiếp với MySQL. \\
		\hline
		4 & Domain/Schema Layer & Thư mục \texttt{schemas/} (\texttt{index.js}): Định nghĩa các Type, Input, Enum của GraphQL, đóng vai trò như hợp đồng dữ liệu giữa Client và Server. \\
		\hline
		-- & Các gói phụ trợ & \texttt{database.js} (kết nối DB), \texttt{server.mjs} (cấu hình Server, Middleware xác thực). \\
		\hline
	\end{tabular}
	\caption{Chức năng của từng gói trong kiến trúc Node.js/GraphQL}
\end{table}

\subsection*{4.2.4. Luồng xử lý dữ liệu qua các lớp}
\addcontentsline{toc}{subsection}{4.2.4. Luồng xử lý dữ liệu qua các lớp}

Minh họa luồng xử lý cho chức năng ``Sinh viên đăng ký môn học'' (UC-04):

\begin{itemize}
	\item \textbf{Request}: Client gửi một \texttt{Mutation: enrollCourse} thông qua GraphQL API.
	\item \textbf{Interface Layer (Resolver)}: Hàm \texttt{enrollCourse} trong \texttt{resolvers/index.js} nhận yêu cầu. Nó kiểm tra quyền truy cập (thông qua \texttt{context.userId}) và gọi các hàm kiểm tra logic.
	\item \textbf{Data Access Layer (Model)}:
	\begin{itemize}
		\item Gọi \texttt{courseExists(id)} trong \texttt{models/courseModel.js} để kiểm tra môn học có tồn tại không.
		\item Gọi \texttt{isCourseRegistered(...)} để đảm bảo sinh viên chưa đăng ký trùng.
		\item Cuối cùng gọi \texttt{enrollCourse(...)} để thực hiện câu lệnh \texttt{INSERT} vào database.
	\end{itemize}
	\item \textbf{Response}: Kết quả (True/False hoặc Error Message) được trả về từ tầng Model -> Resolver -> Client theo định dạng JSON chuẩn của GraphQL.
\end{itemize}
%========================================================================================
\section*{4.3. Sơ đồ lớp chi tiết (Detailed Class Diagram)}
\addcontentsline{toc}{section}{4.3. Sơ đồ lớp chi tiết (Detailed Class Diagram)}

\subsection*{4.3.1. Giới thiệu}
\addcontentsline{toc}{subsection}{4.3.1. Giới thiệu}
Sơ đồ lớp là một sơ đồ cấu trúc tĩnh trong UML, mô tả chi tiết các lớp (\textit{class}), thuộc tính (\textit{attributes}), phương thức (\textit{methods}) và mối quan hệ giữa các lớp đó trong \textbf{Hệ thống hỗ trợ Tutor}. Sơ đồ này đóng vai trò là bản thiết kế chi tiết cho việc triển khai mã nguồn, đảm bảo hệ thống được xây dựng một cách nhất quán, có cấu trúc và dễ dàng bảo trì, mở rộng.

Dựa trên kiến trúc mã nguồn thực tế, sơ đồ được tổ chức thành 4 cụm chức năng cốt lõi:

\begin{itemize}
	\item \textbf{Cụm Quản lý Người dùng (User Management)}: Định nghĩa các thực thể người dùng và phân quyền trong hệ thống.
	\item \textbf{Cụm Khóa học và Đăng ký (Course \& Registration)}: Quản lý danh mục môn học và các liên kết đăng ký từ cả hai phía Sinh viên và Tutor.
	\item \textbf{Cụm Lịch trình và Lớp học (Scheduling \& Class)}: Quản lý việc mở lớp, thiết lập thời gian và hình thức dạy học.
	\item \textbf{Cụm Đánh giá và Hỗ trợ (Feedback \& Support)}: Các chức năng mở rộng hỗ trợ quá trình vận hành lớp học như tài liệu, đánh giá và điểm danh.
\end{itemize}

\subsection*{4.3.2. Sơ đồ lớp chi tiết của hệ thống}
\addcontentsline{toc}{subsection}{4.3.2. Sơ đồ lớp chi tiết của hệ thống}

\begin{figure}[H]
	\centering
	\setlength{\fboxsep}{2pt}     
	\setlength{\fboxrule}{0.5pt}   
	\fbox{\includegraphics[scale=0.48]{Picture/ClassDiagram.png}}
	\caption{Sơ đồ lớp chi tiết của hệ thống (Refined)}
\end{figure}

\subsection*{4.3.3. Mô tả cụm chức năng}
\addcontentsline{toc}{subsection}{4.3.3. Mô tả cụm chức năng}

\begin{itemize}
	\item \textbf{Cụm Quản lý Người dùng (User Management):}
	\begin{itemize}
		\item Lớp trừu tượng \texttt{User} là lớp cơ sở, chứa các thuộc tính chung (UserID, Email, PasswordHash, Phone) và phương thức xác thực.
		\item Lớp \texttt{Student} kế thừa từ \texttt{User}, bổ sung thuộc tính đặc thù như \texttt{studentCode} (MSSV) và \texttt{gpa}.
		\item Lớp \texttt{Tutor} kế thừa từ \texttt{User}, bổ sung thông tin về giới thiệu bản thân (\texttt{bio}) và điểm đánh giá trung bình (\texttt{averageRating}).
	\end{itemize}
	
	\item \textbf{Cụm Khóa học và Đăng ký (Course \& Registration):}
	\begin{itemize}
		\item \texttt{Course} (thay thế cho TutoringProgram): Đại diện cho một môn học cụ thể được quản lý trong cơ sở dữ liệu (ví dụ: "Cấu trúc dữ liệu").
		\item \texttt{CourseRegistration}: Ghi nhận việc Sinh viên đăng ký nhu cầu học một môn học cụ thể.
		\item \texttt{TutorCourseRegistration}: Ghi nhận năng lực giảng dạy của Tutor, liên kết Tutor với các môn học (\texttt{Course}) mà họ có thể dạy. Đây là cơ sở dữ liệu để hệ thống thực hiện việc tìm kiếm và lọc Tutor phù hợp.
	\end{itemize}
	
	\item \textbf{Cụm Lịch trình và Lớp học (Scheduling \& Class):}
	\begin{itemize}
		\item \texttt{Class}: Đây là thực thể trung tâm của hệ thống (thay thế cho \texttt{AvailabilitySlot} và \texttt{Session} cũ). Tutor sẽ trực tiếp "mở lớp" (\texttt{openClass}) với đầy đủ thông tin: Thời gian bắt đầu/kết thúc, Ngày trong tuần, Hình thức (Online/Offline) và Phòng học.
		\item Lớp \texttt{Class} đóng vai trò kết nối giữa Tutor (người mở lớp) và Sinh viên (người tham gia lớp học).
	\end{itemize}
	
	\item \textbf{Cụm Đánh giá và Hỗ trợ (Feedback \& Support):}
	\begin{itemize}
		\item \texttt{Feedback}: Cho phép lưu trữ đánh giá hai chiều giữa Sinh viên và Tutor. Mỗi phản hồi được gắn kết chặt chẽ với một \texttt{Class} cụ thể để đảm bảo tính xác thực.
		\item \texttt{Material}: Quản lý các tài liệu học tập (Slide, Bài tập) được Tutor tải lên cho một lớp học.
		\item \texttt{Attendance}: Hỗ trợ Tutor ghi nhận trạng thái tham gia (Có mặt/Vắng mặt) của sinh viên trong từng buổi học.
	\end{itemize}
\end{itemize}

%========================================================================================
\section*{4.4. Mô tả các lớp chi tiết và phương thức}
\addcontentsline{toc}{section}{4.4. Mô tả các lớp chi tiết và phương thức}

\subsection*{4.4.1. Cụm Quản lý Người dùng (User Management)}
\addcontentsline{toc}{subsection}{4.4.1. Cụm Quản lý Người dùng (User Management)}

Đây là cụm lõi, định nghĩa các đối tượng người dùng trong hệ thống. Việc sử dụng kế thừa từ một lớp trừu tượng \texttt{User} giúp tối ưu hóa cấu trúc và tránh lặp lại mã nguồn.

\textbf{Mô tả:} Là lớp cơ sở (base class) cho tất cả các loại người dùng trong hệ thống. Lớp này chứa các thông tin chung nhất.

\textbf{Thuộc tính (Attributes):}
\begin{table}[H]
	\centering
	\begin{tabularx}{\textwidth}{|l|l|X|}
		\hline
		\textbf{Tên thuộc tính} & \textbf{Kiểu dữ liệu} & \textbf{Mô tả} \\
		\hline
		- UserID & UUID & Khóa chính, mã định danh duy nhất cho mỗi người dùng. \\
		- FullName & String & Họ và tên đầy đủ. \\
		- Email & String & Địa chỉ email dùng để đăng nhập và nhận thông báo. \\
		- Password & String & Mật khẩu đã được mã hóa. \\
		- Role & Enum & Vai trò (STUDENT, TUTOR, ADMIN). \\
		- Phone & String & Số điện thoại liên lạc. \\
		- Introduce & String & Giới thiệu bản thân. \\
		\hline
	\end{tabularx}
\end{table}

\textbf{Phương thức (Methods):}
\begin{table}[H]
	\centering
	\begin{tabularx}{\textwidth}{|l|X|l|X|}
		\hline
		\textbf{Tên phương thức} & \textbf{Tham số} & \textbf{Kiểu trả về} & \textbf{Mô tả} \\
		\hline
		+ updateUser() & data: UserInput & User & Cập nhật thông tin cá nhân. \\
		+ requestOtp() & email: String & Boolean & Gửi yêu cầu mã OTP để xác thực. \\
		+ verifyOtp() & email: String, code: String & Boolean & Xác thực mã OTP người dùng nhập. \\
		\hline
	\end{tabularx}
\end{table}

\textbf{Mô tả:} Lớp \texttt{Student} kế thừa từ \texttt{User}, đại diện cho sinh viên.

\textbf{Thuộc tính (Attributes):}
\begin{table}[H]
	\centering
	\begin{tabularx}{\textwidth}{|l|l|X|}
		\hline
		\textbf{Tên thuộc tính} & \textbf{Kiểu dữ liệu} & \textbf{Mô tả} \\
		\hline
		- StudentCode & String & Mã số sinh viên (MSSV). \\
		\hline
	\end{tabularx}
\end{table}

\textbf{Mô tả:} Lớp \texttt{Tutor} kế thừa từ \texttt{User}, đại diện cho người dạy.

\textbf{Thuộc tính (Attributes):}
\begin{table}[H]
	\centering
	\begin{tabularx}{\textwidth}{|l|l|X|}
		\hline
		\textbf{Tên thuộc tính} & \textbf{Kiểu dữ liệu} & \textbf{Mô tả} \\
		\hline
		- Bio & String & Giới thiệu về kinh nghiệm và chuyên môn. \\
		- AverageRating & Double & Điểm đánh giá trung bình. \\
		\hline
	\end{tabularx}
\end{table}

\subsection*{4.4.2. Cụm Khóa học và Đăng ký (Course \& Registration)}
\addcontentsline{toc}{subsection}{4.4.2. Cụm Khóa học và Đăng ký (Course \& Registration)}

Cụm này quản lý danh mục môn học và các liên kết đăng ký từ cả hai phía.

\textbf{Mô tả:} Lớp \texttt{Course} đại diện cho một môn học cụ thể trong hệ thống.

\textbf{Thuộc tính (Attributes):}
\begin{table}[H]
	\centering
	\begin{tabularx}{\textwidth}{|l|l|X|}
		\hline
		\textbf{Tên thuộc tính} & \textbf{Kiểu dữ liệu} & \textbf{Mô tả} \\
		\hline
		- Cid & String & Mã định danh môn học (ví dụ: CO3001). \\
		- Cname & String & Tên môn học. \\
		- Faculty & String & Khoa quản lý môn học. \\
		\hline
	\end{tabularx}
\end{table}

\textbf{Phương thức (Methods):}
\begin{table}[H]
	\centering
	\begin{tabularx}{\textwidth}{|l|X|l|X|}
		\hline
		\textbf{Tên phương thức} & \textbf{Tham số} & \textbf{Kiểu trả về} & \textbf{Mô tả} \\
		\hline
		+ getAvailableCourses() & userId: UUID & List<Course> & Lấy danh sách các môn học sinh viên chưa đăng ký. \\
		+ getRegisteredCourses() & userId: UUID & List<Course> & Lấy danh sách các môn học sinh viên đã đăng ký. \\
		\hline
	\end{tabularx}
\end{table}

\textbf{Mô tả:} Lớp \texttt{CourseRegistration} liên kết Sinh viên với Môn học.

\textbf{Phương thức (Methods):}
\begin{table}[H]
	\centering
	\begin{tabularx}{\textwidth}{|l|X|l|X|}
		\hline
		\textbf{Tên phương thức} & \textbf{Tham số} & \textbf{Kiểu trả về} & \textbf{Mô tả (Liên kết với Use Case)} \\
		\hline
		+ enrollCourse() & userId: UUID, courseId: String & Boolean & Sinh viên đăng ký một môn học (UC-04). \\
		+ cancelEnrollCourse() & userId: UUID, courseId: String & Boolean & Hủy đăng ký môn học (UC-05). \\
		\hline
	\end{tabularx}
\end{table}

\textbf{Mô tả:} Lớp \texttt{TutorCourseRegistration} liên kết Tutor với Môn học họ dạy.

\textbf{Phương thức (Methods):}
\begin{table}[H]
	\centering
	\begin{tabularx}{\textwidth}{|l|X|l|X|}
		\hline
		\textbf{Tên phương thức} & \textbf{Tham số} & \textbf{Kiểu trả về} & \textbf{Mô tả} \\
		\hline
		+ upsertTutorMajor() & tutorId: UUID, courses: List<String> & Boolean & Cập nhật danh sách các môn mà Tutor có thể dạy. \\
		\hline
	\end{tabularx}
\end{table}

\subsection*{4.4.3. Cụm Lịch trình và Lớp học (Scheduling \& Class)}
\addcontentsline{toc}{subsection}{4.4.3. Cụm Lịch trình và Lớp học (Scheduling \& Class)}

Cụm này quản lý việc mở lớp và thông tin chi tiết của lớp học.

\textbf{Mô tả:} Lớp \texttt{Class} đại diện cho một lớp học do Tutor mở, bao gồm thông tin thời gian và hình thức. Đây là thực thể trung tâm thay thế cho AvailabilitySlot và Session cũ.

\textbf{Thuộc tính (Attributes):}
\begin{table}[H]
	\centering
	\begin{tabularx}{\textwidth}{|l|l|X|}
		\hline
		\textbf{Tên thuộc tính} & \textbf{Kiểu dữ liệu} & \textbf{Mô tả} \\
		\hline
		- ClassID & UUID & Khóa chính của lớp học. \\
		- TutorID & UUID & ID của Tutor phụ trách lớp. \\
		- StartTime & String & Giờ bắt đầu (ví dụ: "07:00"). \\
		- EndTime & String & Giờ kết thúc (ví dụ: "09:00"). \\
		- TeachingDay & String & Ngày dạy trong tuần (ví dụ: "2-4-6"). \\
		- Method & String & Hình thức dạy (Online/Offline). \\
		\hline
	\end{tabularx}
\end{table}

\textbf{Phương thức (Methods):}
\begin{table}[H]
	\centering
	\begin{tabularx}{\textwidth}{|l|X|l|X|}
		\hline
		\textbf{Tên phương thức} & \textbf{Tham số} & \textbf{Kiểu trả về} & \textbf{Mô tả (Liên kết với Use Case)} \\
		\hline
		+ openClass() & input: ClassInput & Class & Tutor mở một lớp học mới (UC-08). \\
		+ updateClass() & input: ClassUpdateInput & Class & Cập nhật thông tin lớp học. \\
		+ deleteClass() & classId: UUID & Boolean & Xóa một lớp học. \\
		+ getClassesByTutor() & tutorId: UUID & List<Class> & Lấy danh sách các lớp do Tutor phụ trách. \\
		\hline
	\end{tabularx}
\end{table}

\subsection*{4.4.4. Cụm Đánh giá và Hỗ trợ (Feedback \& Support)}
\addcontentsline{toc}{subsection}{4.4.4. Cụm Đánh giá và Hỗ trợ (Feedback \& Support)}

Các lớp hỗ trợ (Feedback, Material, Attendance) được thiết kế để mở rộng tính năng cho lớp học.

\textit{Lưu ý: Các lớp này đã được thiết kế trong Database Schema nhưng chưa được implement đầy đủ trong phiên bản hiện tại.}

\textbf{Mô tả:} Lớp \texttt{Feedback} quản lý đánh giá giữa Sinh viên và Tutor.

\textbf{Thuộc tính (Attributes):}
\begin{table}[H]
	\centering
	\begin{tabularx}{\textwidth}{|l|l|X|}
		\hline
		\textbf{Tên thuộc tính} & \textbf{Kiểu dữ liệu} & \textbf{Mô tả} \\
		\hline
		- FeedbackID & UUID & Khóa chính. \\
		- Rating & Integer & Điểm đánh giá (1-5 sao). \\
		- Comment & String & Nội dung nhận xét. \\
		- CreatedAt & DateTime & Thời gian tạo. \\
		\hline
	\end{tabularx}
\end{table}

\textbf{Mô tả:} Lớp \texttt{Material} quản lý tài liệu học tập.

\textbf{Thuộc tính (Attributes):}
\begin{table}[H]
	\centering
	\begin{tabularx}{\textwidth}{|l|l|X|}
		\hline
		\textbf{Tên thuộc tính} & \textbf{Kiểu dữ liệu} & \textbf{Mô tả} \\
		\hline
		- MaterialID & UUID & Khóa chính. \\
		- FileName & String & Tên file. \\
		- FilePath & String & Đường dẫn lưu trữ file. \\
		- UploadedAt & DateTime & Thời gian tải lên. \\
		\hline
	\end{tabularx}
\end{table}