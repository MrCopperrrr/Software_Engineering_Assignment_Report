\part*{4. Triển khai hệ thống}
\addcontentsline{toc}{part}{4. Triển khai hệ thống}

%========================================================================================
\section*{4.1. Sơ đồ triển khai (Deployment View)}
\addcontentsline{toc}{section}{4.1. Sơ đồ triển khai (Deployment View)}
\subsection*{4.1.1 Giới thiệu}
\addcontentsline{toc}{subsection}{4.1.1 Giới thiệu}
Sơ đồ triển khai (Deployment Diagram) mô tả kiến trúc vật lý của "Hệ thống hỗ trợ Tutor", thể hiện cách các thành phần phần mềm được phân bổ và vận hành trên các nút (node) phần cứng. Sơ đồ này cung cấp một cái nhìn tổng quan về môi trường thực thi của hệ thống, bao gồm các máy chủ, cơ sở dữ liệu, và sự tương tác giữa chúng cũng như với các hệ thống bên ngoài.
Kiến trúc được lựa chọn là mô hình \textbf{client-server ba lớp (3-tier)} hiện đại, bao gồm:

\begin{itemize}
    \item \textbf{Presentation Tier (Client)}: Giao diện người dùng trên trình duyệt web.
    \item \textbf{Application Tier (Server)}: Máy chủ ứng dụng xử lý logic nghiệp vụ.
    \item \textbf{Data Tier (Database)}: Máy chủ cơ sở dữ liệu để lưu trữ và quản lý dữ liệu.
\end{itemize}

Mô hình này đảm bảo tính linh hoạt, khả năng mở rộng và bảo mật cho hệ thống.

\subsection*{4.1.2. Sơ đồ triển khai hệ thống}
\addcontentsline{toc}{subsection}{4.1.2. Sơ đồ triển khai hệ thống}
Sơ đồ dưới đây được tạo bằng PlantUML, mô tả chi tiết các thành phần và luồng tương tác.
\begin{figure}[H]
    \centering
    \setlength{\fboxsep}{2pt}     
    \setlength{\fboxrule}{0.5pt}   
    \fbox{\includegraphics[scale=0.3]{Picture/Deploymentview.png}}
    \caption{Sơ đồ triển khai hệ thống hỗ trợ Tutor}
\end{figure}

\subsection*{4.1.3. Mô tả các thành phần}
\addcontentsline{toc}{subsection}{4.1.3. Mô tả các thành phần}

\begin{table}[H]
\centering
\begin{tabular}{|c|p{3.5cm}|p{6cm}|p{3.5cm}|}
\hline
\textbf{STT} & \textbf{Tên thành phần} & \textbf{Mô tả} & \textbf{Công nghệ/Phần mềm} \\
\hline
1 & Người dùng & Các tác nhân (Sinh viên, Tutor, Admin) tương tác với hệ thống thông qua trình duyệt web & Trình duyệt Web (Chrome, Firefox, Safari) \\
\hline
2 & Web Server (Nginx) & Đóng vai trò là Reverse Proxy, tiếp nhận yêu cầu từ Internet và chuyển tiếp đến Application Server. Tăng cường bảo mật và cân bằng tải. & Nginx \\
\hline
3 & Application Server & Chứa toàn bộ logic nghiệp vụ của hệ thống, được đóng gói trong Docker container để dễ dàng triển khai và quản lý. & Java Spring Boot / Node.js, Docker \\
\hline
4 & Database Server & Chịu trách nhiệm lưu trữ và quản lý toàn bộ dữ liệu của hệ thống, được triển khai trong Docker container. & PostgreSQL / MySQL, Docker \\
\hline
5 & Hệ thống của HCMUT & Các dịch vụ công nghệ thông tin tập trung của trường mà hệ thống cần tích hợp, bao gồm: HCMUT\_SSO , HCMUT\_DATACORE, HCMUT\_LIBRARY . & API (REST/SOAP) \\
\hline
6 & Email Service & Dịch vụ bên ngoài chịu trách nhiệm gửi các thông báo và nhắc nhở qua email cho người dùng. & SMTP Server (ví dụ: SendGrid, AWS SES) \\
\hline
\end{tabular}
\caption{Các thành phần trong sơ đồ triển khai hệ thống hỗ trợ Tutor}
\end{table}

\subsection*{4.1.4. Luồng hoạt động}
\addcontentsline{toc}{subsection}{4.1.4. Luồng hoạt động}
Để làm rõ hơn sự tương tác giữa các thành phần, dưới đây là mô tả luồng hoạt động của hai kịch bản tiêu biểu:

\begin{enumerate}[label=\Alph*.]
    \item \textbf{Kịch bản 1: Người dùng đăng nhập vào hệ thống}
    \begin{enumerate}[label*=\arabic*.]
        \item Người dùng mở trình duyệt, truy cập vào địa chỉ của hệ thống và nhấn nút đăng nhập thông qua \textbf{HCMUT\_SSO}.
        \item Yêu cầu (\textit{HTTPS Request}) được gửi qua Internet đến \textbf{Web Server (Nginx)}.
        \item \textbf{Nginx} giải mã SSL và chuyển tiếp yêu cầu đến \textbf{Application Server}.
        \item \textbf{Application Server} nhận yêu cầu, chuyển hướng người dùng đến trang đăng nhập của \textbf{HCMUT\_SSO}.
        \item Sau khi người dùng đăng nhập thành công trên \textbf{HCMUT\_SSO}, dịch vụ này sẽ trả về một \textit{token} xác thực cho \textbf{Application Server}.
        \item \textbf{Application Server} sử dụng \textit{token} này để gọi API đến \textbf{HCMUT\_DATACORE}, lấy và đồng bộ thông tin cơ bản của người dùng (họ tên, MSSV, khoa, vai trò) vào \textbf{Database Server}.
        \item Cuối cùng, \textbf{Application Server} trả về một phiên làm việc (\textit{session}) và giao diện trang chủ cho người dùng.
    \end{enumerate}

    \item \textbf{Kịch bản 2: Tutor tạo lịch rảnh mới}
    \begin{enumerate}[label*=\arabic*.]
        \item \textbf{Tutor} sau khi đăng nhập, truy cập chức năng ``Tạo lịch rảnh'' và điền thông tin (ngày, giờ).
        \item Yêu cầu tạo lịch được gửi qua Internet, đi qua \textbf{Web Server} và đến \textbf{Application Server}.
        \item \textbf{Application Server} xác thực quyền của Tutor, kiểm tra tính hợp lệ của dữ liệu (ví dụ: không trùng lịch đã có).
        \item Nếu hợp lệ, \textbf{Application Server} sẽ thực hiện một câu lệnh ghi (\texttt{INSERT}) để lưu thông tin lịch rảnh mới vào bảng \texttt{Schedules} trong \textbf{Database Server}.
        \item \textbf{Database Server} xác nhận ghi thành công.
        \item \textbf{Application Server} gửi lại một thông báo ``Tạo lịch thành công'' cho giao diện của Tutor.
    \end{enumerate}

    \item \textbf{Kịch bản 3: Sinh viên đặt lịch học cố định với Tutor}
    \begin{enumerate}[label*=\arabic*.]
        \item Sinh viên, sau khi đăng nhập, chọn môn học và xem các khung giờ rảnh (\textit{availability}) mà Tutor đã thiết lập. Sinh viên chọn một khung giờ phù hợp và nhấn ``Đặt lịch''.
        \item Yêu cầu đặt lịch (\textit{HTTPS Request}) được gửi qua Internet tới \textbf{Web Server (Nginx)}.
        \item \textbf{Nginx} chuyển tiếp yêu cầu đến \textbf{Application Server}.
        \item \textbf{Application Server} nhận yêu cầu và thực hiện các bước xác thực nghiệp vụ:
        \begin{itemize}
            \item \textbf{Bước 4a}: Truy vấn \textbf{Database Server} để kiểm tra lại xem khung giờ mà sinh viên chọn có còn trống không và có nằm trong lịch rảnh hợp lệ của Tutor không.
            \item \textbf{Bước 4b}: Truy vấn \textbf{Database Server} để kiểm tra lịch trình của chính sinh viên đó, đảm bảo không bị trùng với các lịch học/lịch thi khác đã đăng ký.
        \end{itemize}
        \item Nếu tất cả các điều kiện đều hợp lệ, \textbf{Application Server} sẽ tạo một bản ghi lịch học cố định mới trong \textbf{Database Server}, liên kết sinh viên, Tutor, và môn học với khung giờ đã chọn. Trạng thái của lịch được đặt là ``Đã xác nhận'' (\textit{Confirmed}).
        \item Sau khi \textbf{Database Server} xác nhận lưu thành công, \textbf{Application Server} thực hiện hai hành động song song:
        \begin{itemize}
            \item \textbf{Bước 6a}: Gửi một phản hồi thành công về cho trình duyệt của sinh viên, hiển thị thông báo ``Bạn đã đặt lịch thành công''.
            \item \textbf{Bước 6b}: Kết nối đến \textbf{Email Service (SMTP Server)} để gửi email thông báo xác nhận lịch học cho cả sinh viên và Tutor.
        \end{itemize}
    \end{enumerate}
\end{enumerate}
%========================================================================================
\section*{4.2. Sơ đồ phát triển (Development View)}
\addcontentsline{toc}{section}{4.2. Sơ đồ phát triển (Development View)}
\subsection*{4.2.1 Giới thiệu}
\addcontentsline{toc}{subsection}{4.2.1 Giới thiệu}
Sơ đồ phát triển, hay \textbf{Sơ đồ gói (Package Diagram)}, mô tả cấu trúc tĩnh và cách tổ chức mã nguồn của hệ thống từ góc nhìn của đội ngũ phát triển. Mục tiêu của sơ đồ này là trình bày một kiến trúc phần mềm có tính module hóa cao, rõ ràng và dễ bảo trì, tuân thủ nguyên tắc \textit{tách biệt các mối quan tâm (Separation of Concerns)}.

\noindent Hệ thống được thiết kế theo kiến trúc \textbf{phân lớp (Layered Architecture)}, một mô hình phổ biến trong các ứng dụng web hiện đại. Các lớp chính bao gồm:

\begin{itemize}
    \item \textbf{Controller Layer}: Tiếp nhận và xử lý các yêu cầu HTTP từ người dùng.
    \item \textbf{Service Layer}: Chứa đựng toàn bộ logic nghiệp vụ cốt lõi của hệ thống.
    \item \textbf{Repository Layer}: Chịu trách nhiệm truy cập và tương tác với cơ sở dữ liệu.
    \item \textbf{Domain/Model Layer}: Định nghĩa các đối tượng và thực thể dữ liệu.
\end{itemize}

Ngoài ra, hệ thống còn có các gói hỗ trợ cho các chức năng xuyên suốt như:
\begin{itemize}
    \item \textbf{Security}: Quản lý xác thực, phân quyền và bảo mật truy cập.
    \item \textbf{Configuration}: Cấu hình hệ thống, môi trường và thông số hoạt động.
    \item \textbf{Integration}: Tích hợp với các hệ thống bên ngoài như HCMUT\_SSO, HCMUT\_DATACORE, Email Service,...
\end{itemize}

\subsection*{4.2.2. Sơ đồ gói (Package Diagram)}
\addcontentsline{toc}{subsection}{4.2.2. Sơ đồ gói (Package Diagram)}
Sơ đồ dưới đây minh họa cách mã nguồn được tổ chức thành các gói logic và mối quan hệ phụ thuộc giữa chúng.
\begin{figure}[H]
    \centering
    \setlength{\fboxsep}{2pt}     
    \setlength{\fboxrule}{0.5pt}   
    \fbox{\includegraphics[scale=0.45]{Picture/Developmentview.png}}
    \caption{Sơ đồ tổ chức các gói của hệ thống}
\end{figure}

\subsection*{4.2.3. Mô tả các gói}
\addcontentsline{toc}{subsection}{4.2.3. Mô tả các gói}
\begin{table}[H]
\centering
\caption{Chức năng của từng gói (package) trong kiến trúc phần mềm}
\begin{tabular}{|c|p{4cm}|p{9cm}|}
\hline
\textbf{STT} & \textbf{Tên gói} & \textbf{Chức năng chính} \\
\hline
1 & Controller Layer & Chứa các lớp chịu trách nhiệm tiếp nhận yêu cầu HTTP từ client, gọi đến các Service tương ứng để xử lý và trả về phản hồi (response). \\
\hline
2 & Service Layer & Chứa toàn bộ logic nghiệp vụ của ứng dụng. Đây là nơi các quy trình như ghép cặp Tutor, xử lý đăng ký, tạo báo cáo được thực thi. Gói này điều phối hoạt động giữa Repositories và các thành phần khác. \\
\hline
3 & Repository Layer & Chứa các interface/lớp định nghĩa các phương thức để truy cập dữ liệu. Gói này trừu tượng hóa lớp truy cập dữ liệu, giúp Service không cần biết chi tiết về cách dữ liệu được lưu trữ. \\
\hline
4 & Domain (Entities) & Chứa các lớp thực thể (Entity) ánh xạ trực tiếp tới các bảng trong cơ sở dữ liệu, định nghĩa cấu trúc dữ liệu cốt lõi của hệ thống. \\
\hline
5 & DTOs (Data Transfer Objects) & Chứa các lớp dùng để truyền dữ liệu giữa các lớp, đặc biệt là giữa Controller và Service. Việc sử dụng DTO giúp che giấu cấu trúc của Domain và chỉ truyền đi những dữ liệu cần thiết. \\
\hline
-- & Các gói phụ trợ & Bao gồm Security, Integration, Config. Các gói này cung cấp các chức năng xuyên suốt và được sử dụng chủ yếu bởi Service Layer nhưng không được vẽ ra để giữ cho sơ đồ đơn giản. \\
\hline
\end{tabular}
\end{table}

\subsection*{4.2.4. Luồng xử lý dữ liệu qua các lớp}
\addcontentsline{toc}{subsection}{4.2.4. Luồng xử lý dữ liệu qua các lớp}

Để minh họa cách các gói tương tác với nhau, luồng xử lý cho chức năng ``Sinh viên đăng ký môn học'' (UC-04):

\begin{itemize}
    \item \textbf{Request}: Sinh viên gửi yêu cầu đăng ký môn học từ giao diện người dùng. Yêu cầu này được gửi đến một endpoint trong Controller Layer.
    \item \textbf{Controller Layer}: Lớp \texttt{RegistrationController} nhận yêu cầu. Nó xác thực dữ liệu đầu vào (đóng gói trong một đối tượng \texttt{RegistrationRequestDto} từ gói DTOs) và gọi phương thức trong Service Layer.
    \item \textbf{Service Layer}: Lớp \texttt{RegistrationService} thực thi logic nghiệp vụ. Nó sử dụng các lớp trong Repository Layer để kiểm tra các quy tắc nghiệp vụ (ví dụ: giới hạn số môn đăng ký).
    \item \textbf{Repository Layer}: Các lớp như \texttt{UserRepository} và \texttt{RegistrationRepository} tương tác với cơ sở dữ liệu. Chúng làm việc với các đối tượng từ Domain Layer (ví dụ: \texttt{User}, \texttt{CourseRegistration}).
    \item \textbf{Response}: Sau khi xử lý xong, Service Layer trả về kết quả (thường dưới dạng một DTO khác) cho Controller Layer, và Controller sẽ tạo phản hồi HTTP gửi về cho client.
\end{itemize}
%========================================================================================
