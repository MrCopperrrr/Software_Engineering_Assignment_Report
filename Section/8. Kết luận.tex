\part*{8. Kết luận}
\addcontentsline{toc}{part}{8. Kết luận}
Dự án "Hệ thống hỗ trợ Tutor tại Trường Đại học Bách khoa – ĐHQG TP.HCM" đã được thực hiện nhằm mục tiêu xây dựng một bản thiết kế phần mềm toàn diện, đáp ứng nhu cầu hiện đại hóa và nâng cao hiệu quả cho chương trình Tutor/Mentor của nhà trường. Trải qua các giai đoạn từ phân tích yêu cầu, mô hình hóa hệ thống, đến thiết kế kiến trúc và thiết kế chi tiết, nhóm đã hoàn thành các mục tiêu chính đã đề ra.
%========================================================================================
\vspace{0.25cm}
\noindent Báo cáo này đã trình bày một cách hệ thống toàn bộ quá trình làm việc của nhóm. Khởi đầu bằng việc phân tích bối cảnh, xác định các tác nhân và yêu cầu chức năng/phi chức năng, nhóm đã xây dựng nên các mô hình hóa nghiệp vụ thông qua sơ đồ Use Case, sơ đồ hoạt động và sơ đồ tuần tự. Dựa trên nền tảng phân tích đó, nhóm đã tiến hành thiết kế một kiến trúc phần mềm vững chắc với Sơ đồ triển khai theo mô hình ba lớp và Sơ đồ phát triển theo kiến trúc phân lớp, đảm bảo hệ thống có khả năng mở rộng, bảo trì và tích hợp tốt trong tương lai. Cuối cùng, một Sơ đồ lớp chi tiết cùng với mô tả đầy đủ các thuộc tính và phương thức đã được hoàn thiện, đóng vai trò là một bản thiết kế chi tiết sẵn sàng cho giai đoạn triển khai.
\section*{8.1. Kết quả đạt được}
\addcontentsline{toc}{section}{8.1. Kết quả đạt được}
Kết quả quan trọng nhất mà nhóm đã đạt được là một bộ tài liệu thiết kế phần mềm toàn diện và chi tiết. Đây là một nền tảng vững chắc, bao gồm:
\begin{itemize}
    \item Mô hình hóa rõ ràng: Các chức năng của hệ thống được trực quan hóa chi tiết, giúp mọi thành viên và các bên liên quan có cùng một cái nhìn thống nhất về hệ thống.
    \item Kiến trúc hiện đại: Việc lựa chọn kiến trúc phân lớp và các công nghệ phổ biến như Docker đảm bảo hệ thống sẽ hoạt động ổn định và dễ dàng nâng cấp.
    \item Thiết kế hướng đối tượng chi tiết: Sơ đồ lớp được xây dựng cẩn thận, đảm bảo tính đóng gói, kế thừa và đa hình, giúp cho việc lập trình trong giai đoạn tiếp theo trở nên thuận lợi và có hệ thống.
\end{itemize}

\section*{8.2. Bài học kinh nghiệm}
\addcontentsline{toc}{section}{8.2. Bài học kinh nghiệm}
Trong suốt quá trình thực hiện dự án, nhóm đã rút ra được nhiều bài học quý báu:
\begin{itemize}
    \item Tầm quan trọng của giai đoạn phân tích: Việc đầu tư thời gian kỹ lưỡng để phân tích yêu cầu và định nghĩa rõ ràng các Use Case ở giai đoạn đầu đã giúp cho quá trình thiết kế ở giai đoạn sau trở nên suôn sẻ và ít phải sửa đổi hơn rất nhiều.
    \item Giá trị của UML: Các sơ đồ UML không chỉ là công cụ để vẽ, mà còn là một ngôn ngữ giao tiếp hiệu quả, giúp các thành viên trong nhóm thống nhất ý tưởng và phát hiện các mâu thuẫn trong logic từ sớm.
\end{itemize}


\section*{8.3. Hướng phát triển trong tương lai}
\addcontentsline{toc}{section}{8.3. Hướng phát triển trong tương lai}
Bản thiết kế hiện tại đã hoàn thành các chức năng cốt lõi. Tuy nhiên, để hệ thống trở nên thông minh và hữu ích hơn, nhóm đề xuất các hướng phát triển nâng cao trong tương lai, như đã được gợi ý trong đề bài:
\begin{itemize}
    \item Tích hợp AI để ghép cặp thông minh: Xây dựng module sử dụng Trí tuệ nhân tạo để phân tích hồ sơ sinh viên và Tutor, từ đó đưa ra gợi ý ghép cặp tối ưu nhất dựa trên nhiều tiêu chí (kết quả học tập, tính cách, lịch trình...).
    \item Xây dựng cộng đồng trực tuyến: Phát triển các tính năng mạng xã hội như diễn đàn, nhóm trò chuyện (chat groups) để Tutor và sinh viên có thể tương tác, trao đổi kiến thức ngoài các buổi học chính thức.
    \item Hỗ trợ học tập cá nhân hóa (AI Integration): Áp dụng AI để phân tích quá trình học tập của sinh viên, từ đó tự động đề xuất lộ trình học tập riêng, gợi ý các tài liệu phù hợp để cải thiện điểm yếu.
\end{itemize}


