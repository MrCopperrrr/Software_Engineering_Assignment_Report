\part*{1. Tổng quan dự án}
\addcontentsline{toc}{part}{1. Tổng quan dự án}
Dự án "Hệ thống hỗ trợ Tutor tại Trường Đại học Bách khoa – ĐHQG TP.HCM" là một sáng kiến công nghệ nhằm mục tiêu hiện đại hóa và nâng cao hiệu quả của chương trình Tutor/Mentor. Đây là một chương trình có ý nghĩa quan trọng, được nhà trường triển khai nhằm hỗ trợ sinh viên phát triển một cách toàn diện cả về tri thức học thuật lẫn các kỹ năng cần thiết. Báo cáo này sẽ trình bày chi tiết về quá trình, từ việc phân tích bối cảnh, xác định yêu cầu, đến thiết kế và xây dựng hệ thống.

%========================================================================================

\section*{1.1. Giới thiệu dự án}
\addcontentsline{toc}{section}{1.1. Giới thiệu dự án}
Trong bối cảnh giáo dục đại học đang không ngừng đổi mới, việc ứng dụng công nghệ để tối ưu hóa các hoạt động hỗ trợ sinh viên là một yêu cầu tất yếu. Dự án này ra đời nhằm xây dựng một nền tảng phần mềm chuyên biệt, đóng vai trò xương sống cho chương trình Tutor, qua đó tạo một môi trường học tập tương tác, hiệu quả và có hệ thống tại Trường Đại học Bách khoa TP.HCM.

%========================================================================================

\subsection*{1.1.1. Mục đích}
\addcontentsline{toc}{subsection}{1.1.1. Mục đích}

Mục đích chính của dự án là phát triển một phần mềm quản lý tập trung, giúp vận hành chương trình Tutor một cách hiệu quả, hiện đại và có khả năng mở rộng. Hệ thống này sẽ là cầu nối vững chắc giữa Tutor (giảng viên, nghiên cứu sinh, sinh viên năm trên có thành tích tốt) và sinh viên cần hỗ trợ. Thông qua đó, dự án hướng đến việc nâng cao chất lượng học tập, tăng cường sự tương tác và gắn kết trong cộng đồng sinh viên, đồng thời góp phần vào việc phát triển kỹ năng mềm và định hướng nghề nghiệp cho người học.

%========================================================================================

\subsection*{1.1.2. Bối cảnh và lý do cần hệ thống}
\addcontentsline{toc}{subsection}{1.1.2. Bối cảnh và lý do cần hệ thống}
Trên thực tế, chương trình Tutor/Mentor tại Trường Đại học Bách khoa TP.HCM đã được triển khai và mang lại những lợi ích nhất định. Tuy nhiên, quy trình vận hành hiện tại vẫn còn phụ thuộc nhiều vào các phương pháp thủ công, dẫn đến một số thách thức và hạn chế đáng kể:
\begin{itemize}
    \item \textbf{Về quản lý thông tin và kết nối:} Việc quản lý hồ sơ năng lực của Tutor và nhu cầu cụ thể của sinh viên còn khó khăn, khiến quá trình ghép cặp chưa đạt được hiệu quả tối ưu. Sinh viên thường gặp khó khăn trong việc chủ động tìm kiếm và kết nối với người hướng dẫn phù hợp nhất.
    \item \textbf{Về tổ chức và sắp xếp:} Công tác lên lịch, thay đổi hoặc hủy các buổi học phụ thuộc nhiều vào việc trao đổi cá nhân, tiềm ẩn nguy cơ nhầm lẫn, thiếu sót và tốn nhiều thời gian không cần thiết.
    \item \textbf{Về đo lường và cải tiến:} Việc thiếu một công cụ theo dõi và đánh giá bài bản đã tạo ra một khoảng trống trong việc đo lường tiến bộ của sinh viên cũng như chất lượng của các buổi học, gây khó khăn cho việc cải tiến và nâng cao hiệu quả chương trình.
\end{itemize}
Trước những bất cập đó, việc xây dựng một "Hệ thống hỗ trợ Tutor" là một giải pháp cấp thiết. Bằng cách tự động hóa các quy trình từ quản lý, ghép cặp, lên lịch cho đến đánh giá, hệ thống được kỳ vọng sẽ giải quyết hiệu quả những tồn tại này, đáp ứng các đòi hỏi thực tiễn của môi trường giáo dục đại học trong kỷ nguyên số.

%========================================================================================

\subsection*{1.1.3. Kỳ vọng và mục tiêu}
\addcontentsline{toc}{subsection}{1.1.3. Kỳ vọng và mục tiêu}
Dự án được định hướng bởi những kỳ vọng và mục tiêu rõ ràng, hướng đến lợi ích của các bên liên quan: \\
\textbf{Kỳ vọng:}
\begin{itemize}
    \item \textbf{Đối với Sinh viên:}  Hệ thống được kỳ vọng sẽ trở thành một cổng thông tin thân thiện, giúp sinh viên dễ dàng tìm kiếm sự hỗ trợ học thuật, chủ động lựa chọn Tutor, linh hoạt sắp xếp lịch học và nhận được sự giúp đỡ kịp thời, đúng nhu cầu.
    \item \textbf{Đối với Tutor:} Cung cấp một bộ công cụ số hóa mạnh mẽ để quản lý thông tin cá nhân, sắp xếp lịch làm việc một cách khoa học, theo dõi và ghi nhận sự tiến bộ của sinh viên
    \item \textbf{Đối với Nhà trường:} Trao cho các Khoa và Phòng ban một công cụ quản lý tổng thể, cho phép giám sát, phân tích và đánh giá hiệu quả của chương trình. Dữ liệu thu thập được sẽ là cơ sở thực tiễn để tối ưu hóa việc phân bổ nguồn lực và đưa ra các quyết sách quan trọng, chẳng hạn như xét điểm rèn luyện hoặc học bổng.
\end{itemize}
\textbf{Mục tiêu:}
\begin{itemize}
    \item Phát triển một nền tảng phần mềm hoàn chỉnh, bao quát các chức năng cốt lõi: quản lý hồ sơ, đăng ký, ghép cặp, lên lịch, thông báo và đánh giá.
    \item Tích hợp liền mạch và an toàn với hạ tầng công nghệ thông tin hiện có của trường, bao gồm dịch vụ xác thực tập trung (HCMUT\_SSO), cơ sở dữ liệu lõi (HCMUT\_DATACORE) và thư viện số (HCMUT\_LIBRARY).
    \item Đảm bảo giao diện người dùng (UI/UX) thân thiện, trực quan, dễ sử dụng trên nhiều nền tảng, đồng thời thiết kế kiến trúc hệ thống theo hướng mở, sẵn sàng cho việc mở rộng và tích hợp các tính năng nâng cao trong tương lai.
\end{itemize}

%========================================================================================

\subsection*{1.1.4. Sản phẩm bàn giao}
\addcontentsline{toc}{subsection}{1.1.4. Sản phẩm bàn giao}
Kết thúc dự án, nhóm sẽ bàn giao các sản phẩm sau:
\begin{itemize}
    \item \textbf{Báo cáo phân tích yêu cầu phần mềm:} Tài liệu mô tả chi tiết các yêu cầu chức năng và phi chức năng của hệ thống, cùng với các biểu đồ Use-case.
    \item \textbf{Tài liệu thiết kế hệ thống:}  Bao gồm thiết kế kiến trúc tổng quan, thiết kế chi tiết các module, thiết kế giao diện người dùng (UI), và các biểu đồ liên quan như biểu đồ tuần tự, biểu đồ lớp.
    \item \textbf{Mã nguồn của ứng dụng (MVP - Minimum Viable Product):} Một phiên bản phần mềm có thể hoạt động được, bao gồm các chức năng cốt lõi đã được thống nhất.
    \item \textbf{Tài liệu hướng dẫn sử dụng:} Hướng dẫn chi tiết cho các đối tượng người dùng khác nhau như sinh viên, Tutor và quản trị viên.
    \item \textbf{Slide thuyết trình và video demo sản phẩm:} Trình bày tổng quan về dự án, các chức năng chính của hệ thống và demo cách thức hoạt động.
    \item \textbf{Báo cáo cuối kỳ:} Tổng hợp toàn bộ quá trình thực hiện dự án, từ việc phân tích yêu cầu, thiết kế, triển khai cho đến kết quả đạt được và những bài học kinh nghiệm.
\end{itemize}

%========================================================================================

\section*{1.2. Phạm vi dự án}
\addcontentsline{toc}{section}{1.2. Phạm vi dự án}
Để đảm bảo tính khả thi và sự tập trung của dự án, việc xác định rõ ràng ranh giới là vô cùng quan trọng. Phần này sẽ trình bày cụ thể các chức năng sẽ được xây dựng, những hạng mục nằm ngoài khuôn khổ, cùng các ràng buộc về kỹ thuật và nghiệp vụ mà hệ thống phải tuân thủ.

%========================================================================================

%========================================================================================

\subsection*{1.2.1. Trong phạm vi}
\addcontentsline{toc}{subsection}{1.2.1. Trong phạm vi}
Phạm vi của dự án được xác định rõ ràng, tập trung vào việc phát triển một bộ chức năng cốt lõi, đủ mạnh để quản lý và vận hành chương trình Tutor một cách toàn diện trên nền tảng độc lập. Các nhóm chức năng chính bao gồm:
\begin{itemize}
	\item \textbf{Quản lý Tài khoản và Xác thực:} Xây dựng hệ thống định danh người dùng độc lập, bao gồm đăng ký, đăng nhập và xác thực bảo mật 2 lớp (2FA) thông qua Email OTP.
	\item \textbf{Quản lý Dữ liệu Đào tạo (Giả lập):} Xây dựng cơ sở dữ liệu nội bộ lưu trữ thông tin Môn học, Sinh viên và Tutor theo chuẩn dữ liệu thực tế để phục vụ quy trình nghiệp vụ.
	\item \textbf{Module Đăng ký và Ghép cặp:}
	\begin{itemize}
		\item \textbf{Đăng ký Môn học:} Sinh viên đăng ký nhu cầu học tập dựa trên danh mục môn học có sẵn trong hệ thống.
		\item \textbf{Ghép cặp Thủ công:} Hệ thống cung cấp công cụ tìm kiếm và lọc Tutor dựa trên chuyên môn và lịch mở lớp (Class Schedule), giúp sinh viên chủ động chọn người hướng dẫn.
	\end{itemize}
	\item \textbf{Quản lý Lớp học (Class Management):} Cho phép Tutor chủ động "mở lớp" (Open Class) với các thông tin chi tiết về thời gian, hình thức (Online/Offline) và địa điểm. Sinh viên đăng ký trực tiếp vào các lớp học này.
	\item \textbf{Cơ chế Đánh giá và Phản hồi:} Thu thập và lưu trữ đánh giá hai chiều giữa Sinh viên và Tutor sau khi kết thúc khóa học/buổi học để xây dựng hồ sơ năng lực uy tín.
\end{itemize}

%========================================================================================

\subsection*{1.2.2. Ngoài phạm vi}
\addcontentsline{toc}{subsection}{1.2.2. Ngoài phạm vi}
Để đảm bảo tiến độ và tính khả thi trong khuôn khổ đồ án, các hạng mục sau được xác định nằm ngoài phạm vi phiên bản hiện tại (MVP):
\begin{itemize}
	\item \textbf{Tích hợp trực tiếp hệ thống Nhà trường:} Việc đấu nối API thực tế với HCMUT\_SSO hay HCMUT\_DATACORE không được thực hiện do quy định bảo mật.
	\item \textbf{Lưu trữ tài liệu vật lý:} Hệ thống quản lý thông tin (metadata) của tài liệu chia sẻ, nhưng không xây dựng server lưu trữ file vật lý (File Storage Server).
	\item \textbf{Thuật toán Ghép cặp AI:} Các tính năng gợi ý thông minh dựa trên học máy (Machine Learning) chưa được triển khai.
	\item \textbf{Tính năng Mạng xã hội:} Chat trực tuyến, diễn đàn thảo luận chưa được tích hợp.
\end{itemize}

%========================================================================================

\subsection*{1.2.3. Ràng buộc hệ thống}
\addcontentsline{toc}{subsection}{1.2.3. Ràng buộc hệ thống}
Quá trình thiết kế và phát triển hệ thống phải tuân thủ nghiêm ngặt các ràng buộc quan trọng sau đây để đảm bảo tính khả thi và phù hợp với môi trường triển khai thực tế:

\begin{itemize}
	\item \textbf{Kiến trúc Hệ thống Độc lập (Standalone Architecture):} Do các quy định về bảo mật và hạn chế quyền truy cập vào hạ tầng công nghệ thực tế của nhà trường, hệ thống được thiết kế để vận hành hoàn toàn độc lập, không phụ thuộc vào các dịch vụ bên thứ ba của trường. Cụ thể:
	\begin{itemize}
		\item \textbf{Cơ chế Xác thực tự chủ:} Thay vì sử dụng HCMUT\_SSO, hệ thống tự xây dựng module xác thực bảo mật sử dụng JWT (JSON Web Token) để quản lý phiên làm việc. Đồng thời, tích hợp dịch vụ SendGrid để thực hiện xác thực hai yếu tố qua Email (OTP) nhằm đảm bảo danh tính người dùng.
		\item \textbf{Quản lý Dữ liệu nội bộ:} Hệ thống không đồng bộ dữ liệu từ HCMUT\_DATACORE. Thay vào đó, nhóm phát triển xây dựng cơ sở dữ liệu riêng (MySQL) và sử dụng dữ liệu giả lập (Mock Data) chuẩn hóa theo format thực tế để mô phỏng các nghiệp vụ đăng ký và quản lý môn học.
	\end{itemize}
	
	\item \textbf{Cơ chế Phân quyền và Bảo mật:} Hệ thống phải tự định nghĩa và quản lý các vai trò người dùng (Sinh viên, Tutor). Việc phân quyền (Authorization) được thực hiện chặt chẽ tại tầng Backend thông qua các GraphQL Resolvers, đảm bảo người dùng chỉ truy cập được các tài nguyên cho phép dựa trên thông tin được mã hóa trong Token.
	
	\item \textbf{Yêu cầu về Cơ sở dữ liệu và Giao thức:}
	\begin{itemize}
		\item Hệ thống sử dụng MySQL làm hệ quản trị cơ sở dữ liệu quan hệ để đảm bảo tính toàn vẹn dữ liệu giữa các thực thể phức tạp (User, Course, Class).
		\item Sử dụng giao thức GraphQL (Apollo Server) thay vì REST API truyền thống để tối ưu hóa việc truy vấn dữ liệu từ phía Client.
	\end{itemize}
	
\end{itemize}

%========================================================================================

\section*{1.3. Tài liệu tham khảo}
\addcontentsline{toc}{section}{1.3. Tài liệu tham khảo liên quan}
Để đảm bảo hệ thống được xây dựng không chỉ đáp ứng yêu cầu mà còn hiệu quả và phù hợp với thực tiễn, nhóm đã tiến hành một quá trình nghiên cứu và tham chiếu (benchmarking) các nền tảng hỗ trợ học tập và quản lý lịch hẹn tương tự đang vận hành thành công trên thị trường. Việc phân tích này giúp chúng tôi đúc kết những bài học giá trị, từ đó định hình các tính năng cốt lõi cho hệ thống.

%========================================================================================

\subsection*{1.3.1. Các nền tảng tương tự}
\addcontentsline{toc}{subsection}{1.3.1. Các nền tảng tương tự}
Nhóm đã tập trung phân tích ba nhóm nền tảng chính, mỗi nhóm đại diện cho một khía cạnh quan trọng của hệ thống cần xây dựng:

\begin{itemize}
    \item \textbf{Nền tảng Gia sư Trực tuyến (Online Tutoring Platforms):} Các dịch vụ thương mại như Chegg, TutorMe, và Preply là những ví dụ tiêu biểu. Chúng nổi bật với hệ thống tìm kiếm và bộ lọc mạnh mẽ, cho phép người dùng tìm kiếm gia sư theo môn học, chuyên ngành, khung giờ và mức đánh giá. Đặc biệt, hồ sơ (profile) của gia sư được xây dựng rất chi tiết, tạo sự tin cậy thông qua việc trình bày kinh nghiệm, trình độ học vấn và các nhận xét xác thực từ học viên.
    \item \textbf{Hệ thống Quản lý Sinh viên Chuyên dụng:} Các giải pháp phần mềm như Navigate và Starfish được nhiều trường đại học trên thế giới tin dùng để hỗ trợ sự thành công của sinh viên. Điểm mạnh cốt lõi của chúng là khả năng tích hợp sâu rộng với cơ sở dữ liệu của nhà trường, cho phép theo dõi toàn diện tiến trình học tập. Các hệ thống này không chỉ đơn thuần hỗ trợ đặt lịch hẹn với cố vấn học tập mà còn có khả năng gửi cảnh báo sớm khi nhận diện các sinh viên có dấu hiệu sa sút.
    \item \textbf{Nền tảng Đặt lịch hẹn Chuyên dụng:} Các công cụ như Calendly và Doodle là minh chứng cho hiệu quả của sự đơn giản và tập trung. Chúng giải quyết triệt để bài toán sắp xếp lịch hẹn bằng cách loại bỏ hoàn toàn các bước trao đổi thủ công. Người dùng chỉ cần thiết lập các khung giờ khả dụng và chia sẻ một liên kết duy nhất; hệ thống sẽ tự động xử lý việc đặt lịch, kiểm tra trùng lặp và gửi thông báo xác nhận, nhắc nhở.

\end{itemize}

%========================================================================================

\subsection*{1.3.2. Bài học rút ra}
\addcontentsline{toc}{subsection}{1.3.2. Bài học rút ra}
Từ việc phân tích các giải pháp trên, nhóm đã đúc kết được những bài học kinh nghiệm sâu sắc, đóng vai trò kim chỉ nam cho quá trình thiết kế hệ thống:
\begin{itemize}
    \item \textbf{Trải nghiệm người dùng (UX) là yếu tố quyết định:} Quy trình từ tìm kiếm Tutor đến đặt lịch thành công phải được thiết kế tối giản, nhanh chóng và trực quan. Bất kỳ sự phức tạp nào trong luồng thao tác đều có thể làm giảm tỷ lệ sử dụng của sinh viên.
    \item \textbf{Hiệu quả của tìm kiếm và bộ lọc là giá trị cốt lõi:}  Chức năng quan trọng nhất đối với sinh viên là khả năng tìm được đúng người hướng dẫn mình cần. Do đó, hệ thống bắt buộc phải có các bộ lọc cơ bản và hiệu quả như khoa, môn học và lịch dạy.

    \item \textbf{Tự động hóa là chìa khóa của sự hiệu quả:}  Toàn bộ quy trình đặt, hủy và nhắc lịch cần được tự động hóa. Điều này không chỉ giúp tiết kiệm thời gian cho cả Tutor và sinh viên mà còn giảm thiểu đáng kể các sai sót do con người.

    \item \textbf{Sự tin cậy được vun đắp từ cơ chế đánh giá:}  Việc cho phép sinh viên để lại nhận xét và đánh giá sau mỗi buổi học là phương thức minh bạch và hiệu quả nhất để xây dựng một cộng đồng chất lượng, đồng thời cung cấp thông tin tham khảo giá trị cho những người dùng khác.

    \item \textbf{Quản lý tập trung tạo ra hiệu quả vận hành:}  Thay vì các quy trình rời rạc, một bảng điều khiển (dashboard) quản lý tập trung sẽ cung cấp cho các bên liên quan một cái nhìn tổng thể, giúp việc theo dõi và đánh giá hiệu quả chương trình trở nên dễ dàng và chính xác hơn.

\end{itemize}

%========================================================================================

\subsection*{1.3.3. Tính năng tổng hợp cho hệ thống}
\addcontentsline{toc}{subsection}{1.3.3. Tính năng tổng hợp cho hệ thống}
Dựa trên những bài học đúc kết và các yêu cầu đặc thù của Trường Đại học Bách khoa TP.HCM, nhóm đã tổng hợp và đề xuất các nhóm tính năng trọng tâm cho hệ thống như sau:
\begin{itemize}
    \item \textbf{Nền tảng Quản lý Người dùng và Hồ sơ chuyên nghiệp:}  Mỗi người dùng sẽ sở hữu một không gian hồ sơ riêng, nơi Tutor có thể trình bày một cách chuyên nghiệp về chuyên môn, kinh nghiệm và thành tích của mình.

    \item \textbf{Module Tìm kiếm và Gợi ý Ghép cặp thông minh:}  Cho phép sinh viên tìm kiếm Tutor theo các tiêu chí linh hoạt, đồng thời tích hợp chức năng gợi ý cơ bản dựa trên môn học và lịch trình tương thích.

    \item \textbf{Module Quản lý Lịch hẹn Tinh gọn:}  Lấy cảm hứng từ sự đơn giản của Calendly, module này cho phép Tutor thiết lập lịch dạy một cách trực quan, giúp sinh viên đặt lịch chỉ với vài thao tác. Toàn bộ quy trình thông báo xác nhận và nhắc nhở sẽ được tự động hóa.

    \item \textbf{Cơ chế Đánh giá và Phản hồi hai chiều:}  Sau mỗi buổi học, hệ thống sẽ chủ động mời sinh viên đánh giá, từ đó tạo ra một nguồn dữ liệu quý giá, tạo một vòng lặp cải tiến liên tục cho chất lượng chương trình.

    \item \textbf{Module Báo cáo và Thống kê Trực quan:}  Cung cấp các bảng điều khiển cho phép ban quản lý, các khoa và phòng ban dễ dàng theo dõi các chỉ số hiệu suất quan trọng (KPIs) như số lượng buổi học, tỷ lệ tham gia và mức độ hài lòng của sinh viên.

\end{itemize}
%========================================================================================
