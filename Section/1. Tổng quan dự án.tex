\part*{1. Tổng quan dự án}
\addcontentsline{toc}{part}{1. Tổng quan dự án}
Dự án "Hệ thống hỗ trợ Tutor tại Trường Đại học Bách khoa – ĐHQG TP.HCM" là một sáng kiến công nghệ nhằm mục tiêu hiện đại hóa và nâng cao hiệu quả của chương trình Tutor/Mentor. Đây là một chương trình có ý nghĩa quan trọng, được nhà trường triển khai nhằm hỗ trợ sinh viên phát triển một cách toàn diện cả về tri thức học thuật lẫn các kỹ năng cần thiết. Báo cáo này sẽ trình bày chi tiết về quá trình, từ việc phân tích bối cảnh, xác định yêu cầu, đến thiết kế và xây dựng hệ thống.

%========================================================================================

\section*{1.1. Giới thiệu dự án}
\addcontentsline{toc}{section}{1.1. Giới thiệu dự án}
Trong bối cảnh giáo dục đại học đang không ngừng đổi mới, việc ứng dụng công nghệ để tối ưu hóa các hoạt động hỗ trợ sinh viên là một yêu cầu tất yếu. Dự án này ra đời nhằm xây dựng một nền tảng phần mềm chuyên biệt, đóng vai trò xương sống cho chương trình Tutor, qua đó tạo một môi trường học tập tương tác, hiệu quả và có hệ thống tại Trường Đại học Bách khoa TP.HCM.

%========================================================================================

\subsection*{1.1.1. Mục đích}
\addcontentsline{toc}{subsection}{1.1.1. Mục đích}

Mục đích chính của dự án là phát triển một phần mềm quản lý tập trung, giúp vận hành chương trình Tutor một cách hiệu quả, hiện đại và có khả năng mở rộng. Hệ thống này sẽ là cầu nối vững chắc giữa Tutor (giảng viên, nghiên cứu sinh, sinh viên năm trên có thành tích tốt) và sinh viên cần hỗ trợ. Thông qua đó, dự án hướng đến việc nâng cao chất lượng học tập, tăng cường sự tương tác và gắn kết trong cộng đồng sinh viên, đồng thời góp phần vào việc phát triển kỹ năng mềm và định hướng nghề nghiệp cho người học.

%========================================================================================

\subsection*{1.1.2. Bối cảnh và lý do cần hệ thống}
\addcontentsline{toc}{subsection}{1.1.2. Bối cảnh và lý do cần hệ thống}
Trên thực tế, chương trình Tutor/Mentor tại Trường Đại học Bách khoa TP.HCM đã được triển khai và mang lại những lợi ích nhất định. Tuy nhiên, quy trình vận hành hiện tại vẫn còn phụ thuộc nhiều vào các phương pháp thủ công, dẫn đến một số thách thức và hạn chế đáng kể:
\begin{itemize}
    \item \textbf{Về quản lý thông tin và kết nối:} Việc quản lý hồ sơ năng lực của Tutor và nhu cầu cụ thể của sinh viên còn khó khăn, khiến quá trình ghép cặp chưa đạt được hiệu quả tối ưu. Sinh viên thường gặp khó khăn trong việc chủ động tìm kiếm và kết nối với người hướng dẫn phù hợp nhất.
    \item \textbf{Về tổ chức và sắp xếp:} Công tác lên lịch, thay đổi hoặc hủy các buổi học phụ thuộc nhiều vào việc trao đổi cá nhân, tiềm ẩn nguy cơ nhầm lẫn, thiếu sót và tốn nhiều thời gian không cần thiết.
    \item \textbf{Về đo lường và cải tiến:} Việc thiếu một công cụ theo dõi và đánh giá bài bản đã tạo ra một khoảng trống trong việc đo lường tiến bộ của sinh viên cũng như chất lượng của các buổi học, gây khó khăn cho việc cải tiến và nâng cao hiệu quả chương trình.
\end{itemize}
Trước những bất cập đó, việc xây dựng một "Hệ thống hỗ trợ Tutor" là một giải pháp cấp thiết. Bằng cách tự động hóa các quy trình từ quản lý, ghép cặp, lên lịch cho đến đánh giá, hệ thống được kỳ vọng sẽ giải quyết hiệu quả những tồn tại này, đáp ứng các đòi hỏi thực tiễn của môi trường giáo dục đại học trong kỷ nguyên số.

%========================================================================================

\subsection*{1.1.3. Kỳ vọng và mục tiêu}
\addcontentsline{toc}{subsection}{1.1.3. Kỳ vọng và mục tiêu}
Dự án được định hướng bởi những kỳ vọng và mục tiêu rõ ràng, hướng đến lợi ích của các bên liên quan:
\textbf{Kỳ vọng:}
\begin{itemize}
    \item \textbf{Đối với Sinh viên:}  Hệ thống được kỳ vọng sẽ trở thành một cổng thông tin thân thiện, giúp sinh viên dễ dàng tìm kiếm sự hỗ trợ học thuật, chủ động lựa chọn Tutor, linh hoạt sắp xếp lịch học và nhận được sự giúp đỡ kịp thời, đúng nhu cầu.
    \item \textbf{Đối với Tutor:} Cung cấp một bộ công cụ số hóa mạnh mẽ để quản lý thông tin cá nhân, sắp xếp lịch làm việc một cách khoa học, theo dõi và ghi nhận sự tiến bộ của sinh viên
    \item \textbf{Đối với Nhà trường:} Trao cho các Khoa và Phòng ban một công cụ quản lý tổng thể, cho phép giám sát, phân tích và đánh giá hiệu quả của chương trình. Dữ liệu thu thập được sẽ là cơ sở thực tiễn để tối ưu hóa việc phân bổ nguồn lực và đưa ra các quyết sách quan trọng, chẳng hạn như xét điểm rèn luyện hoặc học bổng.
\end{itemize}
\textbf{Mục tiêu:}
\begin{itemize}
    \item Phát triển một nền tảng phần mềm hoàn chỉnh, bao quát các chức năng cốt lõi: quản lý hồ sơ, đăng ký, ghép cặp, lên lịch, thông báo và đánh giá.
    \item Tích hợp liền mạch và an toàn với hạ tầng công nghệ thông tin hiện có của trường, bao gồm dịch vụ xác thực tập trung (HCMUT\_SSO), cơ sở dữ liệu lõi (HCMUT\_DATACORE) và thư viện số (HCMUT\_LIBRARY).
    \item Đảm bảo giao diện người dùng (UI/UX) thân thiện, trực quan, dễ sử dụng trên nhiều nền tảng, đồng thời thiết kế kiến trúc hệ thống theo hướng mở, sẵn sàng cho việc mở rộng và tích hợp các tính năng nâng cao trong tương lai.
\end{itemize}

%========================================================================================

\subsection*{1.1.4. Sản phẩm bàn giao}
\addcontentsline{toc}{subsection}{1.1.4. Sản phẩm bàn giao}
Kết thúc dự án, nhóm sẽ bàn giao các sản phẩm sau:
\begin{itemize}
    \item \textbf{Báo cáo phân tích yêu cầu phần mềm:} Tài liệu mô tả chi tiết các yêu cầu chức năng và phi chức năng của hệ thống, cùng với các biểu đồ Use-case.
    \item \textbf{Tài liệu thiết kế hệ thống:}  Bao gồm thiết kế kiến trúc tổng quan, thiết kế chi tiết các module, thiết kế giao diện người dùng (UI), và các biểu đồ liên quan như biểu đồ tuần tự, biểu đồ lớp.
    \item \textbf{Mã nguồn của ứng dụng (MVP - Minimum Viable Product):} Một phiên bản phần mềm có thể hoạt động được, bao gồm các chức năng cốt lõi đã được thống nhất.
    \item \textbf{Tài liệu hướng dẫn sử dụng:} Hướng dẫn chi tiết cho các đối tượng người dùng khác nhau như sinh viên, Tutor và quản trị viên.
    \item \textbf{Slide thuyết trình và video demo sản phẩm:} Trình bày tổng quan về dự án, các chức năng chính của hệ thống và demo cách thức hoạt động.
    \item \textbf{Báo cáo cuối kỳ:} Tổng hợp toàn bộ quá trình thực hiện dự án, từ việc phân tích yêu cầu, thiết kế, triển khai cho đến kết quả đạt được và những bài học kinh nghiệm.
\end{itemize}
