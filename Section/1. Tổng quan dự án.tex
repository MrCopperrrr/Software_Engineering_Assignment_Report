\part*{1. Tổng quan dự án}
\addcontentsline{toc}{part}{1. Tổng quan dự án}
Dự án "Hệ thống hỗ trợ Tutor tại Trường Đại học Bách khoa – ĐHQG TP.HCM" là một sáng kiến công nghệ nhằm mục tiêu hiện đại hóa và nâng cao hiệu quả của chương trình Tutor/Mentor. Đây là một chương trình có ý nghĩa quan trọng, được nhà trường triển khai nhằm hỗ trợ sinh viên phát triển một cách toàn diện cả về tri thức học thuật lẫn các kỹ năng cần thiết. Báo cáo này sẽ trình bày chi tiết về quá trình, từ việc phân tích bối cảnh, xác định yêu cầu, đến thiết kế và xây dựng hệ thống.

%========================================================================================

\section*{1.1. Giới thiệu dự án}
\addcontentsline{toc}{section}{1.1. Giới thiệu dự án}
Trong bối cảnh giáo dục đại học đang không ngừng đổi mới, việc ứng dụng công nghệ để tối ưu hóa các hoạt động hỗ trợ sinh viên là một yêu cầu tất yếu. Dự án này ra đời nhằm xây dựng một nền tảng phần mềm chuyên biệt, đóng vai trò xương sống cho chương trình Tutor, qua đó tạo một môi trường học tập tương tác, hiệu quả và có hệ thống tại Trường Đại học Bách khoa TP.HCM.